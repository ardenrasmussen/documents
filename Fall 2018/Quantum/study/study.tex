\documentclass[10pt]{amsart}
\usepackage{amsmath}
\usepackage{csquotes}
\usepackage{enumitem}
\usepackage{physics}

\usepackage[marign=1in]{geometry}
\usepackage{multicol}

\title{Quantum Mechanics Midterm 2}
\author{Arden Rasmussen}
\date{\today}

\begin{document}
\maketitle
\begin{multicols}{2}

\section{Time-Independent Schr\"odinger Equations}%
\label{sec:time_independent_schr"odinger_equations}

\subsection{Harmonic Oscillator}%
\label{sub:harmonic_oscillator}

\subsubsection{Base State}%
\label{ssub:base_state}

\begin{align*}
  \Psi_0&={\left(\frac{m\omega}{\pi\hbar}\right)}^{\frac{1}{4}}e^{-\frac{m\omega}{2\hbar}x^2}\\
  E_0&=\frac{\hbar\omega}{2}
\end{align*}

\subsubsection{Ladder Operators}%
\label{ssub:ladder_operators}

\begin{align*}
  \hat{a_{+}}&=\frac{1}{\sqrt{2m\omega\hbar}}\left[-i\hat{p}+m\omega x\right]\\
  \hat{a_{-}}&=\frac{1}{\sqrt{2m\omega\hbar}}\left[i\hat{p}+m\omega x\right]\\
  \left[\hat{a_{-}},\hat{a_{+}}\right]&=1\\
\end{align*}

\subsubsection{Solving}%
\label{ssub:solving}

\begin{align*}
  \hat{a_{+}}\psi_n&=\sqrt{n+1}\psi_{n+1}\\
  \hat{a_{-}}\psi_n&=\sqrt{n}\psi_{n-1}\\
  \psi_n&=\frac{1}{\sqrt{n!}}{\left(\hat{a_{+}}\right)}^n\psi_0\\
  E_n&=\left(n+\frac{1}{2}\right)\hbar\omega
\end{align*}

\subsubsection{Operators}%
\label{ssub:operators}

\begin{align*}
  \hat{x}=\sqrt{\frac{\hbar}{2m\omega}}\left(\hat{a_{+}}+\hat{a_{-}}\right)\\
  \hat{p}=i\sqrt{\frac{\hbar m \omega}{2}}\left(\hat{a_{+}}-\hat{a_{-}}\right)
\end{align*}

\subsection{Free Particle}%
\label{sub:free_particle}

Here be dragons

\section{Formalism}%
\label{sec:formalism}

\subsection{Hilbert Space}%
\label{sub:hilbert_space}

All wave functions live in Hilbert space. Any function in that Hilbert space
can be represented as a linear combination of basis functions of that Hilbert
space.

\subsection{Observables}%
\label{sub:observables}

The $\left<O\right>$ is always real for observables. Observables are
represented by hermitian operators.

\begin{align*}
  \left<O\right>=\expval{\hat{O}}{\Psi}
\end{align*}

The eigenvalues of hermitian operators is always real.

\subsection{Uncertainty Principle}%
\label{sub:uncertainty_principle}

\begin{align*}
  \sigma_A^2\sigma_B^2\geq {\left(\frac{1}{2i}\expval{\left[\hat{A},\hat{B}\right]}\right)}^2
\end{align*}

Observables whose operators do not commute car called incompatible observables,
meaning that measuring one changes the other.

\subsection{Dirac Notation}%
\label{sub:dirac_notation}

We can express functions as a basis vector in Hilbert space.

\begin{align*}
  \ket{\alpha}=\begin{pmatrix}
    \alpha_1\\
    \alpha_2\\
    \vdots \\
    \alpha_n
  \end{pmatrix}\quad
  \bra{\alpha}=\begin{pmatrix}
    \alpha_1^*  \alpha_2^*  \cdots  \alpha_n^*
  \end{pmatrix}
\end{align*}

The identity operator is

\begin{align*}
  \hat{I}=\sum\ket{\alpha}\bra{\alpha}
\end{align*}

For some operator $\hat{O}$ and eigenvalues $o_n$

\begin{align*}
  \hat{O}\ket{n}=o_n\ket{n}
\end{align*}

\subsubsection{Matrix Operators}%
\label{ssub:matrix_operators}

\begin{align*}
  O_{nm}\equiv\matrixel{n}{\hat{O}}{m}
\end{align*}

Hermitian if

\begin{align*}
  O_{nm}=O_{mn}^*
\end{align*}

\subsubsection{Harmonic Oscillator}%
\label{ssub:harmonic_oscillator}

\begin{align*}
  \hat{a_{+}}\ket{n}&=\sqrt{n+1}\ket{n+1}\\
  \hat{a_{-}}\ket{n}&=\sqrt{n}\ket{n-1}
\end{align*}

\subsubsection{Wave Function}%
\label{ssub:wave_function}

\begin{align*}
  \ket{\Psi}&=\sum_nA_n\ket{n}\\
  A_n&=\braket{n}{\Psi}\\
  \ket{\Psi}&=\sum_n\braket{n}{\Psi}\ket{n}
\end{align*}
   
\end{multicols}
\end{document}
