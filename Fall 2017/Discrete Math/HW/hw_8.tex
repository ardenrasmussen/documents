\documentclass{article}

\usepackage{amsthm}
\usepackage{amsmath}

% \renewcommand\qedsymbol{$\blacksquare$}

\title{HW.8}
\author{Arden Rasmussen}

\newtheoremstyle{problemstyle}
        {3pt}
        {3pt}
        {\normalfont}
        {-0.5in}
        {\bfseries}
        {\normalfont\bfseries:\\}
        {.5em}
        {}
\theoremstyle{problemstyle}
\newtheorem{problem}{Problem}


\begin{document}
\maketitle

\begin{problem}
   
Prove that for all positive integers $n$ that

\begin{align*}
  \sum_{i=1}^{n}i^2=1+4+9+\cdots + {(n-1)}^2+n^2 = \frac{n(n+1)(2n+1)}{6}
\end{align*}

\begin{proof}
  We will prove by induction. We will prove the statement
  \begin{align*}
    P(n):\quad \sum_{i=1}^{n}i^2=1+4+9+\cdots + {(n-1)}^2+n^2 = \frac{n(n+1)(2n+1)}{6}
  \end{align*}
  
  \textbf{Base Case:} First we will prove for $P(1)$
  \begin{align*}
    1^2 &= \frac{1(1+1)(2(1)+1)}{6}\\
    1 &= \frac{1(2)(3)}{6}\\
    1 &= \frac{6}{6}\\
    1 &= 1
  \end{align*}
  Thus we see that the case for $P(1)$ is true.

  \textbf{Inductive Hypothesis:} Assume $P(k)$ is true for some $k\ge1$.
  \begin{align*}
    \sum_{i=1}^{k}i^2=1+4+9+\cdots + {(k-1)}^2+k^2 = \frac{k(k+1)(2k+1)}{6}
  \end{align*}

  \textbf{Inductive Step:} Prove $P(k+1)$ using the \textit{inductive
  hypothesis}. So
  \begin{align*}
    1+4+9+\cdots + k^2+{(k+1)}^2 &= \frac{(k+1)((k+1)+1)(2(k+1)+1)}{6}\\
    \frac{k(k+1)(2k+1)}{6} + {(k+1)}^2 &= \frac{(k+1)(k+2)(2k+3)}{6}\\
    \frac{2k^3+3k^2+k}{6} + \left(k^2+2k+1\right) &= \frac{2k^3+9k^2+13k+6}{6}\\
    \frac{2k^3+3k^2+k+6k^2+12k+6}{6} &= \frac{2k^3+9k^2+13k+6}{6}\\
    \frac{2k^2+9k^2+13k+6}{6} &= \frac{2k^3+9k^2+13k+6}{6}
  \end{align*}
  Thus we see that the case of $P(k+1)$ is true if the case of $P(k)$ is true.
  Thus we have shown that
  \begin{align*}
    \sum_{i=1}^{n}i^2=1+4+9+\cdots + {(n-1)}^2+n^2 = \frac{n(n+1)(2n+1)}{6}
  \end{align*}
  is true for any positive interger $n$.
\end{proof}
\end{problem}

\begin{problem}
   
Prove that for all non-negative integers $n$ that

\begin{align*}
  \sum_{i=0}^{n}3^i = 1+3+9+\cdots+3^{n-1}+3^n=\frac{3^{n+1}-1}{2}
\end{align*}

\begin{proof}
  We will prove by induction. We will prove the statement
  \begin{align*}
    P(n):\quad \sum_{i=0}^{n}3^i = 1+3+9+\cdots+3^{n-1}+3^n=\frac{3^{n+1}-1}{2}
  \end{align*}

  \textbf{Base Case:} First we will prove for $P(0)$.
  \begin{align*}
    3^0 &= \frac{3^{0+1}-1}{2}\\
    1 &= \frac{3^1-1}{2}\\
    1 &= \frac{3-1}{2}\\
    1 &= \frac{2}{2}\\
    1 &= 1
  \end{align*}
  We see that the equality is true for $P(0)$.

  \textbf{Inductive Hypothesis:} Assume $P(k)$ is true for some $k\ge0$.
  \begin{align*}
    \sum_{i=0}^{k}3^i = 1+3+9+\cdots+3^{k-1}+3^k=\frac{3^{k+1}-1}{2}
  \end{align*}

  \textbf{Inductive Step:} Prove $P(k+1)$ using the \textit{inductive
  hypothesis}.
  \begin{align*}
    1+3+9+\cdots+3^k+3^{k+1} &= \frac{3^{(k+1)+1}-1}{2}\\
    \frac{3^{k+1}-1}{2}+3^{k+1} &= \frac{3^{k+2}-1}{2}\\
    \frac{3^{k+1}-1+2\cdot3^{k+1}}{2} &= \frac{3^{k+2}-1}{2}\\
    \frac{3\cdot3^{k+1}-1}{2} &= \frac{3^{k+2}-1}{2}\\
    \frac{3^{k+1+1}-1}{2} &= \frac{3^{k+2}-1}{2}\\
    \frac{3^{k+2}-1}{2} &= \frac{3^{k+2}-1}{2}
  \end{align*}
  Thus we see that the case of $P(k+1)$ is true. So we conclude that
  \begin{align*}
    \sum_{i=0}^{n}3^i = 1+3+9+\cdots+3^{n-1}+3^n=\frac{3^{n+1}-1}{2}
  \end{align*}
  is true for any non-negative integer $n$.
\end{proof}
\end{problem}

\begin{problem}
   
Prove that the sum of the first $n$ odd numbers is $n^2$. That is, prove
\begin{align*}
  \sum_{i=1}^{n}(2i-1) = 1 + 3 + 5 + \cdots + 2n-1 = n^2
\end{align*}

\begin{proof}
   We will prove by induction. We will prove the statement
   \begin{align*}
     P(n):\quad \sum_{i=1}^{n}(2i-1) = 1 + 3 + 5 + \cdots + 2n-1 = n^2
   \end{align*}

   \textbf{Base Case:} First we will prove for $P(1)$.
   \begin{align*}
     1 &= 1^2\\
     1 &= 1
   \end{align*}
   We see that the case for $P(1)$ is true.

   \textbf{Inductive Hypothesis}: Assume $P(k)$ is true for some $k\ge1$.
   \begin{align*}
     \sum_{i=1}^{k}(2i-1) = 1 + 3 + 5 + \cdots + 2k-1 = k^2
   \end{align*}

   \textbf{Inductive Step:} Prove $P(k+1)$ using the \textit{inductive
   hypothesis}.
   \begin{align*}
   1+3+5+\cdots+2(k+1)-1 &= {(k+1)}^2\\
   k^2 + 2k+2-1 &= k^2+2k+1\\
   k^2+2k+1 &= k^2+2k+1
   \end{align*}
   Thus we see that the case of $P(k+1)$ is true. So we can conclude that 
   \begin{align*}
     \sum_{i=1}^{n}(2i-1) = 1 + 3 + 5 + \cdots + 2n-1 = n^2
   \end{align*}
   is true for all intergers $n$.
\end{proof}
\end{problem}

\begin{problem}
  Let $f_1,f_2,f_3,\ldots,f_n$ be functions. Let $g_0(x)=x,g_1=f_1$, and
  $g_i=f_i\circ g_{i-1}$ for all integers $i \ge 2$. Prove that for all integers
  $n \ge 1$,
  \begin{align*}
    g_n'(x) &= \prod_{i=1}^{n}f_i'(g_{i-1}(x))
  \end{align*}
  \begin{proof}
     We will prove by induction. We will prove the statement
     \begin{align*}
       P(n):\quad g_n'(x) &= \prod_{i=1}^{n}f_i'(g_{i-1}(x))
     \end{align*}

     \textbf{Base Case:} First we will prove for $P(1)$
     \begin{align*}
       g_2'(x) &= \prod_{i=1}^{2}f_i'(g_{i-1}(x))\\
     \end{align*}
  \end{proof}
\end{problem}

\end{document}
