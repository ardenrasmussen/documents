\documentclass[../main.tex]{subfiles}

\begin{document}
\section{Direct and Indirect Proofs in Number Theory}
\label{sec:direct_and_indirect_proofs_in_number_theory}

\begin{description}
  \item[Contrapositive] of $P\to Q$ is $\neg Q \to \neg P$ (Switch order and
    negate both).
  \item[Definition of divides] If there is an integer $q$ such that $n=kq$,
    then $k|n$.
  \item[Division Algorithm] For $m,n\in\mathbb{Z} $ there is a unique pair
    $r,q\in\mathbb{Z}$ such that $n=mq+r$ where $m>0$ and $0 \leq r < m$.
  \item[Odd and Even] Odd if there is a $q\in\mathbb{Z}$ such that $n=2q+1$, it
    is odd. Event if there is a $q\in\mathbb{Z}$ such that $2q=n$, it is even.
  \item[Prime and Ireducible] $p$ is prime if got $a,b\in\mathbb{Z}$, $p|ab\to
    p|a \text { or } p|b$. $p$ is irreducible if $p=ab\to a=1\text{ or }b=1$.If
    $p$ is prime,it is irreducible.
\end{description}   

\section{Induction}
\label{sec:induction}

Use induction when dealing with natural numbers, often in a series or
repetitive numerical situation. Always include \textit{Base case},
\textit{Inductive hypothesis} (for $k\geq BC\#$), \textit{Inductive Step} (use
IH to prove $k+1$). Start with claim $P(n)$. No backward proofs.

\section{Greatest Common Divisors}
\label{sec:greatest_common_divisors}

\begin{description}
  \item[Definition] Let $m,k\in\mathbb{Z}$. $d$ is the $GCD(m,k)$ if
    \begin{enumerate}[(i)]
      \item $d|m$
      \item $d|k$
      \item if $a\in\mathbb{Z}$ and $a|m$ and $a|k$, then $a \leq d$.
    \end{enumerate}
  \item[Euclidean Algorithm] Use \textit{Division algorithm} until you get
    $r=0$. Go backwards to express $GCD$ s linear combination of the two
    numbers.
  \item[GCD pre-hammer] Given $a,b\in\mathbb{Z}$, $\exists x,y\in\mathbb{Z}\to
    GCD(a,b)=xa+yb$.
  \item[GCD hammer] $a$ and $b$ are relatively prime if and only if $\exists
    x,y\in\mathbb{Z}$ such that $xa+yb=1$.
  \item If $d|ab$ and $GCD(a,d)=1$ then $d|b$.
\end{description}

\section{The Fundamental Theorem of Arithmetic}
\label{sec:the_fundamental_theorem_of_arithmetic}

Let $n>1$, $n\in\mathbb{Z}$. Then $n$ has a prime factorization, $n=p_1\cdot
p_2\cdots p_k$ where each $p_i$ is prime. This factorization is \textit{unique
up to reordering the factors}. The $LCM = \frac{ab}{GCD}$ where $a,b$ are two
integers we are concerned with.

\section{Set Theory Basics}
\label{sec:set_theory_basics}

\begin{description}
  \item[Set] A set is a collection of elements $\{a, b, c\}$.
  \item[Union] $A\cup B$, the combination of all the elements from both $A$ and
    $B$, without repeating elements.
  \item[Intersection] $A\cap B$, The set of elements that both $A$ and $B$ have
    in common.
  \item[Cardinality] $|A|$ or $card(A)$, The number of elements in a set.
  \item[Empty Set] The set with no elements is the empty set $\emptyset$ or
    $\{\}$.
  \item[Power Set] $\mathcal{P}(S)$, The set of all possible subsets of $S$.
  \item[Compliment] The set of elements in the universe $U$ that are not
    elements of the original set.
\end{description}

\section{Equivalence Relations}
\label{sec:equivalence_relations}

\begin{description}
  \item[Relation] Given sets $A$ and $B$, a relation is a subset of $A\times B$.
  \item[Equivalence Relation] A relation is an equivalence relation if it is
    \begin{description}
      \item[Reflexive] $\forall a\in A, aRa$.
      \item[Symmetric] $aRb \to bRa$.
      \item[Transitive] $aRb \text{ and } bRc \to aRc$.
    \end{description}
  \item[Partition] Let $A$ be a nonempty set. Define $P=\{A_1, A_2, \ldots,
    A_n\}$ where
    \begin{enumerate}[(i)]
      \item For $i$ from $i=1,2,\ldots,n$ $A_i\subseteq A$.
      \item For $i=1,2,\ldots,n$, $A_i\neq\emptyset$.
      \item For pairs $i,j$ with $i\neq j$, $A_i\cap A_j = \emptyset$ (Each
        subset is unique).
      \item $A_1\cup A_2\cup \cdots\cup A_n=A$.
    \end{enumerate}
    Such a $P$ is a partition. $A_1,A_2,\ldots,A_n$ are the partition elements.
    Each is an equivalence class ${\left[a\right]}_R$ under a relation $R$.
    Each partition gives an equivalence relation.
\end{description}

\section{Functions}
\label{sec:functions}

\begin{description}
  \item[Function] A function $f$ is a relation from $A\mapsto B$ for which each
    element of $A$ appears once and only once in the first coordinates of an
    ordered pair in $f$.
  \item[Injective] $f$ is injective (one-to-one) if $\forall x_1, x_2\in A,
    f\left(x_1\right)=f\left(x_2\right)\implies x_1 = x_2$. If every element in
    the domain maps to a unique value in the co domain.
  \item[Surjective] $f$ is surjective (onto) if $\forall y\in B, \exists x\in
    A, f(x) = y$. If every element in co domain is mapped to by at least one
    element in the domain.
  \item[Bijection] A function is a bijection is it is both injective and
    surjective.
  \item[Composition] Let $f,g$ be functions, the composition of these function
    is defined as $f\circ g \equiv g\left(f\left(x\right)\right)$.
\end{description}

\end{document}
