\documentclass[../main.tex]{subfiles}
\begin{document}
\section{The Pigeonhole Principle}
\label{sec:the_pigeonhole_principle}

Suppose $|A|=n$ and $|B|=m$ and $n>m$. Then the function $f:A\mapsto B$ cannot
be injective. There are too many ``pigeons'' and not enough ``holes''.

\section{The Inclusion/Exclusion Principle}
\label{sec:the_inclusion_exclusion_principle}

If $S_1,S_2,\ldots,S_n$ are all finite sets and $n\geq 2$, then
\begin{align}
  \left|S_1\cup S_2 \cup \cdots \cup S_n\right| &=
  |S_1|+|S_2|+\ldots+|S_n|\\
  &-|S_1\cap S_2| - |S_1\cap S_3| -\ldots -|S_{n-1}\cap S_n|\\
  &+\bullet \bullet \bullet\\
  &+{\left(-1\right)}^{n-1}\left|S_1\cup S_2\cup \cdots \cup S_n\right|
\end{align}
\begin{align}
  \left|\bigcup_{k=1}^{n}S_k\right|=\sum_{k=1}^{n}\left|S_k\right| - \sum_{1\leq k
  < j \leq n}^{n}\left|S_i \cap S_j\right| + \ldots +
  {\left(-1\right)}^{n-1}\left|\bigcap_{k=1}^{n} S_k\right|
\end{align}

\section{Graph Theory}
\label{sec:graph_theory}

\begin{description}
  \item[Graph] A graph is a set $V$ of objects along with a set $E$ of
    unordered two element subsets of $V$. Elements of $V$ are
    \textit{vertices} and elements of $E$ are \textit{edges}.
  \item[Degree] The degree of a vertex is the number of elements in $E$ that
    contain that vertex. The sum of degrees must always be even, as each edge
    connects exactly two vertexes.
  \item[Degree Sequence] The degree sequence of a graph is the list of degrees
    of vertexes of a graph in increasing order.
  \item[Walk] Walks are routes through a graph that connect two vertices and can repeat
    edges and vertexes.
  \item[Trail] Trails are walks that cannot repeat edges.
  \item[Path] Paths are walks that cannot repeat edges or vertexes.
  \item[Circuit] A circuit is a closed trail, (one which ends at the beginning
    vertex).
  \item[Connected] A graph is connected if any 2 vertices have a path in
    between them.
  \item[Component] A component of a graph is the largest connected piece of a
    graph where you include all connected vertices.
  \item[Tree] A tree is a connected graph with no circuit's.
  \item[Forest] A forest is a disconnected graph with no circuit's, or a group
    of disconnected trees.
  \item[Bridge] A bridge is an edge that connects two otherwise disconnected
    components of a graph. If a graph $G$ has a circuit, then there is an edge
    in $G$ that is not a bridge. If you remove an edge from a connected graph,
    you split it into at most two components.
  \item[Equivilant Definition for Finite Tree] If $T$ is a graph with $n$
    vertices, then the following are equivalent:
    \begin{enumerate}[(1)]
      \item $T$ is a tree.
      \item $T$ has no circuits and $n-1$ edges.
      \item $T$ is connected and has $n-1$ edges.
      \item $T$ is connected and removal of an edge disconnects $T$.
      \item There exists a unique path between any vertices in $T$.
      \item $T$ has no circuits, and adding any edge creates one.
    \end{enumerate}
  \item[Minimal Spanning Tree] A minimal spanning tree, is a tree that connects
    a graph using as minimal weight as possible. To create a minimal spanning
    tree
    \begin{enumerate}
      \item Chose the edge with minimum weight.
      \item Look at the remaining edges. Chose edge with minimum weight that does
        not create a circuit.
      \item Stop when you have $n-1$ edges.
    \end{enumerate}
  \item[Planar] A graph is planar if a diagram can be drawn of it where no
    edges cross.
    \begin{itemize}
      \item To prove a graph is planar, can produce a planar drawing of it.
      \item Hard to prove it isn't, best to argue by contradiction with
        \textit{Euler characteristic}.
    \end{itemize}

  \item[Euler Characteristic] $v-e+f=2$ for a connected planar graph.
\end{description}

\end{document}
