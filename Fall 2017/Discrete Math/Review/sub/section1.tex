\documentclass[../main.tex]{subfiles}
\begin{document}
\section{Counting}
\label{sec:counting}

\begin{description}
  \item[The addition principle] $M_1+M_2+M_3=\# ways$, where $M_i$ are
    different methods for the same task.
  \item[The multiplication principle] $t_1\cdot t_2\cdot t_3 = \# ways$, where
    $t_i$ are different tasks to be completed.
  \item[Indirect Method] Count the total number of ways, and subtract the
    unwanted ways that were counted.
  \item[Combinations] A collection of $k$ objects chosen from $n$ distinct
    objects \textit{without regard to the order}.
    \begin{align}
      C(n, k) = \begin{pmatrix}n\\k\end{pmatrix} = \frac{n!}{k!\left(n-k\right)!}
    \end{align}
  \item[Permutations] An \textit{arrangement} of $k$ objects out of a
    collection of $n$ distinct objects \textit{where order matters}.
\end{description}

\section{Basic Logic}
\label{sec:basic_logic}

\begin{description}
  \item[Negation] $\neg P$, not $P$.
  \item[Conjunction] $P\wedge Q$, $P$ and $Q$.
  \item[Disjunction] $P\vee Q$, $P$ or $Q$.
  \item[Conditional] $P\to Q$, if $P$ then $Q$.
  \item[Biconditional] $P\leftrightarrow Q$, $P$ if and only if $Q$.
  \item[Logically Equivalent] $P\iff Q$.
  \item[Implies] $P\implies Q$.
\end{description}

\textbf{Note} that the \textit{Conditional} and \textit{Implies} are often mixed,
as well as the \textit{Biconditional} and \textit{logically equivalent}.

\begin{description}
  \item[Distributive Laws]
    \begin{align}
      P\vee\left(Q\wedge R\right) &\iff \left(P\vee Q\right)\wedge\left(P\vee
      R\right)\\
      P\wedge\left(Q\vee R\right) &\iff \left(P\wedge Q\right)\vee\left(P\wedge
      R\right)
    \end{align}
  \item[Conditional]
    \begin{align}
       P \to Q \iff \neg P \vee Q
    \end{align}
  \item[De Morgan's Laws]
    \begin{align}
      \neg\left(P\wedge Q\right) &\iff \neg P \vee \neg Q\\
      \neg\left(P\vee Q\right) &\iff \neg P \wedge \neg Q
    \end{align}
  \item[Parenthesies]
    \begin{align}
      \left(P \vee Q\right)\vee R &\iff P\vee Q\vee R\\
      P\wedge \left(Q\wedge R\right) &\iff P\wedge Q\wedge R
    \end{align}
  \item[Converse] $P\to Q : Q\to P$.
  \item[Inverse] $P\to Q : \neg P \to \neg Q$.
  \item[Contrapositive] $P\to Q : \neg Q \to \neg P$.
\end{description}

\end{document}
