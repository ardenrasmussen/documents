\documentclass{report}

\usepackage{amsthm}
\usepackage{mathtools}

\newtheorem{theorem}{Theorem}[section]
\newtheorem{definition}{Definition}[section]

\title{Discrete Mathematics}
\author{Arden Rasmussen}
\date{\today}

\begin{document}

\chapter{Counting Sheep, Among Other Things}
\label{cha:counting_sheep_among_other_things}

\section{Learning to Count}
\label{sec:learning_to_count}

\begin{description}
  \item[The Addition Principle] \hfill \\
    Suppose there is a task, $T$, to be completed and there are numerous
    different methods, $M_1, M_2, \dots,M_k$ which can be used to complete
    $T$. If method $M_1$ can be accomplished in $a_1$ ways, method $M_2$ in
    $a_2$ ways,‥, and method $M_k$ in $a_k$ ways, and $M_1, M_2,\dots,M_k$
    are mutually exclusive, then $T$ can be compleated in $a_1+a_2+\dots+a_k$
    ways.

  \item[The Multiplication Principle] \hfill \\
    Suppose there is a task, $T$,  to be completed, but now the task can be
    completed in stages. That is, $T$ can be broken down into subtasks,
    $t_1,t_2,\dots,t_m$ so that $T$ is accomplished only after all of the sub
    tasks have been completed. If $t_1$ can be done $b_1$ ways, $t_2$ in $b_2$
    ways, ‥, and $t)m$ in $b)m$ ways. Then $T$ can be completed in $b_1 \cdot
    b_2 \cdot \dots \cdot b_m$ ways.
\end{description}

\section{Permutations}
\label{sec:permutations}

\begin{definition}
  The notation $n!$, which we read as \textit{n factorial}, is defined as $n! =
  n\cdot(n-1)\cdot(n-2)\cdot \dots \cdot 3 \cdot 1 \cdot 1$.
\end{definition}

\begin{definition}
  An arrangment of $n$ objects is called a \textit{permutation} of the objects.
\end{definition}

\begin{definition}
   An arrangement of $r$ distinct objects out of a collection of $n$ distinct
   objects is called an \textit{r-permutation} of the $n$ objects.
\end{definition}

\begin{theorem}
  The number of r-permutations of $n$ objects is $\frac{n!}{(n-r)!}$. This is
  often denoted $P(n,r)$ or $\prescript{}{n}{P}_r$.
\end{theorem}

\section{Combinations}
\label{sec:combinations}

\begin{definition}
   A collection of $r$ objects chosen from $n$ distinct objects without regard
   to the order of the objects is called an \textit{r-combination} of the $n$
   objects.
\end{definition}

\begin{theorem}
  The number of r-combinations of $n$ objects is $\frac{n!}{(n-r)!\cdot r!}$.
  This number is denoted $C(n,r)$ or $\prescript{}{n}{C}_r$ or
  $\left(\begin{array}{c}
    n \\
    r \\
  \end{array}\right)$.
\end{theorem}

\end{document}
