\documentclass[10pt]{article}
\usepackage[backend=bibtex]{biblatex}
\addbibresource{abstract.bib}
\title{Four Color Theorem}
\author{Arden Rasumssen \and Jerome Regovich}

\begin{document}
\maketitle
\begin{abstract}

  % The four color theorem, was first proposed by Guthrie in 1852. The four color
  % theorem states that a map can be colored using four colors in such a way that
  % regions sharing a common edge do not share the same color. After the question
  % first appeared, there were many attempts to prove it, but none were
  % successful until Kenneth Appel and Wolfgang Haken. They were able to prove
  % that any map can be colored by only for colors, by looking at a set of 1,936
  % maps and showed that those maps cannot be part of a smallest-sized counter
  % example, and that any map that could be a counter example must contain one of
  % these maps. Thus they were able to conclude that no smallest counterexample
  % exists, as it must contain, and must not contain one of the 1,936 maps. In
  % order to analyze the maps, they were required to use computer assisted
  % software to check that the four color theorem held for each of the maps.
  % Since the first proof, the proof has been optimized more, and many attempts
  % at a disproof have been made. This proof was extremely significant as being
  % one of the first major proofs to be done with computer assistance. It is
  % important to note that the four color theorem does not provide much other
  % than an interesting proof, as it is not useful to cartographers.

  In this presentation we will look at the historical significance of the Four
  Color Theorem and the proofs for it. The Four Color Theorem states that any
  map (or planar graph) can be colored with only four colors no matter how
  complex the borders, so that no two neighboring faces share a color. The
  first proof was made by Kenneth Appel and Wolfgang Haken. They proved the
  theorem with computers. The fact that a computer did much of the work caused
  a lot of controversy, as it was impossible for a human to check that the
  computer did its work properly. In the presentation we will discuss the proof
  by Appel and Haken in further detail. In summary, they utilized a computer to
  prove it for a set of 1,936 maps, and used this to demonstrate that it must
  work for all maps. Since the first proof, the solution has been optimized but
  as it the necessity of computer assistance is still necessary. The
  significances of this proof, is that it is the first major proof made with
  computer assistance. Thus it opened the possibility for using computers in
  more proofs.

\end{abstract}
\nocite{*}
\printbibliography{}
\end{document}
