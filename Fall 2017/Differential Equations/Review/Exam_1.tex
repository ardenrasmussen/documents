\documentclass{article}

\usepackage[margin=1in]{geometry}
\usepackage{amsmath}
\usepackage{csquotes}

\title{Differential Equations (Exam \#1)}
\author{Arden Rasmussen}
\date{1 October 2017}
\linespread{1}

\begin{document}
\pagenumbering{gobble}
\maketitle
\newpage
\tableofcontents
\newpage
\pagenumbering{arabic}

\section{Summary}\label{sec:summary}

\begin{itemize}
  \item Mixing Problems
    \begin{itemize}
      \item Setup
      \item Analyze
    \end{itemize}
  \item Find equilibrium solutions and preform qualitative analysis.
  \item Find solution using analytic methods
  \item Fundamental Theorem
    \begin{itemize}
      \item Solution curves don't cross unless RHS turns \textit{ugly} on you.
    \end{itemize}
  \item Euler method
    \begin{itemize}
      \item Within reason
    \end{itemize}
  \item Slope field
    \begin{itemize}
      \item Not much of number crunching
    \end{itemize}
  \item Some discussion about a parameter
\end{itemize}
\newpage

\section{Reading \& Writing Differential Equations}\label{sec:reading_writing_differential_equations}

The form of a differential equation must be the rate equals some equation that
incorporates the equation.

\begin{align}
  \frac{dy}{dt} &= f(t,y(t))
\end{align}

Where $f(t,y(t))$ is some function that involves $y(t)$ and optional involves
$t$.

\subsection{Exponential Model}\label{sub:exponential_model}

\textbf{Idea:} Relative growth rate is some constant: $k$.

\begin{align}
  \frac{dy}{dt} &= k \cdot y(t)
\end{align}

\subsection{Logistic Model}\label{sub:logistic_model}

\textbf{Idea:} The relative growth rate may not be constant, and may depend on
circumstances. It may depend on $y(t)$ itself. Where $M$ is the maximum limit
for $y(t)$.

\begin{align}
  \frac{dy}{dt} &= ky(t)\left(1-\frac{y(t)}{M}\right)
\end{align}

\section{The Fundamental Theorem of ODE'S}\label{sec:the_fundamental_theorem_of_ode_s}

Given a differential equation of the form:

\begin{align}
  \frac{dy}{dt} &= f(t,y)
\end{align}

There are many solutions to a single differential equation. The solution is
pinned to a single equation when the initial condition $y_0$ is set to a value.

\subsection{Initial Value Problems}\label{sub:initial_value_problems}

Initial value problems (IVP) come in the form:

\begin{align}
  \left\{ \begin{array}{l}
      y\left(0\right) = y_0\\
      \frac{dy}{dt} = f(t,y)
    \end{array}\right.
\end{align}

\textit{Expect} to have exactly one solution. Suppose $f(t,y)$ is a function
which:

\begin{itemize}
  \item Is continuous in a neighborhood of  $(t_0,y_0)$.
  \item Has continuous $\frac{\partial f}{\partial y}$ in a neighborhood of
    $(t_0,y_0)$
\end{itemize}

Then the IVP as exactly one solution. If either of these conditions are not
met, then the fundamental theorem of ODE'S tells us nothing about the solution
to the IVP.\@ A General rule of thumb is:

\begin{displayquote}
  If $f(t,y)$ is nice then there is exactly one solution, if solution is messy,
  then $f(t,y)$ is not nice.
\end{displayquote}

\textbf{Note} that this is only in the neighborhood of $(t_0,y_0)$, so this
says nothing about the long term existence or uniqueness. This only shows
short term existence and uniqueness.

\subsection{Method of Barriers}\label{sub:method_of_barriers}

If the solutions for $f(t,y)$ are always nice, then no solution curves can
cross or intersect, so any solution is stuck between other solution curves.
Using equilibrium solutions (\ref{sub:equilibrium_solutions}) causes solutions to be sandwiched between these
easer to find solutions.

\section{Analytic Methods}\label{sec:analytic_methods}

Analytic methods are used to find the exact equation for the differential
equation.

\subsection{Separable Equations}\label{sub:separable_equations}

Separable equations works by separating the terms for $y$ and $t$ in the
differential equation to different sides of the equation. Given an equation of
the form:

\begin{align}
  \frac{dy}{dt} &= A(y) \cdot B(t)
\end{align}

This method can be applied in the following format.

\begin{align}
  \frac{1}{A(y)}\cdot\frac{dy}{dt} &= B(t)\label{eq:a:se:1}\\
  \int^{s=t}_{s=t_0} \frac{1}{A(y)} \cdot \frac{dy}{ds}ds &=
  \int^{s=t}_{s=t_0}B(s)ds\\
  \int^{y=y(t)}_{y=y_0} \frac{1}{A(y)} dy &= \int^{s=t}_{s=t_0} B(s)ds\\
  f\left(y(t), y_0\right) &= g\left(t, t_0\right)\\
  y(t) &= g\left(t, t_0, y_0\right)
\end{align}

\textbf{Note} that due to [\ref{eq:a:se:1}] the case when $A(y) = 0$ must be
checked to see if it is a solution. And $f$ and $g$ are arbitrary functions
that are the results of the integration, they merely are involving int
parameters.

\subsection{Propagators}\label{sub:propagators}

Propagators uses the method of propagating the initial value, and any
additional values added. This method works when given an equation of the form:

\begin{align}
  \frac{dy}{dt} &= r(t)s(t) + f(t)
\end{align}

The first step is to identify the propagator function.

\begin{align}
  P(t, t_0) &= \exp\left(\int^{s=t}_{s=t_0}r(s)ds\right)
\end{align}

Using the propagator, the solution for $y(t)$ is of the form:

\begin{align}
  y(t) &= P(t,t_0) \cdot y_0 + \int^{s=t}_{s=t_0} P(t, s) \cdot f(s) ds
\end{align}

\paragraph{Some notes on the Propagator}\label{par:some_notes_on_the_propagator}

\begin{align}
  P(a, b) \cdot P(b,a) &= 1\\
  P(a, b) \cdot P(b, c) &= P(a,c)
\end{align}

\section{Numeric Methods}\label{sec:numeric_methods}

Numeric methods are focused on getting approximate numeric values, without
actually solving for the equation like analytic methods.

\subsection{Euler's Method}\label{sub:euler_s_method}

Euler's method uses crating a table to approximate the value of $y(t)$ for
different values of $t$. By using a step size of $\Delta t$, and calculating
the slope for each step, the solution can be approximated through the following
equations.

\begin{align}
y_{n+1} &= y_n + \Delta t \cdot \left.\frac{dy}{dt}\right\vert_{(t_n, y_n)}
\end{align}

\paragraph{Some notes on Euler's Method}\label{par:some_notes_on_euler_s_method}

Because this is a method of approximation, it becomes more accurate with the
smaller the value for $\Delta t$. With a value that is too large, the
approximation could miss asymptotes, or could cross equilibrium solutions that
cannot be crossed. (\ref{sub:method_of_barriers})

\section{Qualitative Analysis}\label{sec:qualitative_analysis}

Qualitative analysis is viewing the general trends of the solution of the
differential equation, without finding values at all.

\subsection{Stream Plots/Slope Fields}\label{sub:stream_plots_slope_fields}

\textbf{Basic Idea:} Do careful accounting of slopes and connect the points.
Creating a stream plot from a slope field (vector field).

Create a slope field, utilizing the Fundamental Theorem of ODE'S
(\ref{sec:the_fundamental_theorem_of_ode_s}). Then connect the points of the
slope field, creating an approximation of the solution plots.

\subsection{Equilibrium Solutions}\label{sub:equilibrium_solutions}

Equilibrium solutions are solutions that are easy to estimate, are are where
$y(t) \equiv C$, where $C$ is some constant. This can also be viewed as when
$\frac{dy}{dt} = 0$.

\end{document}
