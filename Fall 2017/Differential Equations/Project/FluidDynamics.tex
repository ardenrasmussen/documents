\documentclass[10pt]{article}
\usepackage{multicol,caption}
\usepackage{amsmath}
\usepackage{graphicx}
\usepackage{nameref}
\usepackage[rightcaption]{sidecap}
\usepackage[margin=1in]{geometry}
\renewcommand\thesection{}
\renewcommand\thesubsection{}
\newenvironment{Figure}
{\par\medskip\noindent\minipage{\linewidth}}
{\endminipage\par\medskip}
\newcommand{\der}[2]{\frac{\partial{} #1}{\partial{} #2}}
\newcommand{\sder}[3]{\frac{\partial{}^{2} #1}{\partial{} #2 \partial{} #3}}
\title{A Tiny Bit of Fluid Dynamics}
\author{Arden Rasmussen}
\begin{document}
\maketitle
\begin{abstract}
  We consider the vector field $\vec{U}\left(\mathrm{y}\right)$, where
  $\mathrm{y}=\left(x,y,z\right)$. From the fundamental theorem of ODE'S for each given
  initial condition $\mathrm{x}=\left(a,b,c\right)$ there is a unique solution of the
  IVP
  \begin{align*}
    \left\{ \begin{array}{l}
        \frac{\partial \mathrm{y}}{\partial t}=\vec{U}\left(\mathrm{y}\left(t\right)\right)\\
        \mathrm{y}\left(0\right)=\mathrm{x}
    \end{array} \right.
  \end{align*}
\end{abstract}

\begin{multicols}{2}

  \section{Problem 1}\label{sec:problem_1_}

  We first consider the fluid modeled by the velocity vector field
  \begin{align*}
    \vec{U}\left(\mathrm{y}\right) &=  -2x \vec{\imath}+\left(-2y+z\right)\vec{\jmath}+\left(y-2z\right)\vec{k}
  \end{align*}
  From this equation we aim to determine the trajectory of a molecule located at
  an arbitrary initial location.

  First we convert the equation for $\vec{U}$ into a matrix form.
  \begin{align*}
    \vec{U} &= \begin{pmatrix}
      -2x \\ -2y+z \\ y-2z
    \end{pmatrix}\\
    \vec{U} &= \begin{pmatrix}
      -2 & 0 & 0 \\
      0 & -2 & 1 \\
      0 & 1 & -2
      \end{pmatrix} \cdot \begin{pmatrix}
      x \\ y \\ z
    \end{pmatrix}
  \end{align*}

  Using the matrix representation of $U$, we can now attempt to find a general solution to the
  equation. We do this by first determining the eigenvalues of the matrix.
  \begin{align*}
    \det\begin{pmatrix}
      -2-\lambda & 0 & 0\\
      0 & -2-\lambda & 1\\
      0 & 1 & -2-\lambda
    \end{pmatrix} &= 0
  \end{align*}

  Solving the equation we find that the eigenvalues are as follows.
  \begin{align*}
    \lambda_1 = -1 \quad \lambda_2 = -2 \quad \lambda_3 = -3
  \end{align*}
  Using the eigenvalues and the original matrix representation of $U$, we can find the eigenvectors
  using $U\cdot V = \lambda \cdot V$. Thus we can obtain the respective eigenvectors for each
  eigenvalue.
  \begin{align*}
    V_1 = \begin{pmatrix}0\\1\\1\end{pmatrix}
    V_2 = \begin{pmatrix}1\\0\\0\end{pmatrix}
    V_3 = \begin{pmatrix}0\\1\\-1\end{pmatrix}
  \end{align*}

  Using these eigenvalues and eigenvectors, we can determine a general formula
  for $\mathrm{y}$ to be:
  \begin{align*}
    \mathrm{y}\left(t\right) &= C_{1}e^{-t}\begin{pmatrix}0\\1\\1\end{pmatrix} +
    C_{2}e^{-2t}\begin{pmatrix}1\\0\\0\end{pmatrix} +
    C_{3}e^{-3t}\begin{pmatrix}0\\1\\-1\end{pmatrix}
  \end{align*}

  Using the initial condition of $\mathrm{y}\left(0\right) = \mathrm{x}$, we can
  solve for the values of $C_1,\ C_2,\ C_3$.

  \begin{align*}
    \mathrm{y}\left(0\right) &= \begin{pmatrix}a\\b\\c\end{pmatrix}\\
                             &= C_1\begin{pmatrix}0\\1\\1\end{pmatrix} +
    C_2 \begin{pmatrix}1\\0\\0\end{pmatrix} +
    C_3 \begin{pmatrix}0\\1\\-1\end{pmatrix}\\
    \begin{pmatrix}
      a\\b\\c
      \end{pmatrix} &= \begin{pmatrix}
      0 & 1 & 0\\1 & 0 & 1\\1 & 0 & -1
      \end{pmatrix} \cdot \begin{pmatrix}
      C_1 \\ C_2 \\ C_3
    \end{pmatrix}\\
    \begin{pmatrix}
      C_1 \\ C_2 \\ C_3
      \end{pmatrix} &= \begin{pmatrix}
      0 & \frac{1}{2} & \frac{1}{2}\\
      1 & 0 & 0\\
      0 & \frac{1}{2} & -\frac{1}{2}
      \end{pmatrix} \cdot \begin{pmatrix}
      a \\ b \\ c
    \end{pmatrix}
  \end{align*}

  Using the values of $C_1,\ C_2,\ C_3$ we can determine the equation for
  $\mathrm{y}$ based off of the initial conditions $(a, b, c)$.
  \begin{align*}
    \mathrm{y}\left(t\right) &=
    \left(\frac{b}{2}+\frac{c}{2}\right)e^{-t}\begin{pmatrix}0\\1\\1\end{pmatrix}\\
    &+ \left(a\right)e^{-2t}\begin{pmatrix}1\\0\\0\end{pmatrix}\\
    &+\left(\frac{b}{2}-\frac{c}{2}\right)e^{-3t}\begin{pmatrix}0\\1\\-1\end{pmatrix}
  \end{align*}

  We plot this equation using matplotlib, to demonstrate that the phase portrait diagram for the
  equation consists of a three dimensional sink, where all lines converge to the origin.

  \begin{Figure}
    \centering
    \def\svgwidth{\columnwidth}
    \input{p1.pdf_tex}
    \captionof{figure}{This image represents the phase portrait diagram for the
    equation, with a set of different initial conditions.}\label{fig:p1_1}
  \end{Figure}

  \section{Problem 2}\label{sec:problem_2_}

  First we find the formula for $F_t:\mathrm{x}\mapsto \mathrm{y}\left(t\right)$

  \begin{align*}
    F_t\left(a,b,c\right) =\begin{pmatrix}ae^{-2t}\\
      \left(\frac{b}{2}+\frac{c}{2}\right)e^{-t} +
      \left(\frac{b}{2}-\frac{c}{2}\right)e^{-3t}\\
      \left(\frac{b}{2}+\frac{c}{2}\right)e^{-t}-\left(\frac{b}{2}-\frac{c}{2}\right)e^{-3t}
    \end{pmatrix}
  \end{align*}

  Using this equation we find the Jacobi matrix $DF_t$.

  \begin{align*}
    DF_t &= \begin{pmatrix}
      \frac{\partial x}{\partial a} & \frac{\partial x}{\partial b} & \frac{\partial x}{\partial c} \\
      \frac{\partial y}{\partial a} & \frac{\partial y}{\partial b} & \frac{\partial y}{\partial c} \\
      \frac{\partial z}{\partial a} & \frac{\partial z}{\partial b} & \der{z}{c}
    \end{pmatrix} \\
    DF_t &= \begin{pmatrix}
      e^{-2t} & 0 & 0 \\
      0 & \frac{1}{2}e^{-t} + \frac{1}{2}e^{-3t} & \frac{1}{2}e^{-t} - \frac{1}{2}e^{-3t}\\
      0 & \frac{1}{2}e^{-t} - \frac{1}{2}e^{-3t} & \frac{1}{2}e^{-t} + \frac{1}{2}e^{-3t}\\
    \end{pmatrix} \\
    \det \left(DF_t\right) &= e^{-6t}\\
    S\left( t \right) &= e^{-6t}
  \end{align*}
  This result demonstrates that as $t \rightarrow \infty$ then $S(t) \rightarrow 0$. This shows that
  as time goes on, the solution curve compresses closer and closer to zero.
  \begin{Figure}
    \centering
    \def\svgwidth{\columnwidth}
    \input{p2.pdf_tex}
    \captionof{figure}{This image shows the plot of the stretch factor $S(t)$. It can be seen that as
      $t$ grows, the value of $S(t)$ nears zero, indicating the compression of the solution curves at
    larger times.}\label{fig:p2_1}
  \end{Figure}

  \section{Problem $e$}\label{sec:problem_e_}

  Using a simplified generalization of the matrix for $S(t)$ we assume that
  $\mathrm{x} \mapsto \mathrm{y}$ is only two dimensional, so $(a, b) \mapsto
  (x,y)$. Thus the Jacobian matrix becomes a $2 \times 2$ matrix instead of a
  $3 \times 3$ matrix.
  \begin{align*}
    DF_t &= \begin{pmatrix}
      \der{x}{a} & \der{x}{b} \\
      \der{y}{a} & \der{y}{b}
    \end{pmatrix}
  \end{align*}

  Now we utilize this generalized matrix to determine a general formula for
  $S(t)$.

  \begin{align*}
    S(t) = \der{x}{a}\cdot\der{y}{b}-\der{y}{a}\cdot\der{x}{b}
  \end{align*}

  Now we can determine the derivative of $S$ in terms of $t$ like so.

  \begin{align*}
    \der{S}{t} = \sder{x}{a}{t} \cdot \der{y}{b} + \sder{y}{b}{t} \cdot
    \der{x}{a} - \sder{y}{a}{t} \cdot \der{x}{b} - \sder{y}{b}{t} \cdot \der{y}{a}
  \end{align*}

  We note that $\der{x}{t}$ appears in this equation, and thus we can use the
  equation from a velocity vector $U$ like so.

  \begin{align*}
    \der{x}{t} &= U_1\left(x\left(t\right), y\left(t\right)\right)\\
    \der{y}{t} &= U_2\left(x\left(t\right), y\left(t\right)\right)\\
  \end{align*}

  Using these equations we can apply chain rule to the equation for
  $\der{S}{t}$.

  \begin{align*}
    \der{S}{t} &=
    \left(\der{U_1}{x}\cdot\der{x}{a} + \der{U_1}{y}\cdot\der{y}{a}\right)\cdot\der{y}{b}\\
    &+\left(\der{U_2}{x}\cdot\der{x}{b} + \der{U_2}{y}\cdot\der{y}{b}\right)\cdot\der{x}{a}\\
    &-\left(\der{U_2}{x}\cdot\der{x}{a} + \der{U_2}{y}\cdot\der{y}{a}\right)\cdot\der{x}{b}\\
    &-\left(\der{U_1}{x}\cdot\der{x}{b} + \der{U_1}{y}\cdot\der{y}{b}\right)\cdot\der{y}{a}
  \end{align*}

  After simplification we find

  \begin{align*}
    \der{S}{t} &=
    \der{U_1}{x}\cdot\left(\der{x}{a}\cdot\der{y}{b}-\der{x}{b}\cdot\der{y}{a}\right)\\
    &+ \der{U_2}{y}\cdot\left(\der{y}{b}\cdot\der{x}{a}-\der{y}{a}\cdot\der{x}{b}\right)
  \end{align*}

  We can notice that the value of $S(t)$ appears, and so we can then factor it
  out.

  \begin{align*}
    \der{S}{t} = S(t)\cdot\left(\der{U_1}{x}+\der{U_2}{y}\right)
  \end{align*}

  We note that this represents the divergence of the vector $U$.

  \section{Problem 3}\label{sec:problem_3_}

  Using our solutions for \nameref{sec:problem_2_} and
  \nameref{sec:problem_e_}, we can verify our results, as the solution to
  \nameref{sec:problem_2_} should satisfy the differential equation found in
  \nameref{sec:problem_e_}.

  \begin{align*}
    \der{S}{t} &= \left(\der{U_1}{x} + \der{U_2}{y} + \der{U_3}{z}\right)\cdot
    e^{-6t}\\
    \der{U_1}{x} &= -2 \quad \der{U_2}{y} = -2\quad \der{U_3}{z} = -2\\
    \der{S}{t} &= -6\cdot e^{-6t}
  \end{align*}

  This verifies the solution that we expect.

  \section{Problem $\pi$}\label{sec:problem_pi_}

  Now we consider a \textit{incompressible fluid} which is modeled by the
  velocity vector field of the form
  \begin{align*}
    \vec{U}(\mathrm{y}) = \alpha x \vec{\imath} +
    (-2y+z)\vec{\jmath}+(y-2z)\vec{k}
  \end{align*}
  We first must determine the value of $\alpha$ such that the fluid this
  equation models is incompressible. To do so, we use the solution in
  \nameref{sec:problem_e_}. As the value of $S$ is the \textit{stretch factor}
  of the fluid, and if the fluid needs to be incompressible, then this value
  must be constant. Thus we can determine that $\der{S}{t} = 0$. Using this
  knowledge we are able to find the value of $\alpha$.
  \begin{align*}
    0 &= (\alpha + (-2) + (-2))\cdot S\\
    0 &= (\alpha - 4)\\
    \alpha &= 4
  \end{align*}
  Using this value of $\alpha$ we can see that the equation that models an
  incompressible fluid is
  \begin{align*}
    \vec{U}(\mathrm{y}) = 4x\vec{\imath} +
    (-2y+z)\vec{\jmath}+(y-2z)\vec{k}
  \end{align*}

  \subsection{Problem $\pi$.1}\label{sub:problem_pi_1}

  Using the same methods that we utilized in \nameref{sec:problem_1_} we solve
  for and equation for $\textrm{y}(t)$.

  \begin{align*}
    \vec{U} &= \begin{pmatrix}
      4 & 0 & 0\\
      0 & -2 & 1\\
      0 & 1 & -2
    \end{pmatrix} \\
    &\det\begin{pmatrix}
    4-\lambda & 0 & 0\\
    0 & -2-\lambda & 1\\
    0 & 1 & -2-\lambda
  \end{pmatrix}
\end{align*}
Solving this for $\lambda$'s we find:
\begin{align*}
  \lambda_1 = 4 \quad \lambda_2 = -3 \quad \lambda_3 = -1
\end{align*}
Using these $\lambda$'s we find the corresponding eigenvectors to be.
\begin{align*}
  V_1 = \begin{pmatrix}1\\0\\0\end{pmatrix}
  V_2 = \begin{pmatrix}0\\1\\-1\end{pmatrix}
  V_3 = \begin{pmatrix}0\\1\\1\end{pmatrix}
\end{align*}

Using these eigenvalues and eigenvectors we are able to create an equation
for $\mathrm{y}$ for the incompressible fluid:
\begin{align*}
  \mathrm{y}\left(t\right) &= C_{1}e^{4t}\begin{pmatrix}1\\0\\0\end{pmatrix} +
  C_{2}e^{-3t}\begin{pmatrix}0\\1\\-1\end{pmatrix} +
  C_{3}e^{-t}\begin{pmatrix}0\\1\\1\end{pmatrix}
\end{align*}
Using the same methods from \nameref{sec:problem_1_} we can solve for the
values of $C_1,C_2,C_3$. We find these to be
\begin{align*}
  C_1 = a \quad
  C_2 = \frac{b}{2}-\frac{c}{2} \quad
  C_3 = \frac{b}{2}+\frac{c}{2}
\end{align*}
Thus our equation for $\mathrm{y}$ becomes
\begin{align*}
  \mathrm{y}\left(t\right) &=
  ae^{4t}\begin{pmatrix}1\\0\\0\end{pmatrix}\\
  &+\left(\frac{b}{2}-\frac{c}{2}\right)e^{-3t}\begin{pmatrix}0\\1\\-1\end{pmatrix}\\
  &+\left(\frac{b}{2}+\frac{c}{2}\right)e^{-t}\begin{pmatrix}0\\1\\1\end{pmatrix}
\end{align*}

Using matplotlib with the equation for the incompressible fluid equation, we
can generate a phase portrait for the fluid. Note the saddle shape, as either
a sink or source would indicate that the fluid is being compressed (or
\textit{negatively compressed}). The 

\begin{Figure}
  \centering
  \def\svgwidth{\columnwidth}
  \input{ppi1.pdf_tex}
  \captionof{figure}{This image represents the phase portrait for the
    incompressible fluid, showing that there is a 3D saddle shape present in
  the space.}\label{fig:p1_1}
\end{Figure}

\subsection{Problem $\pi$.2}\label{sub:problem_pi_2}

Now we determine the stretch factor for the incompressible fluid. As mentioned
previously the stretch factor should be constant. First is to find the Jacobi
matrix $DF_t$ for the equation we found in \nameref{sub:problem_pi_1}.
\begin{align*}
    DF_t &= \begin{pmatrix}
      e^{4t} & 0 & 0 \\
      0 & \frac{1}{2}e^{-t} + \frac{1}{2}e^{-3t} & \frac{1}{2}e^{-t} - \frac{1}{2}e^{-3t}\\
      0 & \frac{1}{2}e^{-t} - \frac{1}{2}e^{-3t} & \frac{1}{2}e^{-t} + \frac{1}{2}e^{-3t}\\
    \end{pmatrix} \\
\end{align*}
Thus finding the $\det DF_t$ we can find the equation for $S(t)$.

\begin{align*}
  \det \left(DF_t\right) = 1
\end{align*}

This mirrors our initial predictions that the stretch factor would be constant,
and shows that the fluid is not being stretched at all, as that would
contradict the incompressibility of it.

\end{multicols}

\end{document}
