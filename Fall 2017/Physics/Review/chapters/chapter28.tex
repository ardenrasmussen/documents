\documentclass[../main.tex]{subfiles}

\begin{document}

\section{Chapter 28: Magnetic Induction}
\label{sec:chapter_28_magnetic_induction}

\subsection{Magnetic Flux $\phi_m$}
\label{sub:magnetic_flux_phi_m_}

\begin{description}
  \item[General definition]
    \begin{align}
      \phi_m = \int_s\vec{B} \cdot \hat{n} dA
    \end{align}
  \item[Uniform field, flat surface bounded by coil of $N$ turns]
    \begin{align}
      \phi_m = NBA\cos\theta
    \end{align}
    where $A$ is the area of the flat surface bound by a single turn.
  \item[Units]
    \begin{align}
      1Wb = 1T\cdot m^2
    \end{align}
  \item[Due to current in a circuit]
    \begin{align}
      \phi_m = LI
    \end{align}
  \item[Due to current in two circuits]
    \begin{align}
      \phi_{m1} = L_1I_1+MI_2\\
      \phi_{m2}=:_2I_2+MI_1
    \end{align}
\end{description}

\subsection{EMF}
\label{sub:emf}

\begin{description}
  \item[Faraday's law (includes both induction and motion emf)]
    \begin{align}
      \mathcal{E} = -\frac{d\phi_m}{dt}
    \end{align}
  \item[Induction (time-varying magnetic field, $C$ stationary)]
    \begin{align}
      \mathcal{E} = \oint_C\vec{E}\cdot d\vec{l}
    \end{align}
  \item[Rod moving perpendicular to both its length and $\vec{B}$]
    \begin{align}
      |\mathcal{E}| = vBl
    \end{align}
  \item[Self-induced (back emf)]
    \begin{align}
    \mathcal{E} = -L\frac{dI}{dt}
  \end{align}
\end{description}

\subsection{Faraday's Law}
\label{sub:faraday_s_law}

\begin{align}
  \mathcal{E} = -\frac{d\phi_m}{dt}
\end{align}

\subsection{Lenz's Law}
\label{sub:lenz_s_law}

The induced emf and induced current are in such a direction as to oppose, or
tend to oppose, the change that produces them.

\begin{description}
  \item[Alternative statement] \hfill \\
    When a magnetic flux through a surface changes, the magnetic field due to
    any induced current produces a flux of its own --- through the same surface
    and opposite sign to the change in flux.
\end{description}

\subsection{Inductance}
\label{sub:inductance}

\begin{description}
  \item[Self-inductance]
    \begin{align}
      L = \frac{\phi_m}{I}
    \end{align}
  \item[Self-inductance of a solenoid]
    \begin{align}
      L = \mu_0n^2Al
    \end{align}
  \item[Mutual inductance]
    \begin{align}
      M = \frac{\phi_{m21}}{I_1}=\frac{\phi_{m12}}{I_2}
    \end{align}
  \item[Units and constants]
    \begin{align}
      1H = 1Wb/A = 1T\cdot m^2/A\\
      \mu_0 = 4\pi\times10^{-7}H/m
    \end{align}
\end{description}

\subsection{Magnetic Energy}
\label{sub:magnetic_energy}

\begin{description}
  \item[Energy stored in an inductor]
    \begin{align}
      U_m = \frac{1}{2}LI^2
    \end{align}
  \item[Energy density in a magnetic field]
    \begin{align}
      u_m = \frac{B^2}{2\mu_0}
    \end{align}
\end{description}

\end{document}
