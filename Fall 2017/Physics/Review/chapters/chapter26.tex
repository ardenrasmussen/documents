\documentclass[../main.tex]{subfiles}

\begin{document}

\section{Chapter 26: The Magnetic Field}
\label{sec:chapter_26_the_magnetic_field}

\subsection{Magnetic Force}
\label{sub:magnetic_force}

\begin{description}
  \item[On a moving charge]
    \begin{align}
      \vec{F} = q\vec{v}\times\vec{B}
    \end{align}
  \item[On a current element]
    \begin{align}
      d\vec{F} = Id\vec{l}\times\vec{B}
    \end{align}
  \item[Unit of magnetic field] \hfill \\
    The SI unit of magnetic filds is the tesla ($T$). A commonly used unit is
    the gauss ($G$), which is related to the tesla by
    \begin{align}
      1G = 10^{-4}T
    \end{align}
\end{description}

\subsection{Motion of Point Charges}
\label{sub:motion_of_point_charges}

A particle of mass $m$ and charge $q$ moving with speed $v$ is a plane
perpendicular to a uniform magnetic field moves in a circular orbit. The period
and frequency of the circular motion are independent of the radius of the orbit
and of the speed of the particle.

\begin{description}
  \item[Newton's second law]
    \begin{align}
      qvB = m\frac{v^2}{r}
    \end{align}
  \item[Cyclotron period]
    \begin{align}
      T = \frac{2\pi m}{qB}
    \end{align}
  \item[Cyclotron frequency]
    \begin{align}
      f = \frac{1}{T} = \frac{qB}{2\pi m}
    \end{align}
\end{description}

\subsection{Current Loops}
\label{sub:current_loops}

\begin{description}
  \item[Magnetic dipol moment]
    \begin{align}
      \vec{\mu} = NIA\hat{n}
    \end{align}
  \item[Torque]
    \begin{align}
      \vec{\tau} = \vec{\mu}\times\vec{B}
    \end{align}
  \item[Potential energy of a magnetic dipole]
    \begin{align}
      U = -\vec{\mu}\cdot\vec{B}
    \end{align}
  \item[Net force] \hfill \\
    The net force on a current loop in a \textit{uniform} magnetic field is
    zero.
\end{description}

\subsection{The Hall Effect}
\label{sub:the_hall_effect}

When a conducting strip carrying a current is placed in a magnetic field, the
magnetic force on the charge carriers causes a searation of charge called the
Hall effect. This results in a voltage $V_H$, called the Hall voltage. The sign
of the charge carriers can be determined from a measurement of the sign of the
Hall voltage, and the number of carriers per unit colume can be determined from
the magnitude of $V_H$.

\begin{description}
  \item[Hall voltage]
    \begin{align}
      V_H = E_Hw = v_dBw = \frac{|I|}{nte}B
    \end{align}
\end{description}

\end{document}
