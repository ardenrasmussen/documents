\documentclass[../main.tex]{subfiles}

\begin{document}

\section{Chapter 27: Sources of Magnetic Field}
\label{sec:chapter_27_sources_of_magnetic_field}

\subsection{Magnetic Field $\vec{B}$}
\label{sub:magnetic_field_b_}

\begin{description}
  \item[Due to a moving point charge]
    \begin{align}
      \vec{B} = \frac{\mu_0}{4\pi}\frac{q\vec{v}\times\hat{r}}{r^2}
    \end{align}
    where $\hat{r}$ is a unit vector that points to the field point $P$ from
    the charge $q$ moving with velocity $\vec{v}$, and $\mu_0$ is a constant of
    proportionality called the magnetic constant (the permeability of empty
    space):
    \begin{align}
      \mu_0 = 4\pi\times10^{-7}T\cdot m/A=4\pi\times10^{-7}N/A^2
    \end{align}
  \item[Due to a current element]
    \begin{align}
      d\vec{B} = \frac{\mu_0}{4\pi}\frac{Id\vec{l}\times\hat{r}}{r^2}
    \end{align}
  \item[On the axis of a current loop]
    \begin{align}
      B_z = \frac{\mu_0}{4\pi}\frac{2\pi R^2I}{{\left(z^2+R^2\right)}^{\frac{3}{2}}}
    \end{align}
  \item[Inside a long solenoid, far from the ends]
    \begin{align}
      B_z = \mu_{0}nI
    \end{align}
    where $n$ is the number of turns per unit length.
  \item[Due to a straight-wire segment]
    \begin{align}
      B = \frac{\mu_0}{r\pi}\frac{I}{R}\left(\sin\theta_2-\sin\theta_1\right)
    \end{align}
    Where $R$ is the perpendicular distance to the wire and $\theta_1$ and
    $\theta_2$ are the angles subtended at the field point by the end of the
    wire.
  \item[Inside the loops of a tightly wound toriod]
    \begin{align}
      B = \frac{\mu_0}{2\pi}\frac{NI}{r}
    \end{align}
\end{description}

\subsection{Magnetic Field Lines}
\label{sub:magnetic_field_lines}

Magnetic lines neither begin nor end. Either they form closed loops or they
continue indefinitely.

\subsection{Gauss's Law for Magnetism}
\label{sub:gauss_s_law_for_magnetism}

\begin{align}
  \phi_{m \text{ net }} = \oint_S\vec{B}\cdot\hat{n}dA = \oint_SB_ndA = 0
\end{align}

\subsection{Magnetic Poles}
\label{sub:magnetic_poles}

Magnetic poles always occur in north-south pairs. Isolated magnetic poles have
not been found.

\subsection{Amp\`ere's Law}
\label{sub:ampe`re_s_law}

\begin{align}
  \oint_C\vec{B}\cdot d\vec{l} = \oint_CB_tdl = \mu_0I_C
\end{align}

where $C$ is any closed curve.

\begin{description}
  \item[Validity of Amp\`re's law] \hfill \\
    Amp\`ere's law is valid only if the currents are steady and continuous. It
    can be used to derive expressions for the magnetic field for situations
    with a high degree of symmetry, such as a long, straight, current-carrying
    wire or a long, tightly wound solenoid.
\end{description}

\subsection{Magnetism in Matter}
\label{sub:magnetism_in_matter}

Matter can be classified as paramagnetic, ferromagnetic, or diamagnetic.

\begin{description}
  \item[Magnetization] \hfill \\
    A magnetized material is described by its magnetization vector $\vec{M}$,
    which is defined as the magnetic dipole moment per unit volume of the
    material:
    \begin{align}
      \vec{M} = \frac{d\vec{\mu}}{dV}
    \end{align}
    The magnetic field due to a uniformly magnetized cylinder is the same as if
    the cylinder carried a current per unit length of magnitude $M$ on its
    surface. This current, which is due to the intrinsic motion of the atom
    charges in the cylinder, is called an amperian current.
\end{description}

\subsection{$\vec{B}$ in Magnetic Materials}
\label{sub:_b_in_magentic_materials}

\begin{align}
  \vec{B} = \vec{B}_\text{app} + \mu_0\vec{M}
\end{align}

\begin{description}
  \item[Magnetic susceptibility $X_m$]
    \begin{align}
      \vec{M} = \chi_m\frac{\vec{B}_\text{app}}{\mu_0}
    \end{align}
    For paramagnetic materials, $\chi_m$ is a small positive number that
    depends on temperature. For diamagnetic materials, it is a small negative
    constant independent of temperature. For superconductors, $\chi_m = -1$.
    For ferromagnetic materials, the magnetization depends not only on the
    magnetizing current but also on the past history of the material.
  \item[Relative permeability]
    \begin{align}
      \vec{B} = K_m\vec{B}_\text{app}
    \end{align}
    where
    \begin{align}
       K_m = 1 + \chi_m
    \end{align}
\end{description}

\subsection{Atomic Magnetic Moments}
\label{sub:atomic_magnetic_moments}

\begin{align}
  \vec{\mu} = \frac{q}{2m}\vec{L}
\end{align}
where $\vec{L}$ is the orbital angular momentum of the particle.

\begin{description}
  \item[Bohr magneton]
    \begin{align}
      \mu_B = \frac{e\hbar}{2m_e} = 9.27\times10^{-24}A\cdot m^2 =
      9.27\times10^{-24}J/T
    \end{align}
  \item[Due to the orbital motion of an electron]
    \begin{align}
      \vec{\mu}_l = -\mu_B\frac{\vec{L}}{\hbar}
    \end{align}
  \item[Due to electron spin]
    \begin{align}
      \vec{\mu}_S = -2\mu_B\frac{\vec{S}}{\hbar}
    \end{align}
\end{description}

\subsection{Paramagnetism}
\label{sub:paramagnetism}

Paramagnetic materials have permanent atomic magnetic moments that have random
directions in the absence of an applied magnetic field. In an applied field
these dipoles are aligned with the field to some degree, producing a small
contribution to the total field that adds to the applied field. The degree of
alignment is small except in very strong fields and at very low temperatures.
At ordinary temperatures, thermal motion tends to maintain the random
directions of the magnetic moments.

\subsection{Ferromagnetism}
\label{sub:ferromagnetism}

Ferromagnetic materials have small regions of space called magnetic domains in
which all the permanent atomic magnetic moments are aligned. When the material
is unmagnetized, the direction of alignment in one domain is independent of
that in another domain so that no net magnetic field is produced. When the
material is magnetized, the domains of a ferromagnetic material are aligned,
producing a very strong contribution to the magnetic field. This alignment can
persist in magnetically hard materials, even when the external field is
removed, thus leading to permanent magnets.

\subsection{Diamagnetism}
\label{sub:diamagnetism}

Diamagnetic materials are those materials in which the magnetic moments of all
electrons in each atom cancel, leaving each atom with a zero magnetic moment in
the absence of an external field. In an applied magnetic field, a very small
magnetic moment induced that tends to weaken the field. This effect is
independent of temperature. Superconductors are diamagnetic with a magnetic
susceptibility equal to $-1$.

\end{document}
