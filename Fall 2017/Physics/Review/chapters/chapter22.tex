\documentclass[../main.tex]{subfiles}

\begin{document}

\section{Chapter 22: The Electric Field II:\@ Continuous Charge Distributions}\label{sec:chapter_22}

\subsection{Electric Field for Continuous Charge Distribution}\label{sub:electric_field_for_countinuous_charge_distriution}

\begin{equation}
  \vec{E} = \int{d\vec{E}} = \int{\frac{k\hat{r}}{r^2}dq} \quad \text{(Coulomb's law)}
\end{equation}
where \(dq=\rho dV\) for a charge distributed throughout a volume, \(dq =
\sigma dA\) for a charge distributed on a surface, and \(dq = \lambda dL\) for
a charge distributed along a line.

\subsection{Electric Flux}\label{sub:electric_flux}

\begin{equation}
  \phi = \lim_{\Delta A_i \to 0}\sum_{i}\vec{E}_i\cdot\hat{n}_i\Delta A_i =
  \int_S\vec{E}\cdot\hat{n}dA
\end{equation}

\subsection{Gauss's Law}\label{sub:gauss_s_law}

\begin{equation}
  \phi_{\text{net}} = \oint_{S}\vec{E}\cdot\hat{n}dA = \oint_{S}E_{n}dA =
  \frac{Q_{\text{inside}}}{\epsilon_{0}}
\end{equation}
The net outward electric flux through a closed surface equals the net charge
within the surface divided by \(\epsilon_0\).

\subsection{Coulomb Constant $k$ and Electric Constant (Permittivity of Free
Space) $\epsilon_0$}\label{sub:coulomb_constant_k_and_electric_constant_permittivity_of_free_space_epsilon_0_}

\begin{align}
  k &= \frac{1}{4\pi\epsilon_0} = 8.99\times 10^9N\cdot m^2/C^2\\
  \epsilon_0 &= \frac{1}{4\pi k} = 8.85\times 10^{-12}C^2/(N\cdot m^2)
\end{align}

\subsection{Coulomb's Law and Gauss's Law}
\label{sub:coulomb_s_law_and_gauss_s_law}

\begin{align}
  \vec{E} &= \frac{1}{4\pi \epsilon_0}\frac{q}{r^2}\hat{r}\\
  \phi_{\text{net}} &= \oint_{S}E_{n}dA = \frac{Q_{\text{inside}}}{\epsilon_0}
\end{align}

\subsection{Discontinuity of $E_n$}
\label{sub:discontinuity_of_e_n_}

At a surface having a surface charge density of $\sigma$, the component of the
electric field normal to the surface is discontinuous by $\sigma/\epsilon_0$.

\begin{align}
  E_{n+} - E_{n-} = \frac{\sigma}{\epsilon_0}
\end{align}

\subsection{Charge on a Conductor}
\label{sub:charge_on_a_conductor}

In electrostatic equilibrium, the charge density is zero throughout the
material of the conductor. All excess or deficit charge resides on the surfaces
of the conductor.

\subsection{$\vec{E}$ Just Outside a Conductor}
\label{sub:_e_just_outside_a_conductor}

The resultant electric field just outside the surface of a conductor is normal
to the surface and has the magnitude $\sigma/\epsilon_0$, where $\sigma$ is the
local surface charge density on the conductor:
\begin{align}
  E_n = \frac{\sigma}{\epsilon}
\end{align}

\subsection{Electric Fields for Selected Uniform Charge Distributions}
\label{sub:electric_fields_for_selected_uniform_charge_distributions}

\begin{description}
  \item[Of a line charge of infinite length]
    \begin{align}
      E_R = 2k\frac{\lambda}{R} = \frac{1}{2\pi\epsilon_0}\frac{\lambda}{R}
    \end{align}
  \item[On the axis of a chrged ring]
    \begin{align}
      E_z = \frac{kQz}{{\left(z^2+a^2\right)}^{3/2}}
    \end{align}
  \item[On the axis of a charged disk]
    \begin{align}
      E_z = \text{sign}(z) \cdot \frac{\sigma}{2\epsilon_0} 
      \left[1-{\left(1+\frac{R^2}{z^2}\right)}^{-1}\right]
    \end{align}
  \item[Of a charged infinite plane]
    \begin{align}
      E_z = \text{sign}(z)\cdot\frac{\sigma}{2\epsilon_0}
    \end{align}
  \item[Of a charged thin spherical shell]
    \begin{align}
      E_r &= \frac{1}{4\pi\epsilon_0}\frac{Q}{r^2}\quad r > R\\
      E_r &= 0\quad r < R
    \end{align}
\end{description}



\end{document}
