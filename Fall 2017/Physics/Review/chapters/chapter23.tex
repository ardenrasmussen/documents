\documentclass[../main.tex]{subfiles}

\begin{document}

\section{Chapter 23: Electric Potential}
\label{sec:chapter_23_electric_potential}

\subsection{Potential Difference}
\label{sub:potential_difference}

The potential different $V_b - V_a$ is defined as the negative of the work per
unit charge done by the electric field on a test charge as it moves from point
$a$ to point $b$.
\begin{align}
  \Delta V = V_b - V_a = \frac{\Delta U}{q_0} = -\int_a^b\vec{E}\cdot\vec{dl}
\end{align}

\begin{description}
  \item[Potential difference for infinitesminal displacements]
    \begin{align}
      dV = -\vec{E} \cdot \vec{dl}
    \end{align}
\end{description}

\subsection{Electric Potential}
\label{sub:electric_potential}

\begin{description}
  \item[Potential due to a point charge]
    \begin{align}
      V = \frac{kq}{r}-\frac{kq}{r_{\text{ref}}}\quad (V = 0
      \text{ if }r=r_{\text{ref}})
    \end{align}
  \item[Coulomb potential]
    \begin{align}
      V = \frac{kq}{r}\quad(V = 0 \text{ if  } r = \inf)
    \end{align}
  \item[Potential due to a system of point charges]
    \begin{align}
      V = \sum_i\frac{kq_i}{r_i}\quad(V=0 \text{ if } r_i = \inf, i=1,2,\ldots)
    \end{align}
  \item[Potential due to continuous charge distribution]
    \begin{align}
      V = \int{\frac{kdq}{r}}\quad(V=0\text{ if }r=\inf)
    \end{align}
    Where $dq$ is an increment of charge and $r$ is the distance from the
    increment to the field point. This expression can be used only if the
    charge distribution is contained in an finite volume so that the potential
    can be chosen to be zero at infinity.
  \item[Continuity of electric potential] \hfill \\
    The potential function $V$ is continuous everywhere in space.
\end{description}

\subsection{Computing the Electric Field from the Potential}
\label{sub:computing_the_electric_field_from_the_potential}

The electric field points in the direction of the most rapid decrease in the
potential. The charge in potential when a test charge undergoes a displacement
$\vec{dl}$ is given by
\begin{align}
  E_{\text{tan}} = -\frac{dV}{dl}
\end{align}

\begin{description}
  \item[Gradient] \hfill \\
    A vector that points in the direction of the greatest rate of change in a
    scalar function and that has a magnitude equal to the derivative of that
    function, with respect to the distance in that direction, is called the
    gradient of the function. $\vec{E}$ is the negative gradient of $V$.
  \item[Potential a function of $x$ alone]
    \begin{align}
      E_X = -\frac{dV(x)}{dx}
    \end{align}
  \item[Potential a function of $r$ alone]
    \begin{align}
      E_r = -\frac{dV(r)}{dr}
    \end{align}
\end{description}

\subsection{Units}
\label{sub:units}

\begin{description}
  \item[$V$ and $\Delta V$] \hfill \\
    The SI unit of potential and potential difference is the volt ($V$):
    \begin{align}
      1V = 1J/C
    \end{align}
  \item[Electric field]
    \begin{align}
       1N/C = 1V/m
    \end{align}
  \item[Electron volt] \hfill \\
    The electron volt ($eV$) is the charge in potential energy of a particle of
    charge $e$ as it moves from $a$ to $b$ where $V_b - V_a = 1$ volt:
    \begin{align}
      1eV = 1.6\times10^{-19}C\cdot V=1.6\times10^{-19}J
    \end{align}
\end{description}

\subsection{Potential Energy of Two Point Charges}
\label{sub:potential_energy_of_two_point_charges}

\begin{align}
  U = q_0V=\frac{kq_0q}{r}\quad(U=0\text{ if }r=\inf)
\end{align}

\subsection{Potential Functions}
\label{sub:potential_functions}

\begin{description}
  \item[On the axis of a uniformly charged ring]
    \begin{align}
      V=\frac{kQ}{\sqrt{z^2+a^2}}\quad(V=0\text{ if }|z|=\inf)
    \end{align}
  \item[On the axis of a uniformly charged disk]
    \begin{align}
      V=2\pi k\sigma|z|\left(\sqrt{1+\frac{R^2}{z^2}} - 1\right)\quad(V=0\text{ if }|z|=\inf)
    \end{align}
  \item[For an infinite plane of charge]
    \begin{align}
      V = V_0 -2\pi k \sigma |x|\quad(V=V_0\text{ if }x=0)
    \end{align}
  \item[For a spherical shell of charge]
    \begin{align}
      V=\left\{\begin{array}{ll}\frac{kQ}{r} & r\geq R\\\frac{kQ}{R} & r \leq
      R\end{array}\right.\quad (V=0\text{ if }r=\inf)
    \end{align}
  \item[For an infinite line charge]
    \begin{align}
      V=2k\lambda\ln\frac{R_{\text{ref}}}{R}\quad(V=0\text{ if }r=R_{\text{ref}})
    \end{align}
\end{description}

\subsection{Charge on a Nonspherical Conductor}
\label{sub:charge_on_a_nonspherical_conductor}

On a conductor of arbitrary shape, the surface charge density $\sigma$ is
greatest at points where the radius of curvature is smallest.

\subsection{Dielectric Breakdown}
\label{sub:dielectric_breakdown}

The amount of charge that can be placed on a conductor is limited by the fact
that molecules of the surrounding medium undergo dielectric breakdown at very
high electric fields, causing the medium to become a conductor.

\begin{description}
  \item[Dielectric strength] \hfill \\
    The dielectric strength is the magnitude of the electric field at which
    dielectric breakdown occurs. The dielectric strength of dry air is
    \begin{align}
      E_{\text{max}} \approx 3\times10^6V/m = 3MV/m
    \end{align}
\end{description}

\subsection{Electrostatic Potential Energy}
\label{sub:electrostatic_potential_energy}

The electrostatic potential energy of a system of point charges is the work
needed to bring the charges from a infinite separation to their final
positions.

\begin{description}
  \item[Of point charges]
    \begin{align}
      U = \frac{1}{2}\sum_{i=1}^{n}q_iV_i
    \end{align}
  \item[Of a conductor with charge $Q$ at potential $V$]
    \begin{align}
      U = \frac{1}{2}QV
    \end{align}
  \item[Of a system of conductors]
    \begin{align}
      U = \frac{1}{2}\sum_{i=1}^{n}Q_iV_i
    \end{align}
\end{description}

\end{document}
