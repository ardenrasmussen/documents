\documentclass[../main.tex]{subfiles}

\begin{document}

\section{Chapter 21: The Electric Field I:\@ Discrete Charge Distributions}\label{chapter-21}

\subsection{Charge}\label{charge}

There are two kinds of charge, positive and negative. Charges of like
sign repel, those of opposite sign attract.
\begin{description}
  \item[Quantization] \hfill \\ Charge is quantized --- it always occurs in integer
    multiples of the fundamental charge unit \(e\). The charge of the electron is
    \(-e\) and that of the proton is \(+e\).
  \item[Magnitude] \hfill \\
    \begin{equation}
      e=1.60\times10^{-19}C
    \end{equation}
  \item[Conservation] \hfill \\ Charge is conserved. When charged particles are created
    or annihilated, the total amount of charge carried by the created or
    annihilated particles is zero.
\end{description}

\subsection{Conductors and Insulators}\label{conductors-and-insulators}

In metals, about one electron per atom is delocalized (free to move
about the entire material). In insulators, all the electrons are bound
to nearby atoms.
\begin{description}
  \item[Ground] \hfill \\ A very large conductor (such as Earth) that can
    supply or absorb a virtually unlimited amount of charge is called a ground.
\end{description}

\subsection{Charge by Induction}\label{charge-by-induction}

To charge a conductor buy induction: connect a ground to the conductor,
hold an external charge near the conductor (to attract or repel the
conduction electrons), then disconnect the conductor from the ground,
and finally move the external charge away from the conductor.

\subsection{Coulomb's Law}\label{coulombs-law}

The force exerted by point charge \(q_1\) on point charge \(q_2\) a
distance \(r_{12}\) away is given by

\begin{equation}
  \vec{F}_{12} = \frac{kq_1q_2}{r_{12}^{2}}\hat{r}_{12}
\end{equation}

where unit vector \(\hat{r}_{12}\) points from \(q_1\) towards \(q_2\).
\begin{description}
  \item[Coulomb constant] \hfill \\
    \begin{equation}
      k = 8.99 \times10^9N\cdot m^2/C^2
    \end{equation}
\end{description}

\subsection{Electric Field}\label{electric-field}

The electric field due to a system of charges at a point is defined as
the net force \(\vec{F}\), exerted by those charges on a very small
positive test charge \(q_0\), divided by \(q_0\):

\begin{equation}
  \vec{E} = \frac{\vec{F}}{q_0}
\end{equation}

\begin{description}
  \item[Due to a point charge] \hfill \\
    \begin{equation}
      \vec{E}_{iP} = \frac{kq_i}{r_{iP}^{2}}\hat{r}_{iP}
    \end{equation}
  \item[Due to a system of point charges] \hfill \\
    The Electric field at \(P\) due to several charges is the vector sum of
    the fields at \(P\) due to the individual charges:
    \begin{equation}
      \vec{E}_P = \sum_{i}\vec{E}_{iP}
    \end{equation}
\end{description}

\subsection{Electric Field Lines}\label{sub:electric_field_lines}

The electric field can be represented by electric field lines that emanate from
positive charges and terminate on negative charges. The strength of the
electric field is indicated by the density of the electric field lines.

\subsection{Dipole}\label{sub:dipole}

A dipole is a system of two equal but opposite charges separated by a small
distance.

\begin{description}
  \item[Dipole moment] \hfill \\ 
    \begin{equation}
      \vec{p} = q\vec{L}
    \end{equation}
    where \(\vec{L}\) is the position of the positive charge relative to the
    negative charge.
  \item[Field due to dipole] \hfill \\ The electric field strength far from a
    dipole is proportional to the magnitude of the dipole moment and decreased
    with the cube of the distance.
  \item[Torque on a dipole] \hfill \\ In a uniform electric field, the net
    force on a dipole is zero, but there is a torque that tends to align the
    dipole in the direction of the field.
    \begin{equation}
      \vec{\tau} = \vec{p}\times\vec{E}
    \end{equation}
  \item[Potential energy of a dipole] \hfill \\
    \begin{equation}
      U = -\vec{p}\cdot\vec{E}+U_0
    \end{equation}
    Where \(U_0\) is usually taken to be zero.
\end{description}

\subsection{Polar and Nonpolar Molecules}\label{sub:polar_and_nonpolar_molecules}

Polar molecules, such as \(H_2O\) and \(HCL\), have permanent dipole moments
because their centers of positive and negative charge do not coincide. They
behave like simple dipoles in an electric field. Nonpolar molecules do not have
permanent dipole moments, but they acquire induced dipole moments in the
presence of an electric field.



\end{document}
