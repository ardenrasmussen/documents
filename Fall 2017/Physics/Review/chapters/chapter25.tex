\documentclass[../main.tex]{subfiles}

\begin{document}
\section{Chapter 25: Electric Current and Direct-Current Circuits}
\label{sec:chapter_25_electric_current_and_direct_current_circuits}

\subsection{Electric Current}
\label{sub:electric_current}

Electric current is the rate of flow of electric charge through a
cross-sectional area.
\begin{align}
  I = \frac{\Delta Q}{\Delta t}
\end{align}
in the limit that $\Delta t$ approaches zero.

\begin{description}
  \item[Drift velocity] \hfill \\
    In a conducting wire, electric current is the result of the slow drift of
    negatively charged electrons that are accelerated by an electric field in
    the wire and then collide with the lattice ions. Typical drift speeds of
    electrons in wires are of the order of a few millimeters per second. For
    mobile charges moving in the positive direction,
    \begin{align}
       I = qnAv_d
    \end{align}
    where $q = -e$, $n$ is the number density of free electrons, $A$ is the
    cross-sectional area, and $v_d$ is the drift speed.
  \item[Current density] \hfill \\
    The current density $\vec{\jmath}$ is related to the drift velocity by
    \begin{align}
      \vec{\jmath} = qn\vec{v_d}
    \end{align}
    The current $I$ through a cross-sectional surface is the flux of the
    current density through the surface.
\end{description}

\subsection{Resistance}
\label{sub:resistance}

\begin{description}
  \item[Definition of resistance]
    \begin{align}
      R= \frac{V}{I}
    \end{align}
  \item[Resistivity, $\rho$]
    \begin{align}
      R = \rho\frac{L}{A}
    \end{align}
  \item[Temperature coefficient of resistivity, $\alpha$]
    \begin{align}
      \alpha = \frac{\frac{\rho - \rho_0}{\rho_0}}{T-T_0}
    \end{align}
\end{description}

\subsection{Ohm's Law}
\label{sub:ohm_s_law}

For ohmic materials, the resistance does not depend on either the current or
the potential drop:
\begin{align}
  V = IR,\quad R \text{ constant}
\end{align}

\subsection{Power}
\label{sub:power}

\begin{description}
  \item[Supplied to a device or segment]
    \begin{align}
       P = IV
    \end{align}
  \item[Delivered to a resistor]
    \begin{align}
      P = IV - I^2R=\frac{V^2}{R}
    \end{align}
\end{description}

\subsection{Emf}
\label{sub:emf}

\begin{description}
  \item[Source of emf] \hfill \\
    A device that supplies electric energy to a circuit.
  \item[Power supplied by an ideal emf source]
    \begin{align}
      P = I\mathcal{E}
    \end{align}
\end{description}

\subsection{Battery}
\label{sub:battery}

\begin{description}
  \item[Ideal] \hfill \\
    An ideal battery is a source of emf that maintains a constant potential
    difference between its two terminals, independent of the current through
    the battery.
  \item[Real] \hfill \\
    A real battery can be considered as an ideal battery in series with a small
    resistance, called its internal resistance.
  \item[Terminal voltage]
    \begin{align}
      V_a - V_b = \mathcal{E} - Ir
    \end{align}
    where in the battery the positive direction is the direction of increasing
    potential.
  \item[Total energy stored]
    \begin{align}
      E_{\text{stored}} = Q\mathcal{E}
    \end{align}
\end{description}

\subsection{Equivalent Resistance}
\label{sub:equivalent_resistance}

\begin{description}
  \item[Resistors in series]
    \begin{align}
      R_{\text{eq}} = R_1 + R_2 + \cdots
    \end{align}
  \item[Resistors in parallel]
    \begin{align}
      \frac{1}{R_{\text{eq}}} = \frac{1}{R_1} + \frac{1}{R_2} + \cdots
    \end{align}
\end{description}

\subsection{Kirchhoff's Rules}
\label{sub:kirchhoff_s_rules}

\begin{enumerate}
  \item When any closed loop is traversed, the algebraic sum of the changes in
    potential around the loop must equal zero.
  \item At any junction (branch point)  in a circuit where the current can
    divide, the sum of the current into the junction must equal the sum of the
    currents out of the junction.
\end{enumerate}

\subsection{Measuring Devices}
\label{sub:measuring_devices}

\begin{description}
  \item[Ammeter] \hfill \\
    An ammeter is a very low resistance device that is placed in series with a
    circuit element to measure the current in the element.
  \item[Voltmeter] \hfill \\
    A voltmeter is a very high resistance device that is placed in parallel
    with a circuit element to measure the potential difference across the
    element.
  \item[Ohmmeter] \hfill \\
    An ohmmeter is a device containing a battery connected in series with a
    galvanometer and resistor that is used to measure the resistance of a
    circuit element placed across its terminals.
\end{description}

\subsection{Discharging a Capacitor}
\label{sub:discharging_a_capacitor}

\begin{description}
  \item[Charge on the capacitor]
    \begin{align}
      Q(t) = Q_0e^{-\frac{t}{RC}} = Q_0e^{-\frac{t}{\tau}}
    \end{align}
  \item[Current in the circuit]
    \begin{align}
      I = -\frac{dQ}{dt} = \frac{V_0}{R} e^{-\frac{t}{RC}} =
      I_0e^{-\frac{t}{\tau}}
    \end{align}
  \item[Time constant]
    \begin{align}
       \tau = RC
    \end{align}
\end{description}

\subsection{Charging a Capacitor}
\label{sub:charging_a_capacitor}

\begin{description}
  \item[Charge on the capacitor]
    \begin{align}
      Q = C\mathcal{E}\left[1-e^{-\frac{t}{RC}}\right] =
      Q_f\left(1-e^{-\frac{t}{\tau}}\right)
    \end{align}
  \item[Current in the cirucit]
    \begin{align}
      I = \frac{\mathcal{E}}{R}e^{-\frac{t}{RC}} = I_0e^{-\frac{t}{\tau}}
    \end{align}
\end{description}

\end{document}
