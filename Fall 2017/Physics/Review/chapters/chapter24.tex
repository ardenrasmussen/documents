\documentclass[../main.tex]{subfiles}

\begin{document}
\section{Chapter 24: Capacitance}
\label{sec:chapter_24_capacitance}

\subsection{Capacitor}
\label{sub:capacitor}

A capacitor is a device for storing charge and energy. It consists of two
conductors that are insulated from each other and carry equal and opposite
charges.

\subsection{Capacitance}
\label{sub:capacitance}

Definition of capacitance
\begin{align}
  C = \frac{Q}{V}
\end{align}

\begin{description}
  \item[Single conductor] \hfill \\
    $Q$ is the conductor's total charge, $V$ is the conductor's potential
    relative to its surroundings.
  \item[Capacitor] \hfill \\
    $Q$ is the magnitude of the charge on either conductor, $V$ is the
    magnitude of the potential difference between the conductors.
  \item[Of an isolated spherical conductor]
    \begin{align}
       C = 4\pi\epsilon_0R
    \end{align}
  \item[Of a parallel-plate capacitor]
    \begin{align}
      C= \frac{\epsilon_0A}{d}
    \end{align}
  \item[Of a cylindrical capacitor]
    \begin{align}
      C = \frac{2\pi\epsilon_0L}{\ln\left(\frac{R_2}{R_1}\right)}
    \end{align}
  \item[Energy stored in a capacitor]
    \begin{align}
      U = \frac{1}{2}QV = \frac{1}{2}\frac{Q^2}{C} = \frac{1}{2}CV^2
    \end{align}
  \item[Energy density of an electric field]
    \begin{align}
      u_e = \frac{1}{2}\epsilon_0E^2
    \end{align}
\end{description}

\subsection{Equivalent Capacitance}
\label{sub:equivalent_capacitance}

\begin{description}
  \item[Parallel capacitors] \hfill \\
    When devices are connected in parallel, the voltage drop is the same across
    each.
    \begin{align}
      C_{\text{eq}} = C_1 + C_2 + \cdots
    \end{align}
  \item[Series capacitors] \hfill \\
    When capacitors are in series, the voltage drop add. If the total charge on
    each connected pair of plates is zero, then:
    \begin{align}
      \frac{1}{C_{\text{eq}}} = \frac{1}{C_1} + \frac{1}{C_2} + \cdots
    \end{align}
\end{description}

\subsection{Dielectrics}
\label{sub:dielectrics}

\begin{description}
  \item[Macroscopic behavior] \hfill \\
    A nonconducting material is called a dielectric. When a dielectric is
    inserted between the plates of a charged, electrically isolated capacitor,
    the electric field between the plates is weakened and the capacitance is
    thereby increased by the factor $\kappa$, which is the dielectric
    constant.
  \item[Microscopic view] \hfill \\
    The electric field in the dielectric of a capacitor is weakened because the
    molecular dipole moments (either preexisting or induced) tend to align with
    the applied field and thereby produce a second electric field inside the
    dielectric that opposes the applied field. The aligned dipole moment of the
    dielectric is proportional to the applied field.
  \item[Electric field inside]
    \begin{align}
      E = \frac{E_0}{\kappa}
    \end{align}
  \item[Effect on capacitance]
    \begin{align}
       C = \kappa C_0
    \end{align}
  \item[Permittivity $\epsilon$]
    \begin{align}
       \epsilon = \kappa \epsilon_0
    \end{align}
  \item[Uses of a dielectric]
    \begin{enumerate}
      \item Increases capacitance
      \item Increases dielectric strength
      \item Physically separates conductors
    \end{enumerate}
\end{description}

\end{document}
