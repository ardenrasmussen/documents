\documentclass[10pt]{article}
\usepackage{subfiles}
\usepackage{multicol}
\usepackage{caption}
\usepackage{graphicx}
\usepackage{nameref}
\usepackage{amsmath}
\usepackage{multicol}
\usepackage{minted}
\usepackage{fancyhdr}
\usepackage{nameref}
\usepackage{hyperref}
\usepackage{tikz}
\usepackage{wrapfig}
\usetikzlibrary{snakes,arrows,shapes}
\usepackage[rightcaption]{sidecap}

\graphicspath{ {languages/} }

\pagestyle{fancy}

\newenvironment{Figure}
{\par\medskip\noindent\minipage{\linewidth}}
{\endminipage\par\medskip}


% \newcommand{\svg}[2][1]{\ifthenelse{\equal{#1}{}}{A}{\ifthenelse{\equal{#1}{1}}{B}{C}}}
\newcommand{\svg}[2][\textwidth]{%
  \def\svgwidth{#1}%
  \input{#2}%
}

\newcommand{\usec}[1]{%
\section{#1}%
\noindent\rule{\textwidth}{2pt}%
\vspace{10pt}%
}
\newcommand{\cd}[1]{\texttt{#1}}
\newcommand{\nr}[1]{\nameref{#1}}

\title{Computer Languages}
\author{Arden Rasmussen}

\begin{document}
\pagenumbering{gobble}
\maketitle
\newpage

\begin{abstract}\label{abstract}
   This document aims at compairing a large number of the posible different
   programming languages that are currently avalible, for both their
   efficiency, code size, support, readability and many other features. This
   document does not aim to single out a single language as the best or worst,
   but to mearly show the variations between the languages, and to provide some
   introduction to the languages. It also plans to show the inheritance of the
   language, and the basic syntax of the language.
\end{abstract}

\newpage
\tableofcontents
\newpage
\pagenumbering{arabic}

\subfile{languages/ada}
\subfile{languages/bash}
\subfile{languages/c}
\subfile{languages/cpp}
\subfile{languages/csharp}
\subfile{languages/go}
\subfile{languages/java}
\subfile{languages/javascript}
\subfile{languages/lisp}
\subfile{languages/lua}
\subfile{languages/perl}
\subfile{languages/python2}
\subfile{languages/python3}
\subfile{languages/r}
\subfile{languages/ruby}
\subfile{languages/rust}
\subfile{languages/scala}

\end{document}
