\documentclass[10pt]{article}
\usepackage[margin=1.0in]{geometry}
\usepackage{amsmath}
\usepackage{csquotes}
\usepackage{enumitem}

\title{Thermo Phyics Review}
\author{Arden Rasmussen}
\date{\today}


\begin{document}
\pagenumbering{gobble}
\maketitle
\newpage
\tableofcontents
\newpage
\pagenumbering{arabic}

\section{The Second Law}%
\label{sec:the_second_law}

\subsection{Two State System}%
\label{sub:two_state_system}

\begin{align}
  \Omega(N,n) = \frac{N!}{n!\cdot(N-n)!}\equiv{N \choose n}\\
  \Omega\left(N_{\uparrow}\right)={N\choose
  N_{\uparrow}}=\frac{N!}{N_\uparrow!N_\downarrow!}
\end{align}

\subsection{Einstein Model of a Solid}%
\label{sub:einstein_model_of_a_solid}

\begin{align}
  \Omega(N,q)={q+N-1 \choose q} = \frac{(q+N-1)!}{q!(N-1)!}
\end{align}

\subsection{Interacting Systems}%
\label{sub:interacting_systems}

The spontaneous flow of energy stops when a system is at, or very near, its
most likely macro state, that is, the macro state with the greatest multiplicity.
This ``low of increase of multiplicity'' is one version of the famous
\textbf{second law of thermodynamics}.

\subsection{Large Systems}%
\label{sub:large_systems}

\paragraph{Stirling's Approximation}%
\label{par:stirling_s_approximation}

\begin{align}
  N!\approx N^Ne^{-N}\sqrt{2\pi N}
\end{align}
Often the $\sqrt{2\pi N}$ can be omitted.

\paragraph{Multiplicity of a Large Einstein Solid}%
\label{par:multiplicity_of_a_large_einstein_solid}

\begin{align}
  \Omega(N,q)=e^{N\log\left(\frac{q}{N}\right)}e^N=\left(\frac{eq}{N}\right)^N\quad\text{when
  $q\gg N$}
\end{align}

\paragraph{Sharpness of the Multiplicity Function}%
\label{par:sharpness_of_the_multiplicity_function}

The Gaussian curve has a peak at $x=0$ and a sharp fall-off on either side. The
multiplicity falls of to $\frac{1}{e}$ of its maximum value when
\begin{align}
  N\left(\frac{2x}{q}\right)^2=1\quad\text{or}\quad x=\frac{q}{2\sqrt{N}}
\end{align}
This is a very large number, but comparatively to the full scale it is very
small. The width of the Gaussian is given by
\begin{align}
  \text{Width} = \frac{q}{\sqrt{N}}
\end{align}

\subsection{The Ideal Gas}%
\label{sub:the_ideal_gas}

\paragraph{Multiplicity of a Monatomic Ideal Gas}%
\label{par:multiplicity_of_a_monatomic_ideal_gas}

\begin{align}
  \Omega_N\approx
  \frac{1}{N!}\frac{V^N}{h^{3N}}\frac{\pi^{\frac{3N}{2}}}{\left(\frac{3N}{2}\right)!}\left(\sqrt{2mU}\right)^{3N}\\
  \Omega(U,V,N)=f(N)V^NU^{\frac{3N}{2}}
\end{align}

\subsection{Entropy}%
\label{sub:entropy}

\begin{align}
S\equiv k\log\Omega
\end{align}

\paragraph{Entropy of an Ideal Gas}%
\label{par:entropy_of_an_ideal_gas}

\begin{align}
  S=Nk\left[\log\left(\frac{V}{N}\left(\frac{4\pi
  mU}{3Nh^2}\right)^{\frac{3}{2}}\right)+\frac{5}{2}\right]
\end{align}

If volume changes then

\begin{align}
  \Delta S=Nk\log\frac{V_f}{V_i}\quad\text{U,N fixed}
\end{align}

\paragraph{Entropy of Mixing}%
\label{par:entropy_of_mixing}

\begin{align}
  \Delta S_{total}=\Delta S_{A}+\Delta S_{B}
\end{align}

\section{Interactions and Implications}%
\label{sec:interactions_and_implications}

\subsection{Temperature}%
\label{sub:temperature}

\begin{align}
  \frac{\partial S_A}{\partial U_A}=\frac{\partial S_B}{\partial U_B}\quad
  \text{at equilibrium}\\
  T\equiv \left(\frac{\partial S}{\partial U}\right)^{-1}\\
  \frac{1}{T} \equiv\left(\frac{\partial S}{\partial U}\right)_{N,V}
\end{align}

\subsection{Entropy and Heat}%
\label{sub:entropy_and_heat}

\paragraph{Predicting Heat Capacities}%
\label{par:predicting_heat_capacities}

\begin{align}
  C_V\equiv\left(\frac{\partial U}{\partial T}\right)_{N,V}\\
\end{align}
For an Einstein solid with $q \gg N$
\begin{align}
  C_V=\frac{\partial}{\partial T}\left(NkT\right)=Nk
\end{align}
For a monatomic ideal gas
\begin{align}
  C_V=\frac{\partial}{\partial T}\left(\frac{3}{2}NkT\right)=\frac{3}{2}Nk
\end{align}

\paragraph{Measuring Entropies}%
\label{par:measuring_entropies}

\begin{align}
  dS =\frac{dU}{T}=\frac{Q}{T}\\
  dS=\frac{C_VdT}{T}\\
  \Delta S=S_f-S_i=\int_{T_i}^{T_f}\frac{C_V}{T}dT\\
  C_V\rightarrow0\quad\text{as}\quad T\rightarrow0
\end{align}

\subsection{Paramagnetism}%
\label{sub:paramagnetism}

Weird, and can have negative temperature, as the state with greatest entropy is
when half are up, and half are down, event though more energy is present with
more down spins, it wants to move back to half way, thus it has negative
temperature, but is hotter than high temperature.

\end{document}
