\documentclass[10pt]{article}
\usepackage{amsmath}
\usepackage{csquotes}
\usepackage{enumitem}
\usepackage{fancyhdr}
\usepackage{multicol}
\usepackage{mathtools}
\usepackage{tikz}

\title{Thermo Physics Review}
\date{\today}
\author{Arden Rasmussen}
\fancyhf{}

\newcommand{\V}{{\ooalign{\hfil$V$\hfil\cr\kern0.08em--\hfil\cr}}}
\makeatletter
\DeclareRobustCommand{\pder}[1]{%
  \@ifnextchar\bgroup{\@pder{#1}}{\@pder{}{#1}}}
\newcommand{\@pder}[2]{\frac{\partial#1}{\partial#2}}
\makeatother

\begin{document}
\pagenumbering{gobble}
\maketitle
\newpage
\tableofcontents
\newpage
\pagenumbering{arabic}

\section{3D-Ideal Gas Law}%
\label{sec:3d_ideal_gas_law}

\begin{align}
  P\V&=NkT\\
  Nk&=nR
\end{align}

\subsection{Equipartition Theorem}%
\label{sub:equipartition_theorem}

Only true if $f$ is quadratic. E.g. If kinetic energy and potential energy are
given by quadratic functions

\begin{align}
  y&=nf\frac{1}{2}kT\\
  f&=3\quad\text{Monatomic}\\
  f&=5\quad\text{Diatomic}
\end{align}

\section{1-st Law}%
\label{sec:1_st_law}

\begin{align}
  U&=Q+W
\end{align}

Values are $(+)$ if work or heat is added or done \textit{on}  the system.
Values are $(-)$ if work or heat is removed from or done \textit{by}  the
system.

\begin{align}
  dW=-Pd\V\\
  W=-\int_{\V_i}^{\V_f}Pd\V
\end{align}

\subsection{Cycles}%
\label{sub:cycles}

\begin{figure}[htpb]
  \centering
  \begin{minipage}{.4\linewidth}
    \begin{tikzpicture}[scale=0.5]
      \draw[<->] (6,0) node[below]{$\V$} -- (0,0) -- (0,6) node[left]{$P$};
      \draw[->,line width=1.0pt] (2,2) -- (2,4);
      \draw[->,line width=1.0pt] (2,4) -- (4,4);
      \draw[->,line width=1.0pt] (4,4) -- (4,2);
      \draw[->,line width=1.0pt] (4,2) -- (2,2);
    \end{tikzpicture}
  \end{minipage}
  \begin{minipage}{.4\linewidth}
    \begin{tikzpicture}[scale=0.5]
      \draw[<->] (6,0) node[below]{$\V$} -- (0,0) -- (0,6) node[left]{$P$};
      \draw[<-,line width=1.0pt] (2,2) -- (2,4);
      \draw[<-,line width=1.0pt] (2,4) -- (4,4);
      \draw[<-,line width=1.0pt] (4,4) -- (4,2);
      \draw[<-,line width=1.0pt] (4,2) -- (2,2);
    \end{tikzpicture}
  \end{minipage}
  \caption{Left is Heat engine cycle. Right is Refrigerator cycle.}
\end{figure}

Efficiency is given by benefit divided by cost. Maximum efficiency is only
found in the Caront Cycle.

\begin{align}
  Eff_{max}&=1-\frac{T_C}{T_H}
\end{align}

\section{Gas Expansion}%
\label{sec:gas_expansion}

\begin{description}
  \item[Isothermal] Temperature is constant $T=\text{const}$.
  \item[Abiabatic] No heat is transferred $Q=0$,
    $P\V^{\frac{2+f}{f}}=\text{const}$, $\Delta S =0$.
\end{description}

\begin{align}
  C_\V=\frac{\partial U}{\partial T}&\quad C_P=\frac{\partial U}{\partial
    T}+P\left(\frac{\partial \V}{\partial T}\right)_P\\
  C_\V=\frac{1}{2}fNK&\quad C_P=Nk\left(\frac{f+2}{2}\right)\quad\text{Only for
  ideal gasses}
\end{align}

\section{Multiplicity}%
\label{sec:multiplicity}

\begin{description}
  \item[Einstien Solid]
    $$\Omega=\frac{\left(N+q-1\right)!}{q!\left(N-1\right)!}\approx\left(\frac{eq}{N}\right)^N$$
  \item[Two Level System]
    $$\Omega=\frac{N!}{N_\uparrow!N_\downarrow!}=\frac{N!}{N_\uparrow!\left(N-N_\uparrow\right)!}$$
  \item[Ideal Gas]
    $$\Omega=\frac{\V^N}{N!h^{3N}}\left[\frac{2\pi^{\frac{3N}{2}}}{\left(\frac{3N}{2}-1\right)!}\cdot\left(2mU\right)^{\frac{3N-1}{2}}\right]$$
\end{description}

\section{Entropy and Heat}%
\label{sec:entropy_and_heat}

\paragraph{Heat Capacities}%
\label{par:heat_capacities}

\begin{align}
  C_V\equiv\left(\frac{\partial U}{\partial T}\right)_{N,V}\\
\end{align}
For an Einstein solid with $q \gg N$
\begin{align}
  C_V=\frac{\partial}{\partial T}\left(NkT\right)=Nk
\end{align}
For a monatomic ideal gas
\begin{align}
  C_V=\frac{\partial}{\partial T}\left(\frac{3}{2}NkT\right)=\frac{3}{2}Nk
\end{align}

\paragraph{Measuring Entropies}%
\label{par:measuring_entropies}

\begin{align}
  dS =\frac{dU}{T}=\frac{Q}{T}\\
  dS=\frac{C_VdT}{T}\\
  \Delta S=S_f-S_i=\int_{T_i}^{T_f}\frac{C_V}{T}dT\\
  C_V\rightarrow0\quad\text{as}\quad T\rightarrow0
\end{align}


\section{Thermo-Dynamic Potentials}%
\label{sec:thermo_dynamic_potentials}

\begin{align}
  S&=S(N,\V,U)\\
  dS&=\pder{S}{N}dN+\pder{S}{\V}d\V+\pder{S}{U}dU\\
    &=-\frac{\mu}{T}dN+\frac{P}{T}d\V+\frac{1}{T}dU\\
  dU&=TdS+\mu dN
\end{align}

\begin{align}
  \pder{U}{S}=T\quad\pder{U}{N}=\mu\quad\pder{U}{\V}=-P
\end{align}

\begin{description}
  \item[Enthalpy] $H\equiv U+P\V\quad T=0$
  \item[Helmholts Free Energy] $F\equiv U-TS\quad P=0$
  \item[Gibbs Gree Energy] $G\equiv U-TS+P\V$
\end{description}

\section{Boltzmann Statistics}%
\label{sec:boltzman_statistics}

\subsection{Partition Function}%
\label{sub:partition_function}

\begin{align}
  \Aboxed{Z\equiv \sum_{s}e^{-\frac{E(s)}{kT}}}
\end{align}

Where $s$ are the possible states of a single particle. $Z$ cannot be measured
but it contains all the information relevant to a system.

\subsection{Boltzmann Distribution}%
\label{sub:boltzmann_distribution}

\begin{align}
  \Aboxed{P(s)=\frac{e^{-\frac{E(s)}{kT}}}{Z}}
\end{align}

This gives the probability of a particle being in any given state $s$. Using
this and the partition function, anything about a system can be derived.

\end{document}
