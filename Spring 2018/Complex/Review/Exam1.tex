\documentclass[12pt]{article}

\usepackage[margin=1in]{geometry}
\usepackage{amsmath}
\usepackage{csquotes}
\usepackage{enumitem}

\title{Complex Variables (Exam \#1)}
\author{Arden Rasmussen}
\linespread{1}

\begin{document}
\pagenumbering{gobble}
\maketitle
\newpage
\tableofcontents
\newpage
\pagenumbering{arabic}

\section{Complex Plane}%
\label{sec:complex_plane}

\subsection{How do we represent complex number geometrically?}%
\label{sub:how_do_we_represent_complex_number_geometrically_}

Complex numbers are represented as two dimensional coordinates in a plane, with
the $x$ axis as the real part of the complex number and the $y$ is the
imaginary part of the complex number.

\subsection{What is meant under the following terms?}%
\label{sub:what_is_meant_under_the_following_terms_}

\begin{itemize}
  \item[$Re(z)$] This represents only the real portion of the complex number.
    For example if $z = x + iy$ then $Re(z) = x$.
  \item[$Im(z)$] This represents only the imaginary portion of the complex
    number.  For example if $z = x + iy$ then $Im(z) = y$.
  \item[$|z|$] This is the \textbf{modulus} of $z$, or the length. It is
    calculated as follows: if $z = x+iy$, then $|z| = \sqrt{x^2+y^2}$.
  \item[$\bar{z}$] This is the \textbf{conjugate} of $z$. It is found by
    flipping the sign of the imaginary part of $z$. So if $z=x+iy$, then
    $\bar{z} = x - iy$.
  \item[arg($z$)] This is the \textbf{argument} of $z$. It is the set of all
    angles that represent the position of the complex variable. Since every
    multiple of $2\pi$ is equal, then this is the set of all angles offset by
    some multiple of $2\pi$.
  \item[Arg($z$)] This is the \textbf{principle value} of the argument of $z$.
    It is a single value in the range $-\pi$ to $\pi$ which represents the
    angle that the complex number is at in polar coordinates.
\end{itemize}

\subsection{Properties of modulus and conjugate}%
\label{sub:properties_of_modulus_and_conjugate}



\section{Basic Elementary Functions}%
\label{sec:basic_elementary_functions}

\section{Visualization of Functions}%
\label{sec:visualization_of_functions}

\section{Riemann surfaces and branches of multivalued functions}%
\label{sec:riemann_surfaces_and_branches_of_multivalued_functions}

\section{Functions of complex variable defined through series}%
\label{sec:functions_of_complex_variable_defined_through_series}

\section{Holomorphic functions and Cauchy-Riemann equations}%
\label{sec:holomorphic_functions_and_cauchy_riemann_equations}

\end{document}
