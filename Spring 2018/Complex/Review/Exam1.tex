\documentclass[12pt]{article}

\usepackage[margin=1in]{geometry}
\usepackage{amsmath}
\usepackage{csquotes}
\usepackage{enumitem}

\title{Complex Variables (Exam \#1)}
\author{Arden Rasmussen}
\linespread{1}

\begin{document}
\pagenumbering{gobble}
\maketitle
\newpage
\tableofcontents
\newpage
\pagenumbering{arabic}


\section{Complex Plane}%
\label{sec:complex_plane}

\subsection{Conjugate}%
\label{sub:conjugate}

\paragraph{Definitions}%
\label{par:definitions}

\begin{align*}
  \bar{z} = \overline{a+ib}=a-ib
\end{align*}

\paragraph{Properties}%
\label{par:properties}

\begin{align*}
  \overline{z_1 + z_2}&=\bar{z_1} + \bar{z_2}\\
  \overline{z_1 \cdot z_2}&-\bar{z_1} \cdot \bar{z_2}
\end{align*}

\subsection{Modulus}%
\label{sub:modulus}

\paragraph{Definition}%
\label{par:definition}

\begin{align*}
  |z| &= \sqrt{x^2+y^2}\\
  |z| &= \sqrt{z\cdot \bar{z}}
\end{align*}

\paragraph{Properties}%
\label{par:properties}

\begin{align*}
  |z_1 \cdot z_2| &= |z_1| \cdot |z_2|
\end{align*}

\subsection{Triangle Inequality}%
\label{sub:triangle_inequality}

\begin{align*}
  |z_1+z_2|\leq |z_1|+|z_2|
\end{align*}

\subsection{Exponential Form}%
\label{sub:exponential_form}

\begin{align*}
  z &= x+iy\\
  z &= |z|e^{i\text{Arg}(z)}
\end{align*}

\paragraph{product}%
\label{par:product}

\begin{align*}
  r_1e^{i\theta_1} \cdot r_2e^{i\theta_2} &= \left(r_1 \cdot
    r_2\right)e^{i\left(\theta_1 + \theta_2\right)}
\end{align*}

\paragraph{Quotient}%
\label{par:quotient}

\begin{align*}
  \frac{r_1e^{i\theta_1}}{r_2e^{i\theta_2}} &= 
  \left(\frac{r_1}{r_2}\right)e^{i\left(\theta_1 - \theta_2\right)}
\end{align}

\paragraph{n-th power}%
\label{par:n_th_power}

\begin{align*}
  {\left(re^{i\theta}\right)}^n
\end{align*}

\section{Basic Elementary Functions}%
\label{sec:basic_elementary_functions}

\subsection{Exponential Function}%
\label{sub:exponential_function}

\begin{align*}
  \exp(z) &= \sum_{n=0}^{\infty}\frac{z^n}{n!}\\
  \exp(x+iy) &= e^x\cos y + ie^x \sin y
\end{align*}

\subsection{Logarithmic Function}%
\label{sub:logarithmic_function}

\begin{align*}
  \log(z) &= \ln(|z|)+i\text{arg}(z)\quad\text{Multi-valued}\\
  \text{Log}(z) &= \ln(|z|)+i\text{Arg}(z)\quad\text{Single-Valued}
\end{align*}

\subsection{Power Function}%
\label{sub:power_function}

\begin{align*}
  z^w&=\exp(w\log(z))\quad\text{Multi-valued}\\
  P.V.z^w&=\exp(w\text{Log}(z))\quad\text{Single-valued}
\end{align*}

\subsection{Trigonometic Functions}%
\label{sub:trigonometic_functions}

\paragraph{sin}%
\label{par:sin}

\begin{align*}
  \sin(z) &= \frac{\exp(iz)-\exp(-iz)}{2i}\\
  \sin(z) &= z - \frac{z^3}{6}+\frac{z^5}{120}-\cdots\\
  \sin(z) &= \sin(x)\cosh(y)+i\cos(x)\sinh(y)\\
  \sin(z) &= -i\sinh(iz)
\end{align*}

\paragraph{cos}%
\label{par:cos}

\begin{align*}
  \cos(z) &= \frac{\exp(iz)+\exp(-iz)}{2i}\\
  \cos(z) &= 1 - \frac{z^2}{2}+\frac{z^4}{24}-\cdots\\
  \cos(z) &= \cos(x)\cosh(y)-i\sin(x)\sinh(y)\\
  \cos(z) &- \cosh(iz)
\end{align*}

\paragraph{sinh}%
\label{par:sinh}

\begin{align*}
  \sinh(z) &= \frac{\exp(z)-\exp(-z)}{2}\\
  \sinh(z) &= z + \frac{z^3}{6}+\frac{z^5}{120}+\cdots\\
  i\sinh(z)&= \sin(iz)
\end{align*}

\paragraph{cosh}%
\label{par:cosh}

\begin{align*}
  \cosh(s) &= \frac{\exp(z)+\exp(-z)}{2}\\
  \cosh(z) &= 1 + \frac{z^2}{2}+\frac{z^4}{24}+\cdots\\
  \cosh(z) &= \cos(iz)
\end{align*}

\section{Holomorphic functions and Cauchy-Riemann equations}%
\label{sec:holomorphic_functions_and_cauchy_riemann_equations}

\subsection{Terminology}%
\label{sub:terminology}

\paragraph{Differentiable}%
\label{par:differentiable}

If a function is differentiable, then the Jacobi matrix exists.

\paragraph{Holomorphic}%
\label{par:holomorphic}

A function $f(z)$ is holomorphic if
\begin{align*}
  \lim_{h \rightarrow 0}\frac{f(z+h)-f(z)}{h}
\end{align*}

exists.

If a function is holomorphic, it implies that it is differentiable and the
Cauchy-Remann equations hold for it.

\paragraph{Analytic}%
\label{par:analytic}

A function is analytic if it is expressable as a sum of power series.

\begin{align*}
  \sum a_n{\left(z-z_0\right)}^n
\end{align*}

If a function is analytic then it is also holomorphic.

\subsection{Jacobian}%
\label{sub:jacobian}

\begin{align*}
  Df = \begin{pmatrix}
    \frac{\partial u}{\partial x} & \frac{\partial y}{\partial{y}}\\
    \frac{\partial v}{\partial x} & \frac{\partial v}{\partial{y}}\\
  \end{pmatrix}
\end{align*}

\subsection{Cauchy-Riemann Equations}%
\label{sub:cauchy_riemann_equations}

$f(z)$ is differentiable exactly when

\begin{align*}
  \frac{\partial u}{\partial x}&=\frac{\partial v}{\partial y}\\
  \frac{\partial v}{\partial x}&=-\frac{\partial u}{\partial y}\\
\end{align*}

\paragraph{Derivation}%
\label{par:derivation}

\begin{align*}
   z &= x+iy\\
   f(z) &= u(x,y)+iv(x,y)\\
   &= \lim_{h\rightarrow0}\frac{f(z+h)-f(z)}{h}\\
   f'(z) &= \lim_{a+ib\rightarrow0}\left[\frac{y(x+a, y+b)+iv(x+a,y+b)-u(x,y)-iv(x,y)}{a+ib}\right]\\
         &= \lim_{a\rightarrow0}\left[\lim_{b\rightarrow0}\left[\frac{y(x+a,
   y+b)+iv(x+a,y+b)-u(x,y)-iv(x,y)}{a+ib}\right]\right]\\
   &=
   \lim_{a\rightarrow0}\left[\frac{u(x+a,y)-u(x,y)}{a}+i\frac{v(x+a,y)-v(x,y)}{a}\right]\\
   &= \boxed{\frac{\partial u}{\partial x}+i\frac{\partial v}{\partial x}}\\
         &= \lim_{b\rightarrow0}\left[\lim_{a\rightarrow0}\left[\frac{y(x+a,
   y+b)+iv(x+a,y+b)-u(x,y)-iv(x,y)}{a+ib}\right]\right]\\
   &=
   \lim_{b\rightarrow0}\left[\frac{u(x,y+b)-u(x,y)}{ib}+i\frac{v(x,y+b)-v(x,y)}{ib}\right]\\
   &= \boxed{-i\frac{\partial u}{\partial y}+\frac{\partial v}{\partial y}}\\
   \frac{\partial u}{\partial x}+i\frac{\partial v}{\partial x} &=
   -i\frac{\partial u}{\partial y} + \frac{\partial v}{\partial y}\\
   &=\begin{cases}
     \frac{\partial y}{\partial x} = \frac{\partial v}{\partial y}\\
     \frac{\partial v}{\partial x} = -\frac{\partial u}{\partial y}
   \end{cases}
\end{align*}

\end{document}
