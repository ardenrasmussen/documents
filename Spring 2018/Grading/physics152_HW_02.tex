\documentclass[10pt]{article}
\usepackage{amsmath}
\usepackage{graphicx}
\usepackage{nameref}
\usepackage{enumitem}
\usepackage{empheq}

\title{152 HW 02}

\newcommand{\boxeq}[1]{%
  \begin{center}%
    \boxed{#1}%
  \end{center}%
}

\begin{document}
\section*{Problem 1}
\label{sec:problem_1}

An oscillator consists of a block of mas $m=0.40kg$ attacked to a spring. When
set into oscillation with amplitude $A=6.50cm$, the oscillator repeats its
motion every $0.40s$. Find the (a) period, (b) frequency, (c) angular
frequency, (d) spring constant, (e) maximum speed, and (f) magnitude of the
maximum force on the block from the spring.

\paragraph{a)}
\label{par:a_}

By definition, the period is the time it take for the motion to repeat
itself.

\boxeq{T=0.40s}

\paragraph{b)}
\label{par:b_}

The frequency (\# of cycles per second) is simply the inverse of the period

\boxeq{f=\frac{1}{T}=\frac{1}{0.40s}=2.5Hz}

\paragraph{c)}
\label{par:c_}

The angular frequency is how many radians per second the oscillators go through,
and there are are $2\pi$ radians per cycles, so

\boxeq{\omega=2\pi f=\frac{2\pi}{T} = 15.7\frac{rad}{sec}}

\paragraph{d)}
\label{par:d_}

For a simple harmonic oscillator consisting of a mass on a spring,

\begin{align*}
  \omega=\sqrt{\frac{k}{m}} \Rightarrow k=m\omega^2
\end{align*}

so

\boxeq{k=(0.4kg)(15.7)^2=98.7\frac{N}{m}}

\paragraph{e)}
\label{par:e_}

There are a few ways to approach this...

\subparagraph{I)}
Use $x(t) = Acos(\omega t)$ to get $v(t)=-A\omega sin(\omega t)$ then say
$v_{max} = A\omega$.

\subparagraph{II)}
\begin{align*}
  E_{oscillator} = \frac{1}{2}kA^2=\frac{1}{2}mv^2_{max}\\
  V^2_{max}=\frac{k}{m}A^2\\
  V_{max}=\sqrt{\frac{k}{m}}A=A\omega
\end{align*}

All give the same answer

\boxeq{V_{max}=A\omega =(0.065m)(15.7\frac{1}{sec})\approx1.0\frac{m}{2}}

\paragraph{f)}
\label{par:f_}

Similarly to \nameref{par:e_} there is more than one way to approach this.

\subparagraph{I)}
Use $x(t)=Acos(\omega t)$ to get $a(t)=-A\omega^2cos(\omega t)$ so
$a_{max}=A\omega^2$.

\subparagraph{II)}
Or
\begin{align*}
  F(x) = m\cdot a =-kx\\
  x_{max/min} = \pm A\\
  a = -\frac{kx}{m}\\
  a_{max} = \frac{k}{m}A\\
  a_{max} = \omega^2A
\end{align*}

\boxeq{a_{max}=\omega^2A=\left(15.7\frac{1}{sec}\right)^2(0.065m)=16\frac{m}{s^2}}

\section*{Problem 2}
\label{sec:problem_2}

Two execute simple harmonic motion of the same amplitude and frequency along
close parallel lines. They pass each other moving in opposite directions when
their displacement is half the amplitude of the motion what is the phase
difference between the two oscillators?

Let's imagine the two oscillators like this. Same m's and k's so
$\omega=\sqrt{\frac{k}{m}}$ is the same.

Each oscillator's position can be written as.

\begin{align*}
  x_1(t)=Acos(\omega t+\delta_1)\\
  x_2(t)=Acos(\omega t+\delta_2)
\end{align*}

Lets say $\delta_1 = 0$ (pick time origin arbitrarily). So

\begin{align*}
  x_1(t) = Acos(\omega t)\\
  x_2(t) = Acos(\omega t + \delta_2)
\end{align*}

At the instant in time shown in the picture above,

\begin{align*}
  x_1=x_2=\frac{A}{2}
\end{align*}

So, at that particular $t$,
\begin{align*}
  Acos(\omega t) = Acos(\omega t+\delta_2)=\frac{A}{2}
\end{align*}
so,
\begin{align*}
  cos(\omega t) = cos(\omega t + \delta_2)=\frac{1}{2}
\end{align*}

but we also know that $v_1=-v_2$

\begin{align*}
  v_1(t) = -A\omega sin(\omega t)\\
  v_2(t) = -A\omega sin(\omega t + \delta_2)
\end{align*}

so at the particular $t$ for the picture,

\begin{align*}
  A\omega sin(\omega t) = -A\omega sin(\omega t+\delta_2)\\
  sin(\omega t) = -sin(\omega t+\delta_2)
\end{align*}

Remembering the relation of the unit circle to the trig functions:
so if $\omega t = \frac{\pi}{3}$ and $\omega t+\delta_2 = -\frac{\pi}{3}$ all
our trig relations work, so

\boxeq{\delta_2=-\frac{2\pi}{3}}

\section*{Problem 3}
\label{sec:problem_3}

A piston moves up and down in harmonic motion with period $T=1.0s$. A block is
place don top of the piston at the bottom of the up-down motion. (a) What is
the minimum amplitude for the piston's harmonic motion such that the block and
piston lose contact? (b) If the amplitude of the piston's harmonic motion is
now $A=0.75m$ and the period is still $T=1.0s$, at what displacement do the
block and piston lose contact?

$T=1.0s \rightarrow
\omega=\frac{2\pi}{T}=2\pi\frac{rad}{sec}=6.28\frac{rad}{sec}$

\paragraph{a)}
\label{par:a_}

For what minimum amplitude of oscillation will the block lose contact with the
piston?

Well, for the block to lose contact with the piston, it must be in free-fall, so
its acceleration must be $-g$, so the piston must also have an acceleration
down of at least $g$.

Position of piston as a function of time is

\begin{align*}
  x(t) = Acos(\omega t)\\
  v(t) = \frac{dx}{dt} = -A\omega sin(\omega t)\
  a(t) = \frac{dv}{dt} = -A\omega^2cos(\omega t)
\end{align*}
so, for $a(t) = -g = -A\omega^2cos(\omega t)$ the minimum

\boxeq{A_{min}=\frac{g}{\omega^2}=\frac{9.8\frac{m}{s^2}}{\left(6.28\frac{rad}{sec}\right)^2}=0.25m}

\paragraph{b)}
\label{par:b_}

Now, the amplitude of oscillation has increased to $A=0.75m$. At what
displacement (i.e. position) does the block lose contact from the piston?

By graphing it would appear that the position of the piston at which $a(t)=-g$
is also $0.25m$. Actually for any amplitude, we can do the following:

\begin{align*}
  a(t) = -g = -A\omega^2cos(\omega t)\\
  cos(\omega t) = \frac{g}{A\omega^2} \Rightarrow \omega
  t=cos^{-1}\left[\frac{g}{a\omega^2}\right]
\end{align*}

Plug into $x(t)$ (position)

\begin{align*}
  x(t) &= Acos(\omega t)\\
       &= Acos\left[cos^{-1}\left[\frac{g}{A\omega^2}\right]\right]
\end{align*}

\boxeq{x(t)=\frac{g}{\omega^2}=0.25m}

\section*{Problem 4}
\label{sec:problem_4}

A grandfather clock utilizes a pendulum made with a brass rod of length
$l_{rod}=1m$ and mass $m_{rod}=0.25kg$ attacked to the center of a brass disk
with mass $m_{disk}=3.5kg$ and radius $r_{disk}=10cm$. The clock counts 2
seconds for each full cycle of the pendulum. (a) Treat the pendulum as a simple
pendulum of length $l_{rod}$ making small oscillations, and calculate how far
off the grandfather clock's time would be from the ``correct'' time after 24
hours (counted in the correct time), be sure to include the sign (i.e. is the
grandfather clock fast or slow). (b) Treat the pendulum as a physical pendulum
making small oscillations, and do the same calculation. (c) Calculate how far
up or down on the rod the disk should be moved to have the grandfather clock
keep ``correct'' time.

Clock counts 2 seconds for each full swing of pendulum.

\paragraph{a)}
\label{par:a_}

Treat pendulum as a simple pendulum of length $l=1m$.

\begin{align*}
  \omega_0=\sqrt{\frac{g}{l}}=\sqrt{\frac{9.81}{1.0}} = 3.13 \frac{rad}{s}\\
  T = \frac{2\pi}{\omega_0}=\frac{2\pi}{3.13}=2.007 s
\end{align*}

So the grandfather clock is slow by $7ms=7x10^{-3}s$ for each oscillation. Or
put another way for each ``true'' second that elapses, the grandfather clock is
slow by $3.5ms$.

How many seconds are in $24hours$?

\begin{align*}
  24hrs\frac{60min}{hr}\frac{60sec}{min}=8.64\times 10^4s
\end{align*}
So the total amount the grandfather clock is slow by is

\begin{align*}
   3.5ms*8.64\times 10^4s
\end{align*}

\boxeq{\approx 300\ seconds = 5\ minutes\ slow}

So if we consider the pendulum as a simple pendulum it predicts that the clock
would lose by 5 minutes every day.

\paragraph{b)}
\label{par:b_}

Now, we know that the rod/disk system is not a simple pendulum, but rather, it
is a physics pendulum, so lets consider that.

We know that for a physical pendulum, $\omega_0=\sqrt{\frac{MgD}{I}}$.

\begin{align*}
  M&=m_{rod}+m_{disk}\\
  D&=x_{c.o.m.}=\frac{m_{rod}\frac{l}{2}+m_{disk}l}{M}\\
  I&=\frac{1}{3}m_{rod}l^2+\frac{1}{2}m_{disk}r_{disk}^2+m_{disk}l^2\\
  \\
  \omega_0 &=
           \sqrt{\frac{Mg\frac{m_{rod}\frac{l}{2}+m_{disk}l}{M}}{\frac{1}{3}m_{rod}l^2+\frac{1}{2}m_{disk}r_{disk}^2+m_{disk}l^2}}\\
           &= \sqrt{\frac{9.81*\left(0.25\frac{1}{2}+3.5\right)}{\frac{1}{3}(0.25)+\frac{1}{2}(3.5)(1.0)^2+3.5}}\\
           &= 3.141\frac{rad}{sec} \Rightarrow T=2.0004s
\end{align*}

So, looking at the physical pendulum result, we would predict that the
grandfather clock would be slow by $0.2ms$ for every ``true'' second that
elapses.

Adding this up over the course of an entire day...

\boxeq{\approx 17\ seconds\ slow}

\paragraph{c)}
\label{par:c_}

OK, we are pretty close to getting a grandfather clock that keep accurate time,
but we can do better. We are going to move the disk to adjust the frequency of
the pendulum as that one period of the pendulum is exactly 2 seconds. So

\begin{align*}
  T = \frac{2\pi}{\omega_0}=2\pi\sqrt{\frac{I}{MgD}} = 2s
\end{align*}

now because we move the disk we change $D$ and $I$ so
\begin{align*}
  s\pi\sqrt{\frac{\frac{1}{3}m_{rod}l^2+\frac{1}{2}m_{disk}r^2+m_{disk}x^2}{g\left(m_{rod}\frac{l}{2}+m_{disk}x\right)}}=
  2 s
\end{align*}

where $x$ is the new position of the disk that adjusts the clock to keep
perfect time.

We could square it all out and solve for $x$ but... By Mathematica...

\begin{align*}
  x_{disk} = 0.9996m
\end{align*}

So we want to move the disk up the rod by 

\boxeq{\Delta x=-0.4mm}

If one uses $g=9.81\frac{m}{s^2}$ the answers are:
\boxeq{262s\approx4.5\ minutes\ slow}
\boxeq{27\ seconds\ fast}
\boxeq{\Delta x = +0.65mm}

\section*{Problem 5}
\label{sec:problem_5}

Recently a large meteor broke up in the night sky over Michigan, leading to some
dramatic videos and meteorite finding on the ground. The explosion of the
meteor in the sky registered a 2.0 on the USGS seismic monitors. Let's
consider something more dramatic...

Let's consider a meteor traveling at $|v|=1.5\times 10^5\frac{m}{s}$, with mass
$m_{meteor}=1\times 10^6 kg$. Assume that when this meteor hits the Earth, all
the kinetic energy is converted into oscillations of the earth in the radial
direction. Use the following quantities, effective spring constant
$k_{eff}=7.5\times 10^{17}\frac{N}{m}$, period $T=54min$, and quality factor
$Q=400$ to calculate: (a) the initial amplitude of the oscillation, (b) the
amplitude of oscillation after 10 cycles, (c) the number of oscillations before
the amplitude of oscillation is less than 1 mm.

\paragraph{a)}
\label{par:a_}

\begin{align*}
  E_{oscillator}=\frac{1}{2}kA^2&=E_{meteor}=\frac{1}{2}m_{meteor}v_{meteor}^2\\
  A&=\sqrt{\frac{m_{meteor}}{k_{Earth}}}v_{meteor}\\
   &=\sqrt{\frac{1\times 10^8}{7.5\times 10^{17}}}1.5\times 10^5\\
\end{align*}

\boxeq{A=0.17m=17cm}

\paragraph{b)}
\label{par:b_}

OK, now we want to know what the amplitude of oscillation will be after 10
cycles.

We start with the equation for the amplitude

\begin{align*}
  A(t) = A_0e^{-\frac{t}{2\tau}}
\end{align*}

First, after 10 oscillations, $t=10T$, and $Q=400$. but $Q\equiv \omega_o
\tau$, and $\omega_0=\frac{2\pi}{T}$ so $\tau = \frac{QT}{2\pi}$. Putting it all
back into $A(t)$.

\boxeq{A_{10}=15.7cm}

\paragraph{c)}
\label{par:c_}

Now, how long does it take to gut the amplitude down to 1mm?

\begin{align*}
  A(t) = 1\times 10^{-3}=(0.17m)e^{-\frac{t}{2\pi}}\\
  1\times 10^{-3}=0.17e^{-\frac{t\pi}{QT}}\\
  -\frac{t\pi}{QT}=ln\left[\frac{1\times10^{-3}}{0.17}\right]=-5.1\\
  t=\frac{5.1QT}{\pi}
\end{align*}
\boxeq{t=3.5\times 10^4min}

\section*{Problem 6}
\label{sec:problem_6}

A car of mass $m_{car}=1000kg$ with 4 passengers (each with mass
$m_{pass}=82kg$) is driving down a road with bumps spaced at a distance $d=4m$
apart. The car bounces with maximum amplitude on the suspension springs when
the speed of the car is $|v|=16km/h$. (a) When the car stops and all four
passengers get out of the car, how much does the car rise? One of the
passengers installs dampening on the suspension springs to smooth the ride out.
She designed the damping so the amplitude of the oscillation at $|v|=5km/h$ is
$1/2$ the maximum oscillation amplitude (same road, same passengers). (b) What
is the value of the damping constant $b$ that she chose to make this happen?
(c) At what speed $|v|>16km/h$ does the car's oscillation amplitude equal half
the maximum oscillation amplitude?

\paragraph{a)}
\label{par:a_}

\begin{align*}
  |v| &= 16km/h=4.44\frac{m}{s}\\
  T&=\frac{4m}{4.44\frac{m}{s}} = 0.9s\\
  \omega &= \omega_0=\frac{2\pi}{T}=7\frac{rad}{s}\\
  \omega_0&=\sqrt{\frac{k}{m}} = 7\frac{rad}{s}\\
          &=\sqrt{k}{1328kg} = 7\\
  k &= 6.5\times 10^4 \frac{N}{m}\\
  \Delta x &= |x_1-x_2|=\frac{Mg-mg}{k}\\
\end{align*}

\boxeq{\Delta x = 5cm}

\paragraph{b)}
\label{par:b_}

\begin{align*}
  A(5)=\frac{1}{2}A(16)\\
  5\rightarrow \omega = \frac{2\pi}{T}=\frac{2\pi}{4m}1.4=2.2\frac{rad}{s}\\
  A(2.2)=\frac{1}{2}A(7)\\
  A(\omega)=\frac{F_0}{\sqrt{m^2\left(\omega_0^2-\omega^2\right)+b^2\omega^2}}\\
  \frac{\sqrt{m^2\left(\omega_0^2-\omega^2_0\right)+b^2\omega_0^2}}{\sqrt{m^2\left(\omega_0^2-\omega^2\right)+b^2\omega^2}}
  = \frac{1}{2}\\
  \frac{b\omega_0}{\sqrt{1328^2(49-4.84)^2+b^2)4.84)}} = \frac{1}{2}\\
  196b^2=1328^244^2+b^24.84\\
  191b^2=3.4\times 10^9
\end{align*}

\boxeq{b=4.2\times 10^3\frac{kg}{s}}

\paragraph{c)}
\label{par:c_}

If
\begin{align*}
  \left(\omega_0^2-\omega^2\right) = 49-4.84=44\frac{rad^2}{s^2}
\end{align*}
gives $\frac{A}{2}$, the nearly symmetric behavior of $A(\omega)$ would give
approximately
\begin{align*}
  A(\omega) = \frac{A_{max}}{w}\\
  \omega^2=(49+44)\frac{rad^2}{s^2}\\
  \omega=9.6\frac{rad}{sec}
\end{align*}

This corresponds to $T=\frac{2\pi}{\omega}=0.65sec$. So every $0.65$ seconds.
the car hits a bump. So this means the speed $|v|$ is given by
\begin{align*}
  |v| = \frac{4}{0.65}=6\frac{m}{s}
\end{align*}

and converting to $\frac{km}{h}$
\boxeq{|v| = 22\frac{km}{h}}

Let's check to make sure that our assumption of an approximately symmetric
$A(\omega)$ was correct, so does

\begin{align*}
  \frac{b\omega_0}{\sqrt{m^2\left(\omega_0^2-\omega62\right)+b^2\omega^2}}
  \approx \frac{1}{2}\\
  \frac{2.3\times 10^37}{\sqrt{1328^244+4.2^29.6^2}}=0.42
\end{align*}
Not quite!.

It's only a a quadratic, but I'll still use Mathematica to solve exactly
\begin{align*}
  \omega = 9.1\frac{rad}{s}
\end{align*}

\boxeq{|v|=5.8\frac{m}{s}=20.9\frac{km}{h}}

\end{document}
