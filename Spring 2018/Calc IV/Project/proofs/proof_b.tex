\documentclass[12pt]{amsart}

\usepackage{amsfonts}
\usepackage{amsthm}
\usepackage{amsmath}
\usepackage{amscd}
\usepackage{epstopdf}
\usepackage{csquotes}
\usepackage{enumitem}
\usepackage{mathrsfs}
\usepackage{pgfplots}
\usepackage{hyperref}

\usepackage[draft]{todonotes}

\newcommand{\R}{\mathbb{R}}
\newcommand{\Z}{\mathbb{Z}}
\newcommand{\LP}{L^2(\mathbb{R})}
\newcommand{\norm}[1]{\left\lVert#1\right\rVert}
\newcommand{\abs}[1]{\left\lvert#1\right\rvert}
\newcommand{\nref}[1]{\nameref{#1}}

\newtheorem{thm}{Theorem}[section]
\newtheorem{corollary}{Corollary}[thm]
\newtheorem{lemma}[thm]{Lemma}
\newtheorem{case}{Case}
\usepackage{etoolbox}
\AtEndEnvironment{proof}{\setcounter{case}{0}}

\begin{document}

\begin{proof}\label{proof:2}
  $j=j_1=j_2$, and $k_1 \new k_2$. To show orthogonality, we must show that the
  value of the inner product of the two different functions is zero.
  \begin{align}
    &\left<2^{\frac{j}{2}}\Psi\left(2^jx-k_1\right),
    2^{\frac{j}{2}}\Psi\left(2^jx-k_2\right)\right>\\
    &=2^j\int_{-\infty}^{\infty}\Psi\left(2^jx-k_1\right)\Psi\left(2^jx-k_2\right)dx
  \end{align}
  Similarly to proof \ref{proof:1}, we want to change the bounds of
  integration. We can use much of the work from proof \ref{proof:1}.
  \begin{case}\label{case:2.1}
    $k_1>k_2$. We can then write an expression for $k_1$ in terms of $k_2$.
    $k_1 = \alpha + k_2$, where $\alpha \geq 1$. Taking the lower bound for
    $\Psi_1$, we can say
    \begin{align}
      \frac{1}{2^j}(\alpha+k_2)
    \end{align}
    Once again we are able to ignore the $\frac{1}{2^j}$, Now we can notice
    \begin{align}
      1+k_2 \leq \alpha + k_2
    \end{align}
    Thus for any values of $j$, and $k_1$, and $k_2$, then there will be no
    overlap between the functions, and so the integral will always be zero.
  \end{case}
  \begin{case}\label{case:2.2}
    $k_2>k_1$. Without loss of generality we can apply the same process as in
    case \ref{case:2.1}, because the multiplication of two functions is
    commutative.
  \end{case}
  Thus we are able to conclude that for any case where $k_1 \neq k_2$ and $j_1
  =j_2$ that the two wavelets will not overlap, and thus they will be
  orthogonal.
\end{proof}

\end{document}
