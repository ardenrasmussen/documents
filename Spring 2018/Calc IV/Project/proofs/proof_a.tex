\documentclass[12pt]{amsart}

\usepackage{amsfonts}
\usepackage{amsthm}
\usepackage{amsmath}
\usepackage{amscd}
\usepackage{epstopdf}
\usepackage{csquotes}
\usepackage{enumitem}
\usepackage{mathrsfs}
\usepackage{pgfplots}
\usepackage{hyperref}

\usepackage[draft]{todonotes}

\newcommand{\R}{\mathbb{R}}
\newcommand{\Z}{\mathbb{Z}}
\newcommand{\LP}{L^2(\mathbb{R})}
\newcommand{\norm}[1]{\left\lVert#1\right\rVert}
\newcommand{\abs}[1]{\left\lvert#1\right\rvert}
\newcommand{\nref}[1]{\nameref{#1}}

\newtheorem{thm}{Theorem}[section]
\newtheorem{corollary}{Corollary}[thm]
\newtheorem{lemma}[thm]{Lemma}
\newtheorem{case}{Case}
\usepackage{etoolbox}
\AtEndEnvironment{proof}{\setcounter{case}{0}}

\begin{document}

\begin{proof}\label{proof:1}
  $j_1 \neq j_2$, and $k = k_1 = k_2$. To show orthogonality, we must show that
  the value of the inner product of the two different functions is zero.
  \begin{align}
    &\left<2^{\frac{j_1}{2}}\Psi\left(2^{j_1}x-k\right),2^{\frac{j_2}{2}}\Psi\left(2^{j_2}x-k\right)\right>\\
    &=
    2^{\frac{j_1+j_2}{2}}\int_{-\infty}^{\infty}\Psi\left(2^{j_1}x-k\right)\Psi\left(2^{j_2}x-k\right)dx
  \end{align}
  Because the only parts of the $\Psi$ function that will effect the integral
  is when the inputs are between zero and one. Thus we will want to change our
  bounds of integration to match these bounds. In order to determine these new
  bounds, we will want to solve $2^{j}x-k$ equal to $0$, $\frac{1}{2}$, and
  $1$. We want to solve for the $\frac{1}{2}$ value, because that is the
  discontinuity in the $\Psi$ function, and will become important. We will
  solve these values for both $j_1$ and $j_2$.
  \begin{align}
    0 = 2^jx-k &\Rightarrow x = \frac{k}{2^j} = \frac{1}{2^j}\cdot (k)\\
    \frac{1}{2} = 2^jx-k &\Rightarrow x =
    \frac{1}{2\cdot 2^j}+\frac{k}{2^j} =
    \frac{1}{2^j}\left(\frac{1}{2}+k\right)\\
    1 = 2^jx-k &\Rightarrow x = \frac{1}{2^j}+\frac{k}{2^j} = \frac{1}{2^j}(1+k)
  \end{align}
  We can now use these values, by plugging in $j_1$, and $j_2$ for the bounds
  of integration for each of the $\Psi$ functions. However, this also will
  create some cases for the proof.
  \begin{case}\label{case:1.1}
    $j_1 > j_2$, $k\in\Z^+\cup\{0\}$. We can then write an expression for $j_1$
    in terms of $j_2$, $j_1 = \alpha \cdot j_2$, where $\alpha \geq 1$. Taking
    the upper bound for $j_1$, we can say
    \begin{align}
     \frac{1}{2^\alpha\cdot2^{j_2}}(1+k)\\
     \frac{1}{2^{j_2}}\left(\frac{1}{2^\alpha}+\frac{k}{2^\alpha}\right)
    \end{align}
    We can ignore the $\frac{1}{2^{j_2}}$ as that appears in all of our
    bounds, and so we are able to ignore it for all cases. Now we are able to
    compare our statement to the bounds of the other function with $j_2$. We
    can see that there are two cases when comparing the bounds.
    \begin{align}
      \begin{cases}
        \frac{1}{2^\alpha} \leq \frac{1}{2} &\quad k =0\\
        \frac{1}{2^\alpha}(1+k) \leq k &\quad k >0
      \end{cases}
    \end{align} 
    The first case when $k=0$ states that the upper bound of $\Psi_1$ is
    always less then or equal to the middle bound of $\Psi_2$. Thus the
    integral will just be from $0$ to $\frac{1}{2^\alpha}$, and since $\Phi_2$
    is $1$ for the entire integral, then it is just an integral of $\Psi_1$,
    when results in zero.
    \begin{align}
     \int_{0}^{\frac{1}{2^{j_1}}}\Psi(x)dx = 0
    \end{align}
    The second case when $k>0$ states that the upper bound of $\Psi_1$ is
    always less then or equal to the lower bound of $\Psi_2$. Since $\Psi$ is
    defined as zero outsize of their bounds, then there is no overlap of the
    functions and the resulting integral will become zero.
  \end{case}
  \begin{case}\label{case:1.2}
    $j_1>j_2$, $k\in Z^-$. We can make use of the expression we found in
    case \ref{case:1.1}. However, now instead of analyzing the upper bound of
    $\Psi_1$ we will analize the lower bound of $\Psi_1$.
    \begin{align}
      \frac{1}{2^\alpha\cdot 2^{j_2}}(k)
    \end{align}
    Once again we will ignore the $\frac{1}{2^{j_2}}$. Again two cases will
    appear when comparing the bounds.
    \begin{align}
       \begin{cases}
         -\frac{1}{2^\alpha} \geq -\frac{1}{2} &\quad k=-1\\
         \frac{k}{2^\alpha} \geq (1+k) &\quad k<-1
       \end{cases}
    \end{align}
    In the first case when $k=-1$, it is a mirrored representation of the
    first case in case \ref{case:1.1}, and will also results in zero. The
    second case when $k<-1$, states that the lower bound of $\Psi_1$ is always
    greater then or equal to the upper bound of $\Psi_2$, thus there is once
    again no overlap between the function, and the integral will be zero.
  \end{case}
  \begin{case}\label{case:1.3}
    $j_2>j_1$, $k\in \Z^+ \cup \{0\}$. Without loss of generality we can apply
    the same process as in case \ref{case:1.1}, because the multiplication of
    two functions is commutative.
  \end{case}
  \begin{case}\label{case:1.4}
    $j_2>j_1$, $k\in \Z^-$. Without loss of generality we can apply
    the same process as in case \ref{case:1.2}, because the multiplication of
    two functions is commutative.
  \end{case}

  Thus we are able to conclude that for any case where $j_1 \neq j_2$ and
  $k_1=k_2$ that the two wavelet functions will be orthogonal.
\end{proof}

\end{document}
