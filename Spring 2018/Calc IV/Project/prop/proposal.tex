\documentclass[10pt]{article}
\usepackage{amsmath}
\usepackage{csquotes}
\usepackage{enumitem}
\usepackage{hyperref}
\hypersetup{
    colorlinks=true,
  }

\title{Proposal}
\author{Arden Rasmussen & Grant Mohn}


\begin{document}
\maketitle
One subject that we are interested in doing more research into is that of
\textit{wavelets}. We find them interesting, because they represent a different
method from the Fourier series that can approximate functions. The other
benefits of wavelets is that they allow different accuracy for different
segments of the functions. They are also useful tool in a wide variety of
fields, including computer science, mathematics, and physics. It seems through
some initial research, that wavelets can be applied in many of the same
situations as Fourier series, and it interests us to know of this alternate
options from the Fourier series.

\section{Links}%
\label{sec:links}

Here are some relevant links from primary research:

\textbf{Note} A specific portion of wavelet theory needs to be determined, as
the entire topic is far too broad. But we need to do some more research to be
able to determine the topic of wavelets, that we are able to comprehend and
interpret best.
\begin{itemize}
  \item \url{https://www.eecis.udel.edu/~amer/CISC651/IEEEwavelet.pdf}
  \item \url{http://www.hpl.hp.com/hpjournal/94dec/dec94a6.pdf}
  \item \url{http://web.iitd.ac.in/~sumeet/WaveletTutorial.pdf}
\end{itemize}
\end{document}
