\documentclass[12pt]{amsart}

\usepackage{amsfonts}
\usepackage{amsthm}
\usepackage{amsmath}
\usepackage{amscd}
\usepackage{epstopdf}
\usepackage{csquotes}
\usepackage{enumitem}
\usepackage{mathrsfs}
\usepackage{pgfplots}

\usepackage[draft]{todonotes}

\newcommand{\R}{\mathbb{R}}
\newcommand{\Z}{\mathbb{Z}}
\newcommand{\LP}{L^2(\mathbb{R})}
\newcommand{\norm}[1]{\left\lVert#1\right\rVert}
\newcommand{\abs}[1]{\left\lvert#1\right\rvert}

\newtheorem{thm}{Theorem}[section]
\newtheorem{corollary}{Corollary}[thm]
\newtheorem{lemma}[thm]{Lemma}

\title{Wavelets}
\author{Grant Mohn \and Arden Rasmussen}
\date{\today}

\begin{document}
\pagenumbering{gobble}
\maketitle
\pagenumbering{arabic}

\section{Introduction}%
\label{sec:introduction}

Wavelets are a method for the representation of any function in
$\LP$. Similarly to how the Fourier series is used to represent
the same functions utilizing trigonometric functions, Wavelets are the same,
except instead of utilizing trigonometric functions, it is posible to use any
number of wavelets.

\section{Haar Wavelets}%
\label{sec:haar_wavelets}

\todo{Are there an infinite number of wavelets?}
It is posible to create a basis for $\LP$ using many different
wavelets.  However, for the scope of this paper, we will only focus on the
wavelet basis formed by the wavelets of the Haar function. The Haar function is
defined by

\begin{align}
  \Psi &= \begin{cases} 
    1  & 0\leq x < \frac{1}{2} \\
    -1 & \frac{1}{2} \leq x < 1 \\
    0  & \text{otherwise}
  \end{cases}\label{eq:haar}
\end{align}

\begin{figure}[htpb]
\begin{center}
\begin{tikzpicture}[scale=1, transform shape]
  \begin{axis}[
    axis lines = middle,
    xlabel = $x$,
    ylabel = {$f(x)$}]
  \addplot[domain=-.5:0, samples=100, line width=2pt]{0};
  \addplot[domain=0:.5, samples=100, line width=4pt]{1};
  \addplot[domain=.5:1, samples=100, line width=4pt]{-1};
  \addplot[domain=1:1.5, samples=100, line width=2pt]{0};
  \end{axis}
\end{tikzpicture}
\end{center}
\caption{Haar Function}
\label{fig:haar_function}
\end{figure}

A proposed basis for $\LP$ is

\begin{align}
  \left\{2^{\frac{j}{2}}\Psi\left(2^jx-k\right)\right\}_{j\in\Z,\ k\in\Z}
\end{align}

We can show that this basis is an orthonormal basis, by first proving that this
is an orthogonal set of functions, and that all the functions are unit length
functions.

\subsection{Orthogonality}%
\label{sub:orthogonality}

\subsection{Unit}%
\label{sub:unit}

To show that these functions are unit ``length'' functions, we must show that

\begin{align}
  \norm{2^\frac{j}{2}\Psi\left(2^jx-k\right)}_{L^2} &= 1
\end{align}

In order to show this we can show that

\begin{align}
  {\left<2^\frac{j}{2}\Psi\left(2^jx-k\right),\ 2^\frac{j}{2}\Psi\left(2^jx-k\right)\right>}_{L^2} &= 1
\end{align}

We know that inner product in $\LP$ is the integral of the product of the
functions, we can write this as

\begin{align}
  {\left<2^\frac{j}{2}\Psi\left(2^jx-k\right),\
  2^\frac{j}{2}\Psi\left(2^jx-k\right)\right>}_{L^2} &=
  \int_{-\infty}^{\infty} \abs{2^\frac{j}{2}\Psi\left(2^jx-k\right)}^2dx
\end{align}

Now we solve the integral for any arbitrary $j$ and $k$ in $\Z$.

\begin{align}
  &\int_{-\infty}^{\infty} \abs{2^\frac{j}{2}\Psi\left(2^jx-k\right)}^2dx\\
  =\abs{2^j} &\int_{-\infty}^{\infty} \abs{\Psi\left(2^jx-k\right)}dx
\end{align}

Since the Haar Function \eqref{eq:haar} is $0$ everywhere except for when $0\leq
x < 1$, then the only part of the integral that will not be zero is when the
values passed to the Haar function are between $0$ and $1$. We can calculate
these occurrences for this situation.

Here is the process of computing the value of X that will cause the input to
the Haar function to be zero.
\begin{align}
  2^jx-k=0\\
  x=\frac{k}{2^j}
\end{align}

Using this process for values of $0$, $\frac{1}{2}$, and $1$, we find that

\begin{align}
  0\quad&x=\frac{k}{2^j}\\
  \frac{1}{2}\quad&x=\frac{1}{2\cdot2^j}+\frac{k}{2^j}\\
  1\quad&x=\frac{1}{2^j}+\frac{k}{2^j}
\end{align}

\end{document}
