\documentclass[10pt]{armath}
\usepackage{amsmath}
\usepackage{csquotes}
\usepackage{enumitem}
\usepackage{multicol}

\usepackage[margin=1in]{geometry}

\title{Art 208 Review}
\author{Arden Rasmussen}
\date{\today}

\begin{document}
\maketitle

\begin{multicols}{2}
  \section{Introduction}%
  \label{sec:introduction}

  What is art history? A practice of empathy. Placing objects, images, and
  architecture in contexts:

  \begin{itemize}
    \item material/physical
    \item social/historical
    \item theoretical/philosophical
  \end{itemize}

  Why should we study ancient art? Problems of style, relative and absolute
  chronology, periodization, geography, and diversity of Mediterranean world.

  \section{Agean Bronze Age. 3000-1000BCE}%
  \label{sec:agean_bronze_age_3000_1000bce}
  
  Late Helladic art and island of Agina. Mycenean Culture. Linear B and written
  language. Critical problems: When is the "Age of Heroes" Homer is talking
  about? Homer's \textit{Iliad} and its relationship to archaeological Troy.
  What can myth and art tell us about history? Do they preserve certain
  historical memories? H. Schliemann. Homer's "catalog of ships" lists real
  Mycenean cities. Warrior-Aristocratic culture: time and kleos, citadels. Gold
  trade, Gift exchange. Ca. 1200 BCE: attackers from the sea? Internal
  collapse?

  \textbf{
  Religion and cultural artifacts. Material culture: how do we evaluate it as
  evidence. Burial practices: inhumation or cremation? What is in the tomb and
  why?
  } 

  The problem of seeing what you want to see. Modern perspectives shape ancient
  objects. Agean "Bronze" Age. Ca. 3000-1000BCE. Sir Author Evans. Greeks: not
  the first "great" civilization and not the first people in the Aegean. Troy
  was the major maritime urban center.

  Cycladic islands. Folded arm figurines. Havoc of looting and forgery for
  understanding the ancient world. Polychromy.

  Minoan "Palace Culture". Large administrative, ceremonial, economic
  complexes. How to interpret "mute" evidence. 1628BCE Major "climatic event"
  Knossos, 1700-1400BCE. Conservation vs. "restoration". Problem of myth and
  history. Akrotiri. Frescoes!

  \textbf{Art as "historical document" or "don't be textist". "Descriptive
  Naturalism" vs. "Descriptive Chauvinism". The "uses" of classicism.} 

  \section{Geometric Period 1100-700BCE}%
  \label{sec:geometric_period_1100_700bce}
  
  "Dark Ages" and "Greek Renaissance". Iron. Critical role of the Near East.
  Rise of the polis: cemeteries \& sancturaries.

  Athenian Pottery. Protogeometric 1050-900BCE. Early Geometric period
  900-850BCE. Middle Geometric 850-760BCE. Late Geometric 760-690BCE. Human
  figures. Art as an expression of social status in a new kind of aristocracy
  city-state. What does "community" mean in this culture? A preoccupation wiht
  order, courage in battle, individual excellence admired by others. Can a pot
  reflect the "world-view" of a society? Dipylon cemetery.

  The principle of reciprocity at the heart of ancient Greek religion :
  Mantiklos Apollo.

  \textbf{Body as metaphor. Monuments and memory. "...memory as a resource,
  precious as gold". Patterns are good for remembering. Kinship or community?
We impose our world on the past and so did the ancients. Life sometimes
imitates art. How do we think about "ornament"?} 

\section{The "Orientalizing" Period 710-600BCE}%
\label{sec:the_orientalizing_period_710_600bce}

A word about "periodization". Evolution of the polis: oligarchs and tyrants.
Noble birth not the only factor, now wealth matters. Sanctuaries rather than
elaborate family burials. Family farm and hoplite phalanx remakes the social
order: polis. Individuality is not as vital as unity integrity. Tensions
between individual an community. Greek iconography coalesces: recognizable
visual stories. Greeks migrate and make colonies.

Dedalic style. Egyptian technology reignites monumental Greek sculpture. Sema.
Statues as "signs".

\textbf{Pictures can produce myths not just depict them. In Greece, "masculine
body the base state". Complexity of visual evidence.} 

\section{The Archaic Period 600-520BCE}%
\label{sec:the_archaic_period_600_520bce}

Political centralization = increasing monumental stone architecture. Smaller
shrines replaced by monumental temples, which becomes symbols of communal
authority. Egyptian influence. The Temple: house of the God, not the
worshipper. Religion = politics and economics. Syracuse, Corfu, Paestum, and
Delphi. Importance of number and order. Harmony and proportion. Presocratic
philosophy: especially Pythagoras. Temple decoration. Perhaps Greeks use
mythological figures to signify the sacredness of a space that must be
protected.

\textbf{Why should we care about triglyphs, motopes, or volutes? Powerful
metaphors of architecture. Synergy of Greek architecture and architectural
sculpture. The challenges of adapting sculpture narrative to the pediment. How
do you fit a visual story in a triangle?} 

Ideology and art. "Panhellenic" sanctuaries and treasuries: spaces beyond the
individual polis. Place to come together for all Greeks. The role of
competition in Greek culture. Competition is a form of worship. Olympia.
Delphi. Oracle at the hart of religion and politics. Treasuries like small
temples but erected by city-states to house offerings and statues contributed
by its wealthy citizens. Siphnian Treasury 525BCE. Ionic frieze. Caryatids.
North frieze: new conceptions of pictorial space. Gods vs. Giants visualizing
epic form of combat in contrast with more communal hoplite infantry. Class
conciousness.

\textbf{Using one story to tell another. Mythology and ideology/politics.} 

\section{Archaic Attic Pottery 600-520BCE}%
\label{sec:archaic_attic_pottery_600_520bce}

What did Greek painting look like> Painters and potters. Symposion ritual. New
ways of thinking about space: porthole scenes. Greek preoccupation with
moderation and restraint, fame and shape. Great artists: Kleitias, Kexekias,
and Amasis painter. Moments of quiet, before the storm.

\textbf{What do "gender" and "sexuality" look like in ancient Greece? Male
bonding rituals: ritualized drinking, lyric poetry, intellectual conversation,
sexual activity. Power defined by sexual access/dominance or
activity/passivity. How doe artists give visual shape to social status?
Figure/ground.} 

\section{Freestanding Sculpture in the Archaic Period 600-520BCE}%
\label{sec:freestanding_sculpture_in_the_archaic_period_600_520bce}

The return of monumental sculpture: kouroi \& korai. A "prinicple of
interchangeability". Line rahter than volume dominates sculptural aesthetic.
Symmetry and geometry. Aristocratic associations with kouroi/korai. Private
display in public sacred spaces. If you are an ancient Greek, be young! Case
study from Kerodotus.

\textbf{Body as metaphor. Images as role models. The problem of nakedness: why
are men depicted nude and women depicted always with clothes> Spartan
exceptions. Polychomy in ancient sculpture: what "should" it look like?
Iconographical flexibility and style. Reading facial expressions: the archaic
smile, an artistic convention. Why is there rapid stylistic development and
change over a relatively short period in Greek art?} 

Sappho. Temple of Aphaia at Aegina. "In 480 the Greeks became Greek." Persian
wars. Mythology as a form of visualizing politics and identity. Herodotus's
"ethnography." Trojans get cast as Persians.

Late Archaic Attic potter 530BCE: Red-figure allows light figures on a dark
background. Foreground/background. Visual paradoxes. Neer: social context and
physical context interact. Penchant for paradox. Pioneer group coincides with
democratic reforms in Athens and the expansion of citizenship. Role-playing in
self-representation. Sexuality and gender defined by power. Images of "erotica"
or "sexual slavery"?

\textbf{How do we define "art"> How is that different from how the Greeks
defined "art"? How doe we separate the "art" from the content/subject
matter/what it is that is being represented or depicted?} 

\section{Early Classical 480-440BCE}%
\label{sec:early_classical_480_440bce}

Persian Wars and Greek Sculptures: Greek vs. "Barbarian". Rare examples of wall
painting. Neer: "integration" defines Early Classical. Greek art and Greek
drama: analogies with Greek Chorus> Greek preoccupation with moderation. But
also new emotion and psychology is present. Temple of Zeus at Olympia:
integration of sculpture and architecture; sequential narrative; psychological
depth; naturalism. Privileged view of the spectator.

Bronze free-standing sculpture. Problem of Roman "copies" of Greek sculpture.
Delphi Charioteer. State-sponsored statue groups engage the spectator.
Artemision God. Argive Style: Riace warriors. Canon of Polykleitos. Ratio,
number, symmetry, proportion, "rhythmos". Engaging the spectator in dramatic
ways. Multiple viewpoints. Narrative group sculpture.

\section{High Classical 440-400BCE}%
\label{sec:high_classical_440_400bce}

Periclean Athens: imperialist democracy. Parthenon: Is it a temple, a victory
monument, a "sacred strongbox"? Optical refinements. Combination of orders.
Sculpture of the Parthenon: fusing past and present> Architecture of the body.
Symmetry effect. Pyropylaia. Temples of Athena Polias and Thena Nike.
Panathenaic procession.

\textbf{How to read the combination of Doric and Ionic? What is Neer trying to
say about the complexities of imagery that evokes both "aristocracy" and
"democracy"? What should art look like at a time of "Golden Age" or crisis and
social disintegration?} 

Perciclean Athens. Pyropylaia. Mixture of orders again. The effect of Symmetry
is what counts. Temples of Athena Polias: focus on local cults. Built in
compressed space on an unlevel site. A "classical ground zero". Temple of
Athena Nike. A frieze with specific historical events? A snapshot approach to
narrative> Violence and visual representation? "Classical Greeks could not
resist thinking of military domination in sexualized terms".

\textbf{Changing representations the female body.} 

\end{multicols}

\end{document}
