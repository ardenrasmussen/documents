\documentclass[10pt]{armath}
\usepackage{amsmath}
\usepackage{csquotes}
\usepackage{enumitem}
\usepackage{multicol}

\usepackage[margin=1in]{geometry}

\title{Art 208 Review}
\author{Arden Rasmussen}
\date{\today}

\begin{document}
\maketitle

\section{Rome Intro}%
\label{sec:rome_intro}

Two problems of origins. Foundations myths of Rome. Two traditions: 1. Trojan:
Aeneas. 2. Italian: Romulus and Remus. Anxiety about fratricide. 753 BCE
Romulus founds Rome. The Capitoline She World. Kings of Rome. 509 BCE revolt
against Tarquinius Superbus.

An architecture of cities. Roman concrete technology. Axiality and symmetry.
Facade mentality. Temples. Arches and Barrel vaults. Romans freed from the
constraints of post and lintel architecture.

\section{Pompeii}%
\label{sec:pompeii}

Republican period. Mount Vesuvius 79 CE. Civilization before Romans = Samnites.
Pompeii becomes a Roman colony in 80 BCE. Forum. Romans love axiality and
symmetry. Temples of Jupiter and Apollo. Basilica. Amphitheater, Baths.
Concrete makes new things possible. Architecture is the ultimate social
construction. Social hierarchies and architectural spaces. Bathing and
spectacle are essential in forming community and identity.

\section{Pompeii II}%
\label{sec:pompeii_ii}

Pompeii. Domestic architecture. Prototype for Roman house is Etruscan tomb.
What makes a space public or private> Working at home was a Roman thing. Social
hierarchies and architectural spaces. Clients and patrons. ``Helenized''
Domus. Axiality. Progression of space and light; interest in panorama and vista.
Suburban villas.

Roman Wall painting: four styles. First style/Masonry style: imitates marble
panels. Second Style: Wall becomes a window. Villa of the Mysteries. Roman
religion: mystery cults. Initiation rites for Bacchus.

\section{Republican Portraiture}%
\label{sec:republican_portraiture}

Romans as appropriation artists. Horace: ``Captured Greece and conquered its
savage Roman conqueror and brought the arts into rustic Latium.'' Cato the Elder
was none too pleased. Republican period. Republic, a democracy of sorts,
characterized by checks and balances: term limits for office holders, power
sharing, collegiality and libertas. Particians. ``Verism'' in Roman
portraiture: real or ideal? \textbf{There are no ideal bodies, just changing
ideals.} Architecture becomes a powerful tool for self-fashioning. Julius
Caesar, Pompey, and civil war.

\section{Early Empire I (27 BC - 96 BCE)}%
\label{sec:early_empire_i}

Augustus: Rome the superpower from republic to empire. Focus on the individual.
Head of church and state, but calls himself ``first citizen.'' Pax Augusta:
peace doesn't mean peace for everybody. Augustus shrewdly uses visual language
to shape his public image. He makes Rome a city of marble in the model of the
Athenian Acropolis. New source of marble available at Carrara. Augustan
classicism. Youth is Back! Brags he built or restored some 82 temples.

\section{Early Empire II}%
\label{sec:early_empire II}

Visualizing peace and prosperity. Connecting with Pereiclean Athens.
Monumentalizes treaties with Spain and Gaul. Part of a larger architectural
complex. Children and Family values. Roman narrative: the abundant use of
historical/documentary subjects. The concious mixing of the mythology and
history.

Second and Third style Roman wall painting. Sacro-idyllic landscapes, delicate
linear fantasies, monochromatic backgrounds.

\section{Early Empire III}%
\label{sec:early_empire_iii}

Empire in the provinces. Public works. Roman infrastructure. Pont de Gard
brings water from 30 miles away, providing water to the people. The power of
divie kingship.

Empire = the problem of succession. Julio-Claudians: Tiberius; Caligula. New
recipe for concrete; Claudius: Porta Maggiore: Aqueduct and Triumphal Arch. Did
Claudius have a taste for an archaic form of rustic masonry?
Nero. A monster how contributed greatly to the history of architecture. Domus
Aurea: A villa in the heart of Rome. Designed by Severus and Celer. Octagonal
plan with a dome. Breaking out of the rectilinear prison of Greek architecture;
light plays a key role; shaping the interior space; first use of groin vaults.

Fourth style of Roman wall painting, intricate style: seems to combine elements
of all the first three. Views through the wall return but are impossible.
Pictures within pictures. Illusionistc fragments of architecture in an
irrational space. Fabulous Fabullus self-consciousness about the act of
representation and the plurality of style.

\section{Early Empire IV}%
\label{sec:early_empire_iv}

After Nero's death there is civil war. Vespasian wins out. Portraiture: a
rejection of Julio-Claudian idealism. Vespasian as anti-nero. The problem of
succession again: now emperors will come from the military.

Gladiatorial games have origins in funerary rites. The Flavian Amphitheater.
Importance of location. What would you give in return for you freedom? The
complexity of Roman spectacle. Why an oval? How does one feel oneself to be a
Roman citizen in the Roman Empire: by not being a barbarian. Architecture as
social construction. Architectural engineering and social engineering.
Institutionalized violence and social control. Re-enactment of Greek tragedies
with real people. What can we learn from the Colosseum? Perhaps there is
something more important than groin vaults? How do we turn people into images?

\section{Early to High Empire}%
\label{sec:early_to_high_empire}

\section{High Empire II}%
\label{sec:high_empire_ii}

\section{High Empire III}%
\label{sec:high_empire_iii}

\section{High Empire IV}%
\label{sec:high_empire_iv}

\section{High Empire V}%
\label{sec:high_empire_v}

\section{Late Empire I}%
\label{sec:late_empire_i}

\section{Late Empire II}%
\label{sec:late_empire_ii}

\section{Late Empire III}%
\label{sec:late_empire_iii}



\end{document}
