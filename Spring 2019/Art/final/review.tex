\documentclass[10pt]{armath}
\usepackage{amsmath}
\usepackage{csquotes}
\usepackage{enumitem}
\usepackage{multicol}

\usepackage[margin=1in]{geometry}

\title{Art 208 Review}
\author{Arden Rasmussen}
\date{\today}

\begin{document}
\maketitle

\begin{multicols}{2}
\section{Dates}%
\label{sec:dates}

\begin{description}
  \item[Etruscan] 1000-50BCE
  \item[Republican Period] 509-27BCE
  \item[Early Roman Empire] 27BCE-96CE
    \begin{itemize}
      \item Augustus
      \item Tiberius
      \item Caligula
      \item Claudius
      \item Nero
      \item Vespasian
      \item Titus
      \item Domitian
    \end{itemize}
  \item[High Roman Empire] 96-192CE
    \begin{itemize}
      \item Trajan
      \item Hadrian
      \item Antonius Pius
      \item Marcus Aurelius
    \end{itemize}
  \item[Late Empire] 192-337CE
    \begin{itemize}
      \item The Severans
      \item The Soldier Emperors
      \item Diocletian and the Tetrarchy
      \item Constantine
    \end{itemize}
\end{description}

\section{Rome Intro}%
\label{sec:rome_intro}

Two problems of origins. Foundations myths of Rome. Two traditions: 1. Trojan:
Aeneas. 2. Italian: Romulus and Remus. Anxiety about fratricide. 753 BCE
Romulus founds Rome. The Capitoline She World. Kings of Rome. 509 BCE revolt
against Tarquinius Superbus.

An architecture of cities. Roman concrete technology. Axiality and symmetry.
Facade mentality. Temples. Arches and Barrel vaults. Romans freed from the
constraints of post and lintel architecture.

\section{Pompeii}%
\label{sec:pompeii}

Republican period. Mount Vesuvius 79 CE. Civilization before Romans = Samnites.
Pompeii becomes a Roman colony in 80 BCE. Forum. Romans love axiality and
symmetry. Temples of Jupiter and Apollo. Basilica. Amphitheater, Baths.
Concrete makes new things possible. Architecture is the ultimate social
construction. Social hierarchies and architectural spaces. Bathing and
spectacle are essential in forming community and identity.

\section{Pompeii II}%
\label{sec:pompeii_ii}

Pompeii. Domestic architecture. Prototype for Roman house is Etruscan tomb.
What makes a space public or private> Working at home was a Roman thing. Social
hierarchies and architectural spaces. Clients and patrons. ``Helenized''
Domus. Axiality. Progression of space and light; interest in panorama and vista.
Suburban villas.

Roman Wall painting: four styles. First style/Masonry style: imitates marble
panels. Second Style: Wall becomes a window. Villa of the Mysteries. Roman
religion: mystery cults. Initiation rites for Bacchus.

\section{Republican Portraiture}%
\label{sec:republican_portraiture}

Romans as appropriation artists. Horace: ``Captured Greece and conquered its
savage Roman conqueror and brought the arts into rustic Latium.'' Cato the Elder
was none too pleased. Republican period. Republic, a democracy of sorts,
characterized by checks and balances: term limits for office holders, power
sharing, collegiality and libertas. Particians. ``Verism'' in Roman
portraiture: real or ideal? \textbf{There are no ideal bodies, just changing
ideals.} Architecture becomes a powerful tool for self-fashioning. Julius
Caesar, Pompey, and civil war.

\section{Early Empire I (27 BC - 96 BCE)}%
\label{sec:early_empire_i}

Augustus: Rome the superpower from republic to empire. Focus on the individual.
Head of church and state, but calls himself ``first citizen.'' Pax Augusta:
peace doesn't mean peace for everybody. Augustus shrewdly uses visual language
to shape his public image. He makes Rome a city of marble in the model of the
Athenian Acropolis. New source of marble available at Carrara. Augustan
classicism. Youth is Back! Brags he built or restored some 82 temples.

\section{Early Empire II}%
\label{sec:early_empire II}

Visualizing peace and prosperity. Connecting with Pereiclean Athens.
Monumentalizes treaties with Spain and Gaul. Part of a larger architectural
complex. Children and Family values. Roman narrative: the abundant use of
historical/documentary subjects. The concious mixing of the mythology and
history.

Second and Third style Roman wall painting. Sacro-idyllic landscapes, delicate
linear fantasies, monochromatic backgrounds.

\section{Early Empire III}%
\label{sec:early_empire_iii}

Empire in the provinces. Public works. Roman infrastructure. Pont de Gard
brings water from 30 miles away, providing water to the people. The power of
divine kingship.

Empire = the problem of succession. Julio-Claudians: Tiberius; Caligula. New
recipe for concrete; Claudius: Porta Maggiore: Aqueduct and Triumphal Arch. Did
Claudius have a taste for an archaic form of rustic masonry?
Nero. A monster how contributed greatly to the history of architecture. Domus
Aurea: A villa in the heart of Rome. Designed by Severus and Celer. Octagonal
plan with a dome. Breaking out of the rectilinear prison of Greek architecture;
light plays a key role; shaping the interior space; first use of groin vaults.

Fourth style of Roman wall painting, intricate style: seems to combine elements
of all the first three. Views through the wall return but are impossible.
Pictures within pictures. Illusionistc fragments of architecture in an
irrational space. Fabulous Fabullus self-consciousness about the act of
representation and the plurality of style.

\section{Early Empire IV}%
\label{sec:early_empire_iv}

After Nero's death there is civil war. Vespasian wins out. Portraiture: a
rejection of Julio-Claudian idealism. Vespasian as anti-nero. The problem of
succession again: now emperors will come from the military.

Gladiatorial games have origins in funerary rites. The Flavian Amphitheater.
Importance of location. What would you give in return for you freedom? The
complexity of Roman spectacle. Why an oval? How does one feel oneself to be a
Roman citizen in the Roman Empire: by not being a barbarian. Architecture as
social construction. Architectural engineering and social engineering.
Institutionalized violence and social control. Re-enactment of Greek tragedies
with real people. What can we learn from the Colosseum? Perhaps there is
something more important than groin vaults? How do we turn people into images?

\section{Early to High Empire}%
\label{sec:early_to_high_empire}

Fourth-style Roman wall painting: pictures within pictures; fragmentary
architectural view; eclectic combinations of elements form the first three
styles. Mythological pictures, as if panels inserted into walls. House of the
Vetii: great example of fourth style.

Portraits: self-fashioning. Images of literacy. Mythological painting: is it
based on Greek prototypes? Still life painting: convert with dynamics of light,
composition, observed description.

Trajan. The ``best'' emperor and first from the provinces. Succession by
adoption: Nerva adopts Trajan in 97CE. Expansion of empire to its furthest
extremes. Public architecture. Form with Basilica Ulpia, Column and Markets of
Trajan. Apollodorus of Damascus: an artist and a military engineer. Column of
Trajan as a ``sculptural document.'' Multiple rather than single perspective.
Visual narrative and scrolling. The role of repetition in shaping public memory
and the perception of history. Roman military values visualized. How and why do
we depict our enemies? Roman vs. Decian. We can learn about the Romans by how
they depict the enemy. How did they see barbarians? The body as metaphor:
decapitation imagery.

\section{High Empire II}%
\label{sec:high_empire_ii}

Trajan continued .Column of Trajan and the body as metaphor. Livy's parable of
Menenius. Images of supplication and submission.

Form with Markets of Trajan. More public works, bit ones. 150 tabernae(shops):
origins of the shopping mall. Concrete with brick masonry and travertine
decoration. High point of vaulting technology. Broken pediments; brick faced
concrete. Multicultural Roman society: Trajan; Apollodorus of Damascus: Greek
engineer-architect from Syria.

Arch of Trajan: images of the emperor at work. Emperor as paterfamilias. The
way of the ancestors becomes a public necessity.

\section{High Empire III}%
\label{sec:high_empire_iii}

Hadrian: adopted by Trajan. Emperor, architect, art lover, intellectual, world
traveler, called Greekling. The pantheon: temple to all the gods: traditional
and revolutionary. Probably under construction at the time of Trajan> An
architecture of revelation. Facade mentality combined with a  masterpiece of
shaped interior space. New recipe for concrete. Celestial Geometry:
intersection of horizontal and vertical circles. Metaphoric of light. The
problem with signatures. Roman state religion: pax deorum. What should
revelation/the divine/the transcendent look like?

Hadrian's Temple of Venus and Roma: very Greek. Primary source: Dio Cassius

Villa at Tivoli: turning the empire into an art museum. Hadrian's obsessions:
pumpkin domes and Antinuous.

\section{High Empire IV}%
\label{sec:high_empire_iv}

Hadrian: A few more words about Pantheon. Centrally planned buildings. How does
geometry shape a social space? Celestial Geometry: intersection of horizontal
and vertical circles.

Hadrian's Temple of Venus and Roma: very Greek. Primary source: Dio Cassius

Villa at Tivoli: turning the empire into an art museum. Island Villa. Large
baths. Canopus. Sarpeum (after Egyptian god Serapis). Re-writing Greek
antiquity. Hadrian's obsessions: pumpkin domes and Antinuous.

\section{High Empire V}%
\label{sec:high_empire_v}

The Antonines. Antoninus Pius. Pedestal of his column: conflation of styles,
conflation of time. Antonie plague.

Marcus Aurelius. An emperor philosopher who hated war but spent 8 winter
campaigns in Marcomannic wars. His diary survives. Private ideas written in
Greek for himself. Stoicism: a personal philosophy of how to live. Courage,
wisdom, self-control, justice. Life is not infinite. How to be a good person
and treat people well. How to do politics in a constructive way. How to retreat
into your own mind. Art of living is more like wrestling than dancing. We are
part of a bigger chain of being. Column of Marcus Aurelius: an even darker
side of war.

The germ of the Late Antique Style. Emphasis on frontally. Anti-naturalism?
Anti-Classicism? The need to assert power and legitimacy when little is
available? The co-existence of styles, even in the same monument. Style and
social class? Style sends a message.

\section{Late Empire I}%
\label{sec:late_empire_i}

New conceptions of death and funerary practices begets sarcophagi; a greater
concern with the afterlife and personal virtue. Work as identity. What is the
visual equivalent of the apostrophe, an address to the view? vrontal vs
provile. Are you an intrinsically good or bad person? Is it your actions that
make you so>

Roman Egypt and Fayum portraits: a synergy of cultures; an artist's perspective.
Ancient Roman art is multicultural and diverse.

The Severans. Septimius Severus. Move towards major military autocracy. Roman
Africa, diversity in Roman Empire. Developments in the triumphal arch. Baroque
Roman architecture: Petra in Jordan. Roman North Africa: Timgad and Lepcis
Magna. Clarity and rigid frontally in Imperial art. Caracalla. Extended
citizenship to all subjects of the Roman empire. Baths: biggest is best.
Caracalla as Hercules. Romans looking at Greeks again: Glykon of Athens copy of
Lysippos.

\section{Late Empire II}%
\label{sec:late_empire_ii}

Roman art in the Near East. Jordan: The baroque trend in Roman architecture.
Khazneh: pure facadism. Relations with Roman painting. Inspires 17th-century
Baroque architecture.

Eastern edge of the Roman world: Duraeuropos: a microcosm of religious
diversity. The shift from polytheism to monotheism. Isis cult. Mithras cult:
triumph of light over darkness, good over evil. Why Christianity> Focus on
afterlife, interior piety, allows slaves and women. Jewish art isn't all
an-iconic. Palymyra, oasis city in Syria: a meeting of Eastern and Western
Styles. Roman + Parthian forms. Palmyrene funerary reliefs in our museum. Why
do people destroy works of art?

The Soldier Emperors. How does a superpower fail? 20 emperors + pretenders
after Severans. Third century crisis: things fall apart. Militarization of
society. Empire is too big. Problem of succession gets pretty extreme; army
decides who becomes emperor. Class struggle: generals vs. elite senators.
Inflation. Invasions by resurgent Parthians and Sassanids. Armies staffed by
so-called barbarian tribes. Population decline. City vs Rural tension. Aurelian
Walls: concrete with re-used bricks. No significant architecture except for
Walls. portraits. Do they represent insecurity> Perhaps individuality? Late
antique style in sarcophagi.

Diocletian and the Tetrarchy. A return to order. Diocletian rescues the Roman
empire with effective government. Divides the empire East/West and North/South.
What does unity and stability look like? Abstraction and uniformity. Image of
emperor becomes sacred, untouchable. Diocletian's palace. Return to the
fortified palace. Octagonal mausoleum as part of hist private palace. 293 CE
Tetrarchy: 4 emperors.

\section{Late Empire III}%
\label{sec:late_empire_iii}

293 Tetrarchy. A return to abstraction and geometry. Constatine. Ascends to
power from another civil war. 312 CE defeats Maxentius at Milvian Bridge;
attributes victory to Christian God. 324 establishes himself as the sole Roman
emperor. Founds Constantinople where Rome continues in the East.

Early Christian Art. Edict of Milan 313 CE. Constantine begins the process of
decriminalizing and then favoring Christianity that will lead to the
establishment of it as the official state religion. A religion that privileges
interiority and the after life. Re-purposing pagan iconography. By 390 CE
Empire is 90\% Christian.

What is the best style with which to depict absolute power>
Abstraction/anti-naturalism? Understanding the use of spoglia on the Arch of
Constantine. Constantine as a bridge to the Middle Ages. Art history shows the
imbrication, hybridization, and appropriation that characterizes the visual
arts.
\end{multicols}

\end{document}
