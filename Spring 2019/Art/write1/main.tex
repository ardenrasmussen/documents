\documentclass[12pt]{amsart}
\usepackage{amsmath}
\usepackage{csquotes}
\usepackage{enumitem}
\usepackage{hanging}
% \usepackage[margin=1in]{geometry}

\title{Reasearch Discussion}
\author{Arden Rasmussen}
\date{\today}

\begin{document}
\maketitle

\par\noindent\rule{\textwidth}{0.4pt}

One resource that I learned about was \textit{Grove Art Online}. This is an
art encyclopedia that was put online. It seems to have detailed descriptions of
individual works, collections, and artists. It would appear to be a great
source for finding more information about specific people or artworks, but
would be less useful for general research on a subject. It also provides a
bibliography on each entry, so that I am able to find new sources about the
same subject. I think that this will be a great resource once I already have an
introductory level of knowledge in the subject, and can be used to achieve a
deeper understanding on whatever is being researched.

Currently I have two main ideas. The first is how restoration and
reconstruction is done. More specifically on the techniques that are used to
determine what the original composition was, and how that should be represented
to public audiences. I found the different representations of the pediments
very interesting as to how different museums chose to represent the original
works. And how this might alter the intended meaning of the artwork.

The other idea is to research more on how the Greeks would use their mythology
as a general subject of an artistic work, while also focusing on a more
specific moment in their history. Such as using myths to symbolize the war with
the Persians. I want to understand the effect that this played on the
interpretation of the work, and how it might have altered the Greek perception
of their enemies.

\par\noindent\rule{\textwidth}{0.4pt}
Words: 258
\par\noindent\rule{\textwidth}{0.4pt}

\end{document}
