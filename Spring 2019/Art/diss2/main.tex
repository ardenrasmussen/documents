\documentclass[12pt]{amsart}
\usepackage{amsmath}
\usepackage{csquotes}
\usepackage{enumitem}
\usepackage{hanging}
\usepackage[margin=1.5in]{geometry}

\title{Discussion 2}
\author{Arden Rasmussen}
\date{\today}

\begin{document}
\maketitle

\par\noindent\rule{\textwidth}{0.4pt}

\begin{hangparas}{.25in}{1}
  Connely, Joan. "Parthenon and Parthenoi: A Mythological Interpretation of the
  Parthenon Frieze" \textit{American Journal of Archeaology} Vol. 100, No. 1
  (1996): 53-80
\end{hangparas}

\par\noindent\rule{\textwidth}{0.4pt}

This article began with a short introduction to the frieze of the Parthenon,
and how as it stands we do not possess a full understanding of what the frieze
is intended to represent. Scholars are distributed over many interpretations,
and it fail's to fall into the expected pattern of some mythological
interpretation that all other Greek friezes have.

This article proposes a possible mythological interpretation, that relates the
frieze to the story of Erechthesus's victory over Eumolpos. A short
introduction to this myth, informs us that Erechthesus was forced to sacrifice
one of his daughters in order to save Athens as a whole. The article goes
through many segments of the frieze and demonstrates how the scenes represented
could be placed into this proposed story.

The article also comments on the story of Pandora found within the Parthenon,
and how that could also be placed within the proposed mythological story.

If this proposal were to be correct, then this would require scholars to
rethink their current understanding of this era of Greek culture, and this
would be a large break from what is currently understood.

This argument seems to be very plausible, as all of the arguments make clear
sense. However, I found that I lacked the ability to critically read the
article, as without significant amount of background information, I was unable
to form my own options, and thus the views of the article are imposed upon
myself. This is done with little to no consideration for counter arguments. I
would like to know what alternative proposals would be, so that it would be
possible to compare.

\textbf{What are alternative competing interpretations of the frieze, and what
are the counter-arguments to this proposal?} 

\par\noindent\rule{\textwidth}{0.4pt}

Words: 268

\par\noindent\rule{\textwidth}{0.4pt}

\end{document}
