\documentclass[10pt]{amsart}
\usepackage{amsmath}
\usepackage{csquotes}
\usepackage{enumitem}
\usepackage{multicol}
\usepackage[margin=1in]{geometry}

\title{Theoretical Dynamics Review}
\author{Arden Rasmussen}
\date{\today}

\newcommand{\vc}[1]{\overrightarrow{#1}}


\begin{document}
\maketitle

% Constrained Lagrangian (Shape/Tension)
% Coupled Oscilator

\begin{multicols}{2}

  \section{Rockets}%
  \label{sec:rockets}

  Variable mass
  \begin{align*}
    m\frac{d\vec{v}}{dt}=\vc{F_{ext}}+\underbrace{\vc{U_{e,r}}\frac{dm}{dt}}_{T=\text{thrust}}\\
  \end{align*}

  For rocket fired upwards $\vc{F_{ext}}=-mg$.

  \subsection{Tsiolkovsky's Rocket Equations}%
  \label{sub:tsiolkovsky_s_rocket_equations}

  \begin{align*}
    V_f-V_i=U_{e,r}\ln\left(\frac{m_i}{m_f}\right)
  \end{align*}

  \subsection{Center of Mass}%
  \label{sub:center_of_mass}

  \begin{align*}
    \vc{R}\equiv\frac{\sum m_k\vc{r}_k}{M_{tot}}
  \end{align*}

  \begin{align*}
    K=\underbrace{\frac{1}{2}M_{tot}V^2_{cm}}_\text{translating}
    +\underbrace{\frac{1}{2}I_{cm}\omega^2}_\text{rotating}
  \end{align*}

  \section{Rolling down incline}%
  \label{sec:rolling_down_incline}

  Conserve of E
  \begin{align*}
    Mgh=\frac{1}{2}MV^2_{cm}+\frac{1}{2}I_{cm}\omega^2\quad I_{cm}=fMR^2
  \end{align*}

  No rotational $V_{cm}=\sqrt{2gh}$
  With rotational $V_{cm}=\sqrt{\frac{2gh}{1+f}}$

  \subsection{Rolling Without Slipping}%
  \label{sub:rolling_without_slipping}

  \begin{align*}
    V_{cm}=\omega r
  \end{align*}

  \section{Collisions}%
  \label{sec:collisions}

  \subsection{Inelastic}%
  \label{sub:inelastic}

  $p_{tot}$ conserved, $E_{mech}$ \textbf{not} conserved. Use total mass for
  the final mas, as they stick together.

  \subsection{Elastic}%
  \label{sub:elastic}

  $p_{tot}$ and $E_{mech}$ are both conserved. Things collide and don't stick.
  $p_{1f}\perp p_{2f}$. Use $p_{ix}=p_{fx}$, and same with $y$, then use the
  relation found there in the expression for $E$.

  Objects end on trajectories perpendicular to one another.

  \section{Central Force Motion}%
  \label{sec:central_force_motion}

  Angular momentum $\vc{L}$ conserved, use polar coordinates, motion in plane
  $\perp\vc{L}$.

  \begin{align*}
    \vc{p}&=m\vc{v}\\
    \vc{L}&=I\omega=fmr^2\dot{\theta}
  \end{align*}

  \begin{align*}
    \vc{N}=\frac{d\vc{L}}{t}
  \end{align*}

  Torque is time rate of change of angular momentum.

  \begin{align*}
    V_{eff}(r)=V(r)+\frac{L^2}{2mr^2}
  \end{align*}

  \begin{itemize}
    \item Circular
    \item Elliptical
    \item Parabolic
    \item Hyperbolic
  \end{itemize}

  \begin{align*}
    E(r)=\frac{1}{2}m\dot{r}^2+V_{eff}(r)
  \end{align*}

  \begin{align*}
    T=\frac{2\pi}{\dot{\theta}}\quad
    f=\frac{\dot{\theta}}{2\pi}
  \end{align*}

  \subsection{Small Oscillations}%
  \label{sub:small_oscillations}

  $a$ is the radius of circular motion. Find this by taking the first
  derivative and setting it equal t zero.

  \begin{align*}
    k=\frac{d^2V}{dx^2}\Bigr|_{x_min}&\implies
    k_r=\frac{d^2V_{eff}}{dr^2}\Bigr|_{a}\\
                                     &\qquad\omega_r=\sqrt{\frac{k_r}{m}}
  \end{align*}

  \section{Hamiltonian}%
  \label{sec:hamiltonian}
  
  \begin{align*}
    H=\sum_{k=1}p_k\dot{q}_k-L
  \end{align*}

  \begin{align*}
    p_k\equiv\frac{\partial L}{\partial \dot{q}_k}
  \end{align*}

  \begin{align*}
    \frac{\partial H}{\partial p_k}&=\dot{q}_k\\
    \frac{\partial H}{\partial q_k}&=-\dot{p_k}+Q_k\\
    \frac{\partial H}{\partial t}&=-\frac{\partial L}{\partial t}
  \end{align*}

  $\frac{d H}{dt}=$ if $Q_k=0$, which means all forces are derivable form a
  potential, and $\frac{\partial H}{\partial t}=0$, which means that $H$ is not
  explicit function of time (does not have $t$ directly in it).

  \section{Lagrangians}%
  \label{sec:lagrangians}

  \begin{align*}
     L=K-V
  \end{align*}
  
  \begin{align*}
    \frac{d}{dt}\frac{\partial L}{\partial \dot{x}}-\frac{\partial L}{\partial
    x}=0
  \end{align*}

  \section{Coupled Oscillators}%
  \label{sec:coupled_oscillators}
  
  \begin{align*}
    m\ddot{x_1}&=-k(x_1-l)+k_{12}(x_2-x_1-l_{12})\\
    m\ddot{x_2}&=-k_{12}(x_2-x_1-l_{12})-k(x_2-(l+l_12))
  \end{align*}

  \begin{align*}
    u_1\equiv x_1-l\quad u_2\equiv x_2-(L+l_{12})
  \end{align*}

  \begin{align*}
    m\ddot{u_1}&=-ku_1+k_{12}(u_2-u_1)\\
    m\ddot{u_2}&=-ku_2-k_{12}(u_2-u_1)
  \end{align*}

  Since there is no dampening, we guess
  \begin{align*}
    u_1&=\beta_1\cos(\omega t+\theta)\\
    u_2&=\beta_2\cos(\omega t+\theta)
  \end{align*}
  If there was dampening, one would guess with complex numbers. We assume that
  $\omega$, and $\theta$ are the same for both masses.

  Now plug this into the previous equations, and construct system of equations.
  For this it will be
  \begin{align*}
    \underbrace{
     \begin{pmatrix}
       -m\omega^2+k+k_{12} & -k_{12}\\
       -k_{12} & -m\omega^2+k+k_{12}
     \end{pmatrix}
     }_A
     \begin{pmatrix}
        \beta_1 \\ \beta_2
     \end{pmatrix}=
     \begin{pmatrix}
        0 \\ 0
     \end{pmatrix}
  \end{align*}
  To have non-trivial solutions, we need $\det(A)=0$. So take the determinant,
  and set it equal to zero. Use that to solve for $\omega$, you will probably
  need to use quadratic formula, but also okay to cheat a bit.

\end{multicols}


\end{document}
