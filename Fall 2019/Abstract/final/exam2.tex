\documentclass[10pt]{amsart}
\usepackage{amsmath}
\usepackage{amsthm}
\usepackage{amssymb}
\usepackage{enumitem}
\usepackage{todonotes}
\usepackage{mathrsfs}
\usepackage[makeroom]{cancel}

\title{Abstract Algebra --- Final Exam}
\author{Arden Rasmussen}

\newcommand{\Z}{\mathbb{Z}}
\newcommand{\R}{\mathbb{R}}
\newcommand{\N}{\mathbb{N}}
\newcommand{\C}{\mathbb{C}}
\newcommand{\Q}{\mathbb{Q}}
\newcommand{\ra}{\rightarrow}
\newcommand{\lra}{\longleftrightarrow}
\newcommand{\E}{\exists}
\newcommand{\zw}{\mathbb{Z}\left[\omega\right]}
\newcommand{\zi}{\mathbb{Z}\left[2i\right]}
\newcommand{\rad}[1]{\text{rad}\left(#1\right)}
\renewcommand{\ker}[1]{\text{Ker}\left(#1\right)}
\newcommand{\im}[1]{\text{Im}\left(#1\right)}
\renewcommand{\gcd}[2]{\text{GCD}\left(#1,#2\right)}
\renewcommand{\mod}[2]{#1\left(\text{mod}\ #2\right)}

\renewcommand{\thesubsection}{(\alph{subsection})}

\newenvironment{claim}[1]{\par\noindent\textit{Claim:}\space#1}{}
\newtheorem{lemma}{Lemma}

\begin{document}
\maketitle

\section{Problem}%
\label{sec:problem_1}

\underline{\textit{Basic Concepts:}} For each of the following concepts of
abstract algebra provide a definition \underline{and} and example. When
instructed provide a counterexample.

\subsection{Group}%
\label{sub:group}

A group is a set of elements $G$ and an operation $\star$, expressed as
$(G,\star)$, where
\begin{itemize}
  \item  $\star$ is associative.
  \item $\star$ has an identity in $G$, that is to say there is some $1\in G$
    such that $1\star a=a\star 1=a\forall a\in G$.
  \item Every element in $G$ has an inverse with respect to $\star$, we can
    write this as $\forall a\in G\exists a^{-1}\in G$ such that $a\star
    a^{-1}=a^{-1}\star a=1$.
  \item Optionally $\star$ may be commutative.
\end{itemize}

$(\{0\},+)$ is a group.

\subsection{Ring}%
\label{sub:ring}

A ring is a set of elements $R$, and a $+$ and $\cdot$ operators, expressed as
$(R,+,\cdot)$, and
\begin{itemize}
  \item $(R,+)$ is a commutative group.
  \item $\cdot$ is associative.
  \item $\cdot$ distributes over $+$.
  \item $\cdot$ may or may not be commutative.
  \item $\cdot$ may or may not have an identity.
  \item Elements of $R$ may or may not have an inverse with respect to
    $\cdot$.
\end{itemize}

$\C$ is a ring.

\subsection{Integral domain}%
\label{sub:integral_domain}

An integral domain, is a ring $(R,+,\cdot)$ where there are no zero divisors,
where $a\neq0\in R$ is called a zero divisor if there is some $b\neq0\in R$
such that $a\cdot b=0$.

$\Q$ is an integral domain, $\C'$ is not an integral domain

\subsection{Euclidean domain}%
\label{sub:euclidean_domain}

A commutative ring with identity $(R,+,\cdot)$ that is an Integral domain, is
called an Euclidean Domain if there exists a function
\begin{align*}
  N:R\ra\N\cup\{0\}
\end{align*}
with respect to which $R$ has division algorithm.

$\R[X]$ is a euclidean domain, $\R[X,Y]$ is not a euclidean domain.

\subsection{PID}%
\label{sub:pid}

Integral Domains in which every ideal is principal are called principal ideal
domains (PID)

$\Z$ is a PID, $\R[X,Y]$ is not a PID\@.

\subsection{UFD}%
\label{sub:ufd}

If $(R,+,\cdot)$ is a commutative integral domain with identity, in which every
non-zero, non-unit
\begin{itemize}
  \item has a factorization in terms of irreducible
  \item that factorization is unique up to permutations and associates
\end{itemize}
then $(R,+,\cdot)$ is a Unique Factorization Domain (UFD)

$\R[X]$ is a UFD, $\Z[i\sqrt{5}]$ is not a UFD\@.

\subsection{Homomorphism}%
\label{sub:homomorphism}

A function $F:R\ra S$ between two rings with identity is said to be a
homomorphism if
\begin{itemize}
  \item $F(r_1+r_2)=F(r_1)+F(r_2)$.
  \item $F(r_1\cdot r_2)=F(r_1)\cdot F(r_2)$.
  \item $F(1)=1$.
\end{itemize}

$F:\R[X]\ra\R$ given by $F:P(X)\ra P(1)$ is a homomorphism. The function
$F:\Z[i]\ra\Z[2i]$ given by $F(a+bi)=a+2bi$ is not a homomorphism.

\subsection{Kernel and Image}%
\label{sub:kernel_and_image}

Given a homomorphism $F:R\ra S$ then kernel and image are defined as the below.
\begin{align*}
  \ker{F}&=\left\{r\in R\vert F(r)=0\right\}\subseteq R\\
  \im{F}&=\left\{s\in S\vert \exists r\in R, F(r)=s\right\}\subseteq S
\end{align*}

The kernel an image of the homomorphism $F:\R[X]\ra\R$ given by $F:P(X)\ra
P(1)$ are
\begin{align*}
  \ker{F}=(1-X)\quad\im{F}=\R.
\end{align*}

\subsection{Isomorphism}%
\label{sub:isomorphism}

A homomorphism $F:R\ra S$ is called an isomorphism if $F$ is a bijection.

$F:\Z[\sqrt{2}]\ra\Z[\sqrt{2}]$ given by $F(a+b\sqrt{2})=a-b\sqrt{2}$ is an
isomorphism.

\subsection{Ideal}%
\label{sub:ideal}

In $(R,+,\cdot)$ commutative integral domain with identity, $I\subseteq R$ is an
ideal if
\begin{itemize}
  \item $i_1,i_2\in I\ra i_1+i_2\in I$
  \item $r\in R,i\in I\ra ri\in I$
\end{itemize}

$(5)$ over $\Z$ is an ideal.

\subsection{Prime ideal}%
\label{sub:prime_ideal}

$P$ ideal is called prime if $ab\inP\ra a\in P\ \text{or}\ b\in P$.

$(3)$ over $\Z$, is a prime ideal. $(6)$ over $\Z$, is not a prime ideal, as
$3\cdot 2\in(6)$ but $3\notin(6)$ and $2\notin(6)$.

\subsection{Maximal ideal}%
\label{sub:maximal_ideal}

An ideal is called maximal if it is not contained in any proper ideal.
\begin{align}
  I\subseteq \xcancel{J}\subseteq R
\end{align}

$(X,Y)$ is maximal over $\R[X,Y]$.
$(X)$ is not maximal over $\R[X,Y]$ as $(X)\subseteq(X,Y)$.

\subsection{Quotient ring}%
\label{sub:quotient_ring}

Let $R$ be a commutative ring with identity and an integral domain. Let $I$ be
an ideal of $R$. We define
\begin{itemize}
  \item the relation $\mod{\equiv}{I}$ by $\mod{a\equiv b}{I}$ if and only if
    $a-b\in I$.
  \item the set $R/I=\left\{[a]\vert a\in R\right\}$.
  \item the operations $+,\cdot$ on $R/I$
    \begin{align*}
      [a]+[b]=[a+b]\ \text{and}\ [a]\cdot[b]=[a\cdot b].
    \end{align*}
\end{itemize}
Then $R/I$ is a quotient ring.

$\Z/(3)$ is a quotient ring.

\subsection{Field}%
\label{sub:field}

A field, is a ring $(R,+,\cdot)$ where every element of $R$ has a
multiplicative inverse.

$\Q$ is a field, but $\Z$ is not.

\subsection{Algebra over a field}%
\label{sub:algebra_over_a_field}

An algebra is a ring which also happens to be a vector space over some field of
scalars.

$\R_{m\times n}$ is an algebra, over the field of $\R$ which act as the
scalars.

\subsection{Field extension}%
\label{sub:field_extension}

Field extensions are given two fields $A\subseteq B$, $B$ is an extension of
$A$ if they share the same operations?
\todo[inline]{CHECK THIS}

$\Q(\sqrt{2})$ is an extension of $\Q$.

\section{Problem}%
\label{sec:problem_2}

\underline{\textit{Classic constructions of Abstract Algebra --- Universal
    Properties:}} Recall the construction of the field of quotients of a
commutative integral domain $R$, Let $\equiv$ denote the equivalence relation
on $R\times(R\setminus\{0\})$ given by
\begin{align*}
  (a,b)\equiv(c,d)\lra\exists x,y\in R\setminus\{0\},(ax,bx)=(cy,dy).
\end{align*}
By the field of quotients of $R$ we mean the set $\mathcal{Q}(R)$a of
equivalence classes
\begin{align*}
  \mathcal{Q}(X)=\{[(a,b)]\vert a,b\in R,b\neq 0\}
\end{align*}
of $\equiv$ together with operations
\begin{align*}
  [(a,b)]+[(c,d)]=[(ad+bc,bd)]\ \text{and}\ [(a,b)]\cdot[(c,d)]=[(ac,bd)].
\end{align*}
On homework you proved that $\mathcal{Q}(R)$ indeed is a field. In this problem
I as you to prove the following theorem about the field of quotients.

\subsection{}%
\label{sub:2a}

Show that there is an injective homomorphism $i:R\ra\mathcal{Q}(R)$.

\begin{claim}
  $i:R\ra\mathcal{Q}(R)$ defined by $i(r)=[(r,1)]$ is an injective
  homomorphism.
\end{claim}
\begin{proof}
  To show that $i$ is a homomorphism, we first show that it presevers addition,
  and multiplication. Consider some $a,b\in R$, then
  \begin{align*}
    i(a+b)=[(a+b,1)]=[(a,1)]+[(b,1)]=i(a)+i(b).
  \end{align*}
  Thus $i$ preserves addition. Now we consider
  \begin{align*}
    i(a\cdot b)=[(a\cdot b,1)]=[(a,1)]\cdot[(b,1)]=i(a)\cdot i(b)
  \end{align*}
  Thus $i$ preserves multiplication. Now we verify that $i(1)=1$.
  \begin{align*}
    i(1)=[(1,1)]=1_{\mathcal{Q}(R)}.
  \end{align*}
  Thus $i$ also preserves identity, and so we can conclude that it is indeed a
  homomorphism.

  Now we will show that it is injective. Assume that it is not injective, that
  is to say there exists some $a,b\in R$ with $i(a)=i(b)$. We express this as
  \begin{align*}
    [(a,1)]=[(b,1)].
  \end{align*}
  From the definition of $\equiv$ this means that there exists some $x,y\in
  R\setminus\{0\}$ such that
  \begin{align*}
    (ax,x)=(by,y).
  \end{align*}
  From this it is clear that $x=y$, and so subsequently we find $a=b$, but this
  is a contradiction of our assumption. Thus we can conclude that $i$ is
  injective.
\end{proof}

\subsection{}%
\label{sub:2b}

Suppose $\Gamma:R\ra F$ is another injective homomorphism of $R$ into a field
$F$. Show that
\begin{align*}
  \gamma:\mathcal{Q}(R)\ra F\ \text{defined by}\
  \gamma([a,b])=\Gamma(a)\cdot {\Gamma(b)}^{-1}
\end{align*}
is a (well-defined) homomorphism.

\begin{claim}
  $\gamma:\mathcal{Q}(R)\ra F$ defined by
  $\gamma([(a,b)])=\Gamma(a)\cdot{\Gamma(b)}^{-1}$ is a well-defined
  homomorphism.
\end{claim}
\begin{proof}
  First we show that $\gamma$ is a homomorphism. To show this, we demonstrate
  the preservation of addition, multiplication, and unit. Consider some
  $[(a,b)],[(c,d)]\in\mathcal{Q}(R)$.
  \begin{align*}
    \gamma([(a,b)]+[(c,d)])=\gamma([(ad+bc,bd)])=\Gamma(ad+bc)\cdot{\Gamma(bd)}^{-1}
  \end{align*}
  Then since $\Gamma$ is a homomorphism, we can rewrite this to be
  \begin{align*}
    \Gamma(ad){\Gamma(bd)}^{-1}+\Gamma(bc){\Gamma(bd)}^{-1}&=
    \Gamma(a){\Gamma(b)}^{-1}\Gamma(d){\Gamma(d)}^{-1} +
    \Gamma(b){\Gamma(b)}^{-1}\Gamma(c){\Gamma(d)}^{-1}\\
    &=\Gamma(a){\Gamma(b)}^{-1}+\Gamma(c){\Gamma(d)}^{-1}\\
    &=\gamma([a,b])+\gamma([c,d])
  \end{align*}
  Then to demonstrate the preservation of multiplication, consider
  \begin{align*}
    \gamma([(a,b)]\cdot[(c,d)])=\gamma([(ac,bd)])=\Gamma(ac){\Gamma(bd)}^{-1}
  \end{align*}
  Again by the homomorphic nature of $\Gamma$ we can rewrite this to be
  \begin{align*}
    \Gamma(a){\Gamma(b)}^{-1}\Gamma(c){\Gamma(d)}^{-1}=\gamma([(a,b)])\cdot\gamma([(c,d)]).
  \end{align*}
  Finally we show that $1$ is preserved.
  \begin{align*}
    \gamma([(1,1)])=\Gamma(1){\Gamma(1)}^{-1}=1
  \end{align*}
  Thus it is clear that $\gamma$ is a homomorphism.

  Now we prove that $\gamma$ is well defined. Consider $[(a,b)]=[(c,d)]$, with
  some $x,y\in R\setminus\{0\}$ such that $(ax,bx)=(cy,dy)$. We notice that
  \begin{align*}
    \gamma([(a,b)])=\gamma([(ax,bx)])\ \text{and}\
    \gamma([(c,d)])=\gamma([(cy,dy)]).
  \end{align*}
  Now we compute
  \begin{align*}
    \gamma([(a,b)])=\gamma([(ax,bx)])&=\Gamma(ax){\Gamma(bx)}^{-1}\\
    &=\Gamma(cy){\Gamma(dy)}^{-1}=\gamma([(cy,dy)])=\gamma([(c,d)])
  \end{align*}
  And thus $\gamma$ is well defined.
\end{proof}

\subsection{}%
\label{sub:2c}

Show that $\gamma\circ i=\Gamma$.

\begin{proof}
  Consider our definition of $\gamma$ and $i$, then we consider
  \begin{align*}
    \gamma\circ i = \gamma(i(r))=\gamma([(r,1)])=\Gamma(r){\Gamma(1)}^{-1}
  \end{align*}
  Since $\Gamma$ is a homomorphism, then $\Gamma(1)=1$ and
  ${\Gamma(1)}^{-1}=1^{-1}=1$. So we find
  \begin{align*}
    \Gamma(r){\Gamma(1)}^{-1}=\Gamma(r).
  \end{align*}
  Thus it becomes clear that $\gamma\circ i=\Gamma$.
\end{proof}

\section{Problem}%
\label{sec:problem_3}

\underline{\textit{Application ot Number Theory:}} Let $p$ be a prime number
and consider the field $\Z/(p)$ of integers modulo $p$.

\subsection{}%
\label{sub:3a}

Show that for all $[k]\neq[0]$ the mapping
\begin{align*}
  x\mapsto[k]\cdot x
\end{align*}
is a bijection from the set of nonzero element of $\Z/(p)$ to itself.
Alternativly, argue that
\begin{align*}
  [k],[2k],[3k],\ldots,[(p-1)k]
\end{align*}
is a permutation of $[1],[2],[3],\ldots,[(p-1)]$.

\todo[inline]{First we say $k<p$ and $x<p$, then $kx$ by udf will be some
  $p_1p_2\cdots p_n q_1q_2\cdots q_m$, but with $q_i,p_i<p$ and thus $p$ is not
  in the ufd?}

\subsection{}%
\label{sub:3b}

Argue that for all $[k]\neq [0]$ we have ${[k]}^{p-1}=[1]$.
\todo[inline]{we consider this to be $k\cdot k\cdots k$ $p-1$ times...}

\subsection{}%
\label{sub:3c}

Factorize the polynomial $X^{p-1}-1$ over $\Z/(p)$.

\subsection{}%
\label{sub:3d}

Based on the above prove the following two classic theorems of number theory:
\begin{align*}
  \gcd{k}{p}=q\ra k^{p-1}\equiv \mod{q}{p}\ \text{and}\ (p-1)!\equiv
  \mod{-1}{p}.
\end{align*}

\subsection{}%
\label{sub:3e}

Now let $F$ denote any finite field and let $|F|$ denote the number of elements
of $F$ generalize the above to prove
\begin{align*}
  \alpha^{|F|-1}=1
\end{align*}
for all non-zero $\alpha\in F$. What, if anything, can you say about the
product of all non-zero elements of $F$?


\section{Problem}%
\label{sec:problem_4}

\underline{\textit{Advanced Topic:}} Recall the following
\begin{itemize}
  \item For an ideal $I$ of the polynomail ring $\C[X_1,X_2,\ldots,X_n]$ we
    define
    \begin{align*}
      \rad{I}=\left\{P\in\C[X_1,X_2,\ldots,X_n]|\exists k\in\N,p^k\in I\right\}
    \end{align*}
    Here $\N$ denotes the set of positive integers. Recall that $\rad{I}$ was
    on the first midterm exam.

  \item An ideal $I$ of $\C[X_1,X_2,\ldots,X_n]$ is said to be
    \underline{radical} if $\rad{I}=I$.

  \item For an ideal $I$ in the polynomial ring $\C[X_1,X_2,\ldots,X_n]$ we
    define
    \begin{align*}
      \mathscr{Z}(I)=\left\{\alpha\in\C^n\vert\forall P\in
        I,P(\alpha)=0\right\}.
    \end{align*}

  \item Subsets $\mathbf{X}\subset \C^n$ of the form $\mathscr{Z}(I)$ are called
    \textit{algebraic sets}.

  \item For an algebraic set $\mathbf{X}$ we define
    \begin{align*}
      \mathscr{J}(\mathbf{X})=\left\{P\in\C[X_1,X_2,\ldots,X_n]\vert\forall
        \alpha\in\mathbf{X},P(\alpha)=0\right\}.
    \end{align*}

  \item The Strong Nullstellensatz (due to David Hilbert) states that
    \begin{align*}
      \mathscr{J}(\mathscr{Z}(I))=\rad{I}
    \end{align*}
    for all ideals $I$ of $\C[X_1,X_2,\ldots,X_n]$.
\end{itemize}
In this problem I ask you to prove the following.

\subsection{}%
\label{sub:4a}

Prove that for all algebraic sets $\mathbf{X}$ the set
$\mathscr{J}(\mathbf{X})$ is
\begin{itemize}
  \item An ideal of $\C[X_1,X_2,\ldots,X_n]$.
  \item A radical ideal of $\C[X_1,X_2,\ldots,X_n]$.
\end{itemize}

\subsection{}%
\label{sub:4b}

Prove, through element chasing, that $\mathscr{Z}(I)=\mathscr{Z}(\rad{i})$ for
all ideals $I$ of $\C[X_1,X_2,\ldots,X_n]$.

\subsection{}%
\label{sub:4c}

Prove, through element chasing, that $\mathscr{Z}(\mathscr{J}(\mathbf{X}))$ for
all algebraic sets $\mathbf{X}\subset\C^n$.

\subsection{}%
\label{sub:4d}

Prove that $\mathscr{Z}$ is a bijection between the set of radical ideals of
$\C[X_1,X_2,\ldots,X_n]$ and the set of algebraic sets in $\C^n$.



\end{document}
