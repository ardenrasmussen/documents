\documentclass[10pt]{armath}
\usepackage{amsmath}
\usepackage{amsthm}
\usepackage{amssymb}
\usepackage{csquotes}
\usepackage{enumitem}
% \usepackage[margin=1in]{geometry}


\title{Abstract Algebra - First Midterm Exam}
\author{Arden Rasmussen}
\date{\today}

\newcommand{\Z}{\mathbb{Z}}
\newcommand{\R}{\mathbb{R}}
\newcommand{\N}{\mathbb{N}}
\newcommand{\C}{\mathbb{C}}
\newcommand{\zw}{\mathbb{Z}\left[\omega\right]}
\newcommand{\zi}{\mathbb{Z}\left[2i\right]}
\newcommand{\rad}[1]{\text{rad}\left(#1\right)}


\begin{document}
\maketitle

\section*{Problem 1}%
\label{sec:problem_1}

Let $\omega\in\C$ be a solution of the equation
\begin{align*}
   \omega^2+\omega+1=0.
\end{align*}
Consider the set $\zw=\left\{a+b\omega\vert a,b\in\Z \right\}$. Show that the
set $\zw$ is closed under the ordinary addition and under the ordinary
multiplication. Conclude that $\zw$ is a ring which is a subring of the field of
complex numbers.

\begin{proof}
  Consider the set $\zw=\left\{a+b\omega\vert a,b\in\Z \right\}$. Let
  $a_1+b_1\omega,a_2+b_2\omega\in\zw$, then we compute
  \begin{align*}
    \left(a_1+b_1\omega\right)+\left(a_2+b_2\omega\right)=(a_1+a_2)+(b_1+b_2)\omega.
  \end{align*}
  It is clear that $a_1+a_2\in\Z$ and $b_1+b_2\in\Z$, so we can conclude that
  $\zw$ is closed under ordinary addition. Now we compute
  \begin{align*}
    \left(a_1+b_1\omega\right)\cdot\left(a_2+b_2\omega\right)=a_1a_2+a_1b_2\omega+a_2b_1\omega+b_1b_2\omega^2.
  \end{align*}
  Since $\omega^2+\omega+1=0$, then we know that $\omega^2=-\omega-1$, so we
  can rewrite this to be
  \begin{align*}
    a_1a_2+a_1b_2\omega&+a_2b_1\omega-b_1b_2(\omega+1)\\
                       &=
                       \left(a_1a_2-b_1b_2\right)+\left(a_1b_2+a_2b_1-b_1b_2\right)\omega.
  \end{align*}
  Again, we can see that $a_1a_2-b_1b_2\in\Z$ and $a_1b_2+a_2b_1-b_1b_2\in\Z$,
  thus we conclude that $\zw$ is closed under ordinary addition.

  TODO

  We are able to conclude that $\left(\zw,+,\cdot\right)$ is a ring, and is a
  subring of $\C$.
\end{proof}

\section*{Problem 2}%
\label{sec:problem_2}

Consider the set $\zi = \left\{a+2bi\vert a,b\in\Z\right\}$. Standard number
addition and multiplication turn $\zi$ into a commutative integral domain with
identity.

\subsection*{(a)}%
\label{sub:_a_}

Prove that $2$ is irreducible in this ring.

\subsection*{(b)}%
\label{sub:_b_}

Prove that $2i$ is irreducible in this ring.


\subsection*{(c)}%
\label{sub:_c_}

Is it true that $2|2i$ in this ring?

\subsection*{(d)}%
\label{sub:_d_}

Are $2$ and $2i$ associates in this ring?

\subsection*{(e)}%
\label{sub:_e_}

Can you provide two factorizations of $4$ into irreducible?

\subsection*{(f)}%
\label{sub:_f_}

Is $2$ prime in this ring? Justify your claim.

\subsection*{(g)}%
\label{sub:_g_}

Is $2i$ prime in this ring?

\subsection*{(g)}%
\label{sub:_g_}

Is $\zi$ a Euclidean domain? Is it a PID?

\section*{Problem 3}%
\label{sec:problem_3}

Let $I$ be an ideal of a commutative ring $R$ with identity. Define the
following set:
\begin{align*}
  \rad{I}=\left\{r\in R|r^n\in I\ \text{for some}\ n\in\N\right\}.
\end{align*}
Note: $\N$ is the set of positive integers only. In particular, $0 \notin \N$.

\subsection*{(a)}%
\label{sub:_a3_}

Suppose temporarily that $R=\Z$. Find $\rad{I}$ for the following choices of
$I$:

\subsubsection*{(i)}%
\label{ssub:_i_}

$I=(9)$

\subsubsection*{(i)}%
\label{ssub:_i_}

$I=(43)$

\subsubsection*{(iii)}%
\label{ssub:_i_}

$I=(72)$

\subsection*{(b)}%
\label{sub:_b3_}

Going back to the general situation, show $\rad{I}$ is an ideal. Hint: Look at
your very first homework assignment.
\end{document}
