\documentclass[10pt]{armath}
\usepackage{amsmath}
\usepackage{amsthm}
\usepackage{amssymb}
\usepackage{csquotes}
\usepackage{enumitem}
% \usepackage[margin=1in]{geometry}


\title{Abstract Algebra --- First Midterm Exam}
\author{Arden Rasmussen}
\date{\today}

\newcommand{\Z}{\mathbb{Z}}
\newcommand{\R}{\mathbb{R}}
\newcommand{\N}{\mathbb{N}}
\newcommand{\C}{\mathbb{C}}
\newcommand{\Q}{\mathbb{Q}}
\newcommand{\zw}{\mathbb{Z}\left[\omega\right]}
\newcommand{\zi}{\mathbb{Z}\left[2i\right]}
\newcommand{\rad}[1]{\text{rad}\left(#1\right)}


\begin{document}
\maketitle

\section*{Problem 1}%
\label{sec:problem_1}

Let $\omega\in\C$ be a solution of the equation
\begin{align*}
  \omega^2+\omega+1=0.
\end{align*}
Consider the set $\zw=\left\{a+b\omega\vert a,b\in\Z \right\}$. Show that the
set $\zw$ is closed under the ordinary addition and under the ordinary
multiplication. Conclude that $\zw$ is a ring which is a subring of the field of
complex numbers.

\begin{proof}
  Consider the set $\zw=\left\{a+b\omega\vert a,b\in\Z \right\}$. Let
  $a_1+b_1\omega,a_2+b_2\omega\in\zw$, then we compute
  \begin{align*}
    \left(a_1+b_1\omega\right)+\left(a_2+b_2\omega\right)=(a_1+a_2)+(b_1+b_2)\omega.
  \end{align*}
  It is clear that $a_1+a_2\in\Z$ and $b_1+b_2\in\Z$, so we can conclude that
  $\zw$ is closed under ordinary addition. Now we compute
  \begin{align*}
    \left(a_1+b_1\omega\right)\cdot\left(a_2+b_2\omega\right)=a_1a_2+a_1b_2\omega+a_2b_1\omega+b_1b_2\omega^2.
  \end{align*}
  Since $\omega^2+\omega+1=0$, then we know that $\omega^2=-\omega-1$, so we
  can rewrite this to be
  \begin{align*}
    a_1a_2+a_1b_2\omega&+a_2b_1\omega-b_1b_2(\omega+1)\\
    &=
    \left(a_1a_2-b_1b_2\right)+\left(a_1b_2+a_2b_1-b_1b_2\right)\omega.
  \end{align*}
  Again, we can see that $a_1a_2-b_1b_2\in\Z$ and $a_1b_2+a_2b_1-b_1b_2\in\Z$,
  thus we conclude that $\zw$ is closed under ordinary addition.

  Since $\zw\subset \C$ and it is closed under addition and multiplication,
  then we are able to conclude that $\left(\zw,+,\cdot\right)$ is a ring, and
  is a subring of $\C$.
\end{proof}

\section*{Problem 2}%
\label{sec:problem_2}

Consider the set $\zi = \left\{a+2bi\vert a,b\in\Z\right\}$. Standard number
addition and multiplication turn $\zi$ into a commutative integral domain with
identity.

\subsection*{(a)}%
\label{sub:_a_}

Prove that $2$ is irreducible in this ring.

\begin{proof}
  Assmume that $2$ is reducible, that is that there exists some $a,b\in\zi$
  such that $2=ab$, where $a,b$ are non-unit and non-zero. Then we compute
  \begin{align*}
    2&=ab\\
    N(2)&=N(a)N(b)\\
    4&=N(a)N(b).
  \end{align*}
  Since $a,b$ are not unit, then $N(a),N(b)\neq 1$, so that means
  \begin{align*}
    N(a)=N(b)=2.
  \end{align*}
  If $a=\alpha+2\beta$, with $\alpha,\beta\in\Z$, then
  $N(a)=\alpha^2+4\beta^2=2$. There can be no $\alpha,\beta$ that satisfy this
  equation. Thus we conclude that $a$ must be unit. However, this is a
  contradiction of our assumption, thus $2$ must be irreducible in $\zi$.
\end{proof}

\subsection*{(b)}%
\label{sub:_b_}

Prove that $2i$ is irreducible in this ring.

\begin{proof}
  Assmume that $2i$ is reducible, that is that there exists some $a,b\in\zi$
  such that $2i=ab$, where $a,b$ are non-unit and non-zero. Then we compute
  \begin{align*}
    2i&=ab\\
    N(2i)&=N(a)N(b)\\
    4&=N(a)N(b).
  \end{align*}
  Since $a,b$ are not unit, then $N(a),N(b)\neq 1$, so that means
  \begin{align*}
    N(a)=N(b)=2.
  \end{align*}
  If $a=\alpha+2\beta$, with $\alpha,\beta\in\Z$, then
  $N(a)=\alpha^2+4\beta^2=2$. There can be no $\alpha,\beta$ that satisfy this
  equation. Thus we conclude that $a$ must be unit. However, this is a
  contradiction of our assumption, thus $2i$ must be irreducible in $\zi$.
\end{proof}

\subsection*{(c)}%
\label{sub:_c_}

Is it true that $2|2i$ in this ring?

\begin{proof}
  Assume $2|2i$, this implies that there exists some $q\in\zi$ such that
  $2i=2\cdot q$. However, the only $q$ that would satisfy this statement would
  be $i$, and $i\notin\zi$. Thus $2\nmid2i$
\end{proof}

\subsection*{(d)}%
\label{sub:_d_}

Are $2$ and $2i$ associates in this ring?

\begin{proof}
  Units in this ring are $\pm 1$. Thus $2$ and $2i$ are not associates, as
  they are not off by a unit of one another.
\end{proof}

\subsection*{(e)}%
\label{sub:_e_}

Can you provide two factorizations of $4$ into irreducible?

\begin{align*}
  4&=2\cdot 2\\
  4&=2i\cdot(-2i)
\end{align*}

\subsection*{(f)}%
\label{sub:_f_}

Is $2$ prime in this ring? Justify your claim.

No $2$ is not prime. Consider $2|4\rightarrow2|2i\cdot-2i$, but $2\nmid2i$ and
$2\nmid -2i$.

\subsection*{(g)}%
\label{sub:_g_}

Is $2i$ prime in this ring?

No $2i$ is not prime. Consider $2i|4\rightarrow2|2\cdot2$, but $2i\nmid 2$.

\subsection*{(g)}%
\label{sub:_g_}

Is $\zi$ a Euclidean domain? Is it a PID\@?

It is neither. It is not a Euclidean domain, because primes $\neq$
irreducibles, and it is not a PID, with a counter example of $(2,2i)$.

\section*{Problem 3}%
\label{sec:problem_3}

Let $I$ be an ideal of a commutative ring $R$ with identity. Define the
following set:
\begin{align*}
  \rad{I}=\left\{r\in R|r^n\in I\ \text{for some}\ n\in\N\right\}.
\end{align*}
Note: $\N$ is the set of positive integers only. In particular, $0 \notin \N$.

\subsection*{(a)}%
\label{sub:_a3_}

Suppose temporarily that $R=\Z$. Find $\rad{I}$ for the following choices of
$I$:

\subsubsection*{(i)}%
\label{ssub:_i_}

$I=(9)$

\begin{align*}
  \rad{I}&=\left\{\pm 3, \pm6, \pm 9, \ldots\right\}\\
  &=\left\{k\cdot 3|k\in\Z\right\}\\
  &=(3)
\end{align*}

\subsubsection*{(i)}%
\label{ssub:_i_}

$I=(43)$

\begin{align*}
  \rad{I}&=\left\{\pm43, \pm86, \pm129, \ldots\right\}\\
  &=\left\{k\cdot 43|k\in\Z\right\}\\
  &=(43)
\end{align*}

\subsubsection*{(iii)}%
\label{ssub:_i_}

$I=(72)$

\begin{align*}
  \rad{I}&=\left\{\pm6, \pm12, \pm18, \ldots\right\}\\
  &=\left\{k\cdot 6|k\in\Z\right\}\\
  &=(6)\\
  &=(2\cdot 3)\\
  &=(2)\cap(3)
\end{align*}

\subsection*{(b)}%
\label{sub:_b3_}

Going back to the general situation, show $\rad{I}$ is an ideal. Hint: Look at
your very first homework assignment.

\begin{proof}
  To prove that $\rad{I}$ is an ideal, we show that $\rad{I}$ is closed under
  addition and multiplication, by some other element in $R$. Consider
  $a,b\in\rad{I}$, thus there exists some $n,m\in\N$ such that
  \begin{align*}
     a^m,b^n\in I.
  \end{align*}
  Consider the expression
  \begin{align*}
    \sum_{k=0}^{k=m+n}{m+n\choose k}a^kb^{m+n-k}.
  \end{align*}
  We consider the general element of this sumation, for some $k$. If $k<m$ then
  we can express this as
  \begin{align*}
    &{m+n\choose k}a^kb^{n+\delta}\qquad\delta=m-k>0\\
    &=\left({m+n\choose k}a^kb^\delta\right) b^n\\
    &=rb^n.
  \end{align*}
  Then by the definition of an ideal, and $b^n\in I$, then $rb^n\in I$. Thus
  any element with $k<m$ is an element of $I$. If $k\geq m$ then we can express
  this as
  \begin{align*}
     &{m+n\choose k}a^{m+\delta}b^{\gamma}\qquad\delta=k-m>0, \gamma=m+n-k>0\\
     &=\left({m+n\choose k}a^\delta b^\gamma\right)a^m\\
     &=ra^m.
  \end{align*}
  Thus again by the definition of an ideal, and $a^m\in I$, then this $ra^m\in
  I$. Thus any element with $k\geq m$ is an element of $I$.

  So we conclude that all elements of this sum is an element of the ideal $I$,
  and since ideals are closed under addition, then
  \begin{align*}
    \sum_{k=0}^{k=m+n}{m+n\choose k}a^kb^{m+n-k}\in I.
  \end{align*}
  By the Binomial theorem, we know that
  \begin{align*}
    \sum_{k=0}^{k=m+n}{m+n\choose k}a^kb^{m+n-k}={\left(a+b\right)}^{m+n}\in I.
  \end{align*}
  and thus we can conclude that $a+b\in\rad{I}$.

  Now we show that $\rad{I}$ is closed under multiplication. Consider
  $a\in\rad{I}$, and $r\in R$. We know that there exists some $n\in\N$ such
  that $a^n\in I$. Then there is some $b\in R$ such that $b=r^n$, and by
  definiton of an ideal, $ba^n\in I$ which implies that $r^na^n\in I$. So we
  conclude that ${\left(ra\right)}^n\in I$, and thus $ra\in\rad{I}$.

  Since $\rad{I}$ is closed under addition and multiplication, we can conclude
  that it is indeed a ring.
\end{proof}

\end{document}
