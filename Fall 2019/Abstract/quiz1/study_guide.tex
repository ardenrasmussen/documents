\documentclass[10pt]{amsart}
\usepackage{amssymb}
\usepackage{amsmath}
\usepackage{csquotes}
\usepackage{enumitem}

\title{Quiz Study Guide}
\author{Arden Rasmussen}
\date{\today}


\begin{document}
\maketitle

\section{Terminology}%
\label{sec:terminology}

\begin{enumerate}
  \item State the commutative and the associative rules for an operation. State
    the distributive rule.
    \begin{description}
      \item[Commutative Rule] \[a\star b=b\star a\]
      \item[Associative Rule] \[( a \star b)\star c =
        a\star(b\star c)\]
      \item[Distributive Rule]
        \begin{align*}
          (a\star b)\circ c&= (a\circ c)\star(b \circ c)\\
          a\circ(b\star c)&=(a\circ b)\star(a\circ c)
        \end{align*}
    \end{description}
  \item What do we mean when we say that\ldots
    \begin{enumerate}[listparindent=0.7cm]
      \item Something is an ``identity'' with respect to an operation? E.g.\ what
        do we mean by an ``additive identity''? Or by a ``multiplicative
        identity''?

        Something is an identity if $a\star \text{Id}=a$, and
        $\text{Id}\star a=a$. That operation applied to the identity and any
        other value is equal to that other value.
      \item An element is ``invertible with respect to an operation''? What do we
        mean by a ``multiplicative inverse''?

        An element is invertible with respect to an operation if there exists
        an ``identity'' for that operation, and there exists a value such that
        $a\star a^{-1}=\text{Id}$, and $a^{-1}\star a=\text{Id}$.
      \item $(G,\star)$ constitutes a group?
        \begin{itemize}
          \item $\star$ is associative.
          \item $\star$ has an identity.
          \item Every element of $G$ has an inverse with respect to $\star$.
          \item $\star$ May or may not be commutative.
        \end{itemize}
      \item $(R,\oplus,\otimes)$ constitutes a ring? A commutative ring? How
        about a field?
        \begin{description}
          \item[Ring] 
            \begin{itemize}
              \item $(R,\oplus)$ is a commutative group.
              \item $\otimes$ is associative.
              \item $\otimes$ distributes over $\oplus$.
              \item $\otimes$ may or may not be commutative.
              \item $\otimes$ may or may not have identity.
              \item Elements of $R$ may or may not have an inverse.
            \end{itemize}
          \item[Field]
            \begin{itemize}
              \item Every element of $R$ has a multiplicative inverse.
            \end{itemize}
        \end{description}
      \item an element is a ``unit'' in a ring?

        Any element which is invertible in a ring.
      \item an element is a ``zero divisor'' in a ring?

        $a\in R,a\neq 0$ is a zero divisor if there is $b\in R,b\neq 0$ such
        that $ab=0$.
      \item something is an ``integral domain''?

        An integral domain is a ring with no zero divisors.
    \end{enumerate}
\end{enumerate}

\section{Examples}%
\label{sec:examples}

\begin{enumerate}
  \item Consider a collection $\mathcal{F}$ of functions
    $f:\mathbb{R}\rightarrow\mathbb{R}$. Composition of functions is an
    operation in the set $\mathcal{F}$.  \[f,g\in\mathcal{F}\rightarrow f\circ
    g\in\mathcal{F}.\]
    \begin{enumerate}[listparindent=0.7cm]
      \item Which rules does this operation satisfy?

        $\circ$ satisfies the Associative rule.
      \item Does it have an identity?

        Yes $\text{Id}(x)=x$.
      \item Does every element of $\mathcal{F}$ have its inverse?

        No, consider $f(x)=e^x$, the inverse would be $f^{-1}(x)=\ln(x)$. But
        $\ln: \mathbb{R} > 0 \mapsto \mathbb{R}$, thus
        $f^{-1}(x)\notin\mathcal{F}$.
      \item Does $\mathcal{F}$ constitute a group? A ring? A field?

        Only one operation is defined, so it is not a ring or a field, and
        since not every element has an inverse, then it is not a group.
    \end{enumerate}
  \item Let $M_n(\mathbb{R})$ be the set of $n\times n$ matrices whose entries
    are real numbers.
    \begin{enumerate}[listparindent=0.7cm]
      \item Which rules do matrix addition and multiplication satisfy?

        Matrix addition satisfies commutative rule, and the Associative rule.
        Matrix multiplication satisfies Associative rule, and the distributive
        rule.
      \item Does matrix addition have an identity? If so, what is it? Likewise
        is there a multiplicative identity? If so, what is it?

        The identity of matrix addition is the zero matrix, the multiplicative
        identity is the identity matrix.
      \item Does every matrix in $M_n(\mathbb{R})$ have its additive inverse?
        For matrices with additive inverse explain how one would go about
        finding the supposed inverse. How about multiplicative inverse? Explain
        how one would go about finding it.

        Yes, every matrix has its additive inverse, to find it, multiply every
        element of the matrix by $-1$. Not all matrices have a multiplicative
        inverse.
      \item Is $\begin{pmatrix}2 & 1 \\ 1 & 2\end{pmatrix}$ a unit in
        $M_2(\mathbb{R})$? What if I changed the framework of the problem and
        considered $M_2(\mathbb{Z})$, the set of $2\times 2$ matrices whose
        entries are integers, in in place of $M_2(\mathbb{R})$?

        Yes this matrix is invertible in $M_2(\mathbb{R})$, so it is a unit. In
        the framework of $M_2(\mathbb{Z})$, then it is not invertible, so it is
        not a unit.
      \item Is $\begin{pmatrix}1&1\\2&0\end{pmatrix}$ a zero divisor in
        $M_2(\mathbb{R})$? How about in $M_2(\mathbb{Z})$? How about
        $\begin{pmatrix}0&1\\0&0\end{pmatrix}$ is it a zero division in
        $M_2(\mathbb{R})$? $M_2(\mathbb{Z})$?

        No this is not a zero divisor, there is no matrix such that when
        multiplied with this one equals the zero matrix. Yes
        $\begin{pmatrix}0&1//0&0\end{pmatrix}$ is a zero divisor in
        $M_2(\mathbb{R})$, and $M_2(\mathbb{Z})$.
      \item Does $M_n(\mathbb{R})$ constitute a group? A ring? A field? An
        integral domain?

        It does not constitute a integral domain, as there are zero divisors,
        and it does not constitute a field, as not all elements have a
        multiplicative inverse, thus it is a field.
    \end{enumerate}
  \item Let $S\neq \emptyset$. Union $\cup$ and intersection $\cap$ are
    operation on the power set $\mathcal{P}(S)$.
    \begin{enumerate}[listparindent=0.7cm]
      \item Which rules do $\cup$ and $\cap$ satisfy?

        $\cup$ satisfies commutative and associative. $\cap$ also satisfies
        commutative and associative. And $\cap$ distributes over $\cup$, so the
        distributive rule is satisfied.
      \item Does $\cup$ have an identity? If so, what is it? Does $\cap$ have
        an identity? If so, what is it?

        The identity for $\cup=\emptyset$, and the identity for $\cap=S$.
      \item Does every $X\in\mathcal{P}(S)$ have an inverse with respect to
        $\cup$? If so, what is it? Does every $X\in\mathcal{P}(S)$ have an
        inverse with respect to $\cap$? If so, what is it?

        No, there is no inverse with respect to $\cup$, there is no
        $Y\in\mathcal{P}(S)$ such that $X\cup Y=\emptyset$, only $X=\emptyset$
        has an inverse. Similarly the only element of $\mathcal{P}(S)$ with an
        inverse with respect to $\cap$ is $S$.
      \item Does $\mathcal{P}(S)$ constitute a group? A ring? A field? An
        integral domain?

        Since $(\mathcal{P}(S), \cup)$ is not a group, because not all elements
        have an inverse. Since this is not a group, then
        $(\mathcal{P}(S),\cup,\cap)$ is not a ring, field, or integral domain.
    \end{enumerate}
  \item Consider the set
    \[\mathbb{Z}\left[\sqrt{2}\right]=\left\{a+b\sqrt{2}|a,b\in\mathbb{Z}\right\}\]
    \begin{enumerate}[listparindent=0.7cm]
      \item Is the regular addition of numbers an operation on
        $\mathbb{Z}\left[\sqrt{2}\right]$? I.e.\ is it true that
        \[x,y\in\mathbb{Z}\left[\sqrt{2}\right]\rightarrow
        x+y\in\mathbb{Z}\left[\sqrt{2}\right]?\] If so, address the situation
        with the identity and the inverses with respect to this operation on
        $\mathbb{Z}\left[\sqrt{2}\right]$.

        Yes it is an operation. The identity is $a=b=0$, or $0+b\sqrt{2}$. Thus
        inverses should exists for all elements. Inverses would be found as
        $x=a+b\sqrt{2}$, then $x^{-1}=(-a)+(-b)\sqrt{2}$.
      \item Is the regular multiplication of numbers an operation on
        $\mathbb{Z}\left[\sqrt{2}\right]$? I.e.\ is it true that
        \[x,y\in\mathbb{Z}\left[\sqrt{2}\right]\rightarrow x\cdot
        y\in\mathbb{Z}\left[\sqrt{2}\right]?\] If so, address the situation
        with the identity and the inverses with respect to this operation on
        $\mathbb{Z}\left[\sqrt{2}\right]$. Specifically, describe the units in
        $\mathbb{Z}\left[\sqrt{2}\right]$.

        Yes it is also an operation. The identity is $a=1,b=0$, or
        $1+0\sqrt{2}$. Inverses may not always exists though. For example
        consider $2+0\sqrt{2}$, the inverse would have to be $\frac{1}{2}$, but
        this is not in the set. Thus the units in
        $\mathbb{Z}\left[\sqrt{2}\right]$ are $\pm1+0\sqrt{2}$.
      \item Is $\mathbb{Z}\left[\sqrt{2}\right]$ a ring? A field? An integral
        domain?

        This is a ring, and an integral domain, but it is not a field.
      \item What if instead we were to consider
        \[\mathbb{Q}\left[\sqrt{2}\right]=\left\{a+b\sqrt{2}|a,b\in\mathbb{Q}\right\}?\]

        Using this definition, then more elements would have inverses, but not
        all of them. Consider $0+1\sqrt{2}$. The multiplicative inverse would
        have to be $\frac{1}{\sqrt{2}}$, but this is not in the domain. So this
        would still only be a ring, and an integral domain.
    \end{enumerate}
\end{enumerate}

\end{document}
