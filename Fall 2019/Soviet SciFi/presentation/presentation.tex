\documentclass[10pt]{armath}
\usepackage{amsmath}
\usepackage{csquotes}
\usepackage{enumitem}
\usepackage[english]{babel}
\usepackage[style=apa, natbib]{biblatex}
\usepackage[margin=1in]{geometry}
\addbibresource{references.bib}
\DeclareLabelname[movie]{
  \field{director}
  \field{producer}
}

\title{Konstantin Tsiolkovsky}
\author{Arden Rasmussen}
\date{\today}

\linespread{2.0}

\begin{document}
\maketitle

\section{Introduction}%
\label{sec:introduction}

We were introduced to Konstantin Tsiolkovsky in the article \textit{From Stalin
  to Sputnik and Beyond}, and he was also represented in the film that we have
watched \textit{Cosmic Voyage}. As we learned from the article and the movie,
he was a soviet rocket scientist, who pioneered many of the modern theory for
rockets and space travel. For my presentation I wanted to research him further
so that we can understand his long lasting influence on the science fiction
genre as a whole.

\section{Life}%
\label{sec:life}

He was born in Izhevskoye on September 17, 1857. When he was 9 years old, he
caught scarlet fever and as a result his hearing became impaired. Then only two
years later at the age of 13 his mother died. He was not accepted into
elementary schools because of his hearing problem, so he was home schooled, and
spent much of his time reading books. This is the time when he first became
interested in mathematics and physics, and began to consider space travel.

He became a teacher at a school in Borovsk near Moscow. The married his wife
Varvara Sokolova during this time as well. He continued to read fiction, and do
his own research during his time as a teacher.

Inspired by fiction, Tsiolkovsky began to construct theories about rocketry and
spaceflight. He would later be commonly considered the father of spaceflight.
He was also the first person to conceive of the concept of a space elevator,
apparently being inspired by the Eiffel Tower.

Tsiolkovsky experienced a number of unfortunate years at the start of the 20th
century. In 1902 his son Ignaty committed suicide, then in 1908 many of his
papers were lost in a flood, then in 1911 his daughter Lyubov was arrested for
revolutionary activities.

Tsiolkovsky supported the Bolshevik revolution, but he did not become
prominent, partially due to his support of eugenics making him politely
unpopular. He continued as a high school teacher until his retirement in 1920.
It was not until the mid 1920s that his theories gained traction, and the
importance was acknowledged. Then he only became popular is Soviet Russia in
1932. This was only three years before his death on September 19th 1935. He
died during an operation for his stomach cancer.

\section{Scientific Accomplishments}%
\label{sec:scientific_accomplishments}

Tsiolkovsky first began by outlining the basics of the kinetic theory of gases
in this paper "Theory of Gases", but these discoveries had previously been made
25 years earlier. His second work was titled "The Mechanics of Animal
Organism". Tsiolkovsky's major works focused on four subjects: an all-metal
airship, airplanes and trains, hovercraft, and rockets.

In the 1890s Tsiolkovsky primarily focused on an attempt to construct an
all-metal airship. To do this, he developed the first aerodynamics laboratory
in Russia, and constructed the first wind tunnel in Russia. In 1900 he made a
survey of the coefficients of drag for several simple shapes. He developed
descriptions of airflow around different geometric shapes, this would later
results in modern aerodynamics.

After failing to gain support for the construction of an all-metal airship,
Tsiolkovsky then moved his research to heavier the air aircraft. He first
developed then idea in an article called "An Airplane or a Birdlike Flying
Machine" in 1894. Many of the designs present in the article are similar to the
designs that were used in the beginnings of air flight. Unfortunately this again
did not gain enough support to continue with this research.

Starting in 1896 Tsiolkovsky studied the theory of motion for a rocket. In 1897
he developed the famous classic rocket equation, which is commonly used today
for rudimentary models of rocket propulsion.

\begin{align*}
  % \Delta v=v_e\ln\left(\frac{m_0}{m_f}\right)=I_{sp}g_0\ln\left\(\frac{m_0}{m_f}\right)
\end{align*}

In 1903, Tsiolkovsky published his work titled "Exploration of Outer Space by
Means of Rocket Devices". In this paper, he utilized his rocket equation to
determine the minimum orbit around the earth, and explained a process that
using a multistage rocket it would be possible to achieve this orbit. This is
the first time that it was proved that a rocket could preform space travel. He
continued to develop ideas on the use of liquid rocket engines for space
travel. This initial design that Tsiolkovsky proposed is the bases of modern
spaceship designs.

He continued to develop theories on rocketry and new ideas. He also studied a
large number of different rocket fuels, and determined which combinations of
oxidizers and combustibles would be most effective for space travel. Much of
his research has gone on to lead to the research necessary to develop modern
space travel.

Although many of his ideas were considered impractical, they were of great
influence to many of the early rocket scientists, and marked the path for the
later researchers to continue on. This all eventually lead to the construction
of the first successful spacecraft.

\section{Philosophy}%
\label{sec:philosophy}

Tsiolkovsky also did a large amount of philosophical work, he published a book,
titled \textit{The Will of the Universe. The Unknown Intelligence} in 1928.
Then throughout the last five years of his life, he wrote many articles on his
philosophical theories. Showing that he was not only a scientist but an active
philosopher.

\section{Influence}%
\label{sec:influence}

Tsiolkovsky's theories and research has had a tremendous influence in popular
culture, and science fiction. Much of science fiction with space travel can be
traced back to Tsiolkovsky's theories, and ideas. He developed ideas for
airlocks, and closed biological systems for growing food and producing oxygen
in outer space. And the development of the concept of the space elevator is
quite extensive in modern science fiction. He was also consulted for the script
for \textit{Cosmic Voyage}, and wrote several of his one works, such as
\textit{One the Moon}. To name a few works that Tsiolkovsky has had an
influence in are:
\begin{itemize}
  \item StarTrek
  \item International Space Station
  \item Mass Effect
  \item Doctor Who
  \item The Mars Trilogy
  \item Old Man's War
\end{itemize}
There are too many different works that have some influence from Tsiolkovsky,
almost all of modern science fiction, that exists in a setting with space
travel, has a large number of elements which were originally considered by
Tsiolkovsky.

Much of modern space science fiction would not be as it is today without
Tsiolkovsky. Without his theories, ideas and research, much of modern space
travel, air travel, and science fiction would be year behind, or be completely
different.

\nocite{*}
\printbibliography
\end{document}
