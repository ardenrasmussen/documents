\documentclass[11pt,a4paper,sans]{moderncv}
\moderncvstyle{classic}
\moderncvcolor{blue}
\usepackage[utf8]{inputenc}
\usepackage[scale=0.75]{geometry}
\name{Arden}{Rasmussen}
\address{975 Lovers Leap Rd}{95701 Alta, CA}
\phone[mobile]{+1~(775)~846~6599}
\email{ardenrasmussen@lclark.edu}
\homepage{https://github.com/LuxAtrumStudio}

\begin{document}
\makecvtitle{}

\section{Education}
\cventry{2012--2016}{High school}{Earl Wooster}{Reno, NV}{}{Graduated with
International Baccalaureate degree with focus in Math, Chemistry, and Physics.}
\cventry{2016--2020}{BS}{Lewis \& Clark}{Portland, OR}{}{Majoring in Computer
Science \& Mathematics and Physics. GPA: 3.7}

\section{Experience}
\cventry{2016--2016}{IT}{Lewis \& Clark}{Portland, OR}{}{Working at the
resource lab with Adobe suit, and assisting students and professors with
computers.}
\cventry{2016-2018}{Grading}{Lewis \& Clark}{Portland, OR}{}{Grading for a
selection of the Physics, Computer Science, and Mathematical courses.}

\section{Programming Languages}
\cvitemwithcomment{C/C++}{Expert}{10 years}
\cvitemwithcomment{Python}{Intermediate}{4 years}
\cvitemwithcomment{Javascript}{(React, Node.js, Express.js) Intermediate}{3 year}
\cvitemwithcomment{OS}{(Linux, Windows) Expert}{8 years}
\cvitemwithcomment{GPU}{(OpenGL, OpenGL, Cuda) Intermediate}{2 years}
\cvitemwithcomment{AI}{(TensorFlow, PyTorch, C++)}{2 years}

\section{Relevant Courses}
\begin{cvcolumns}
  \cvcolumn{Physics}{
    \begin{itemize}
      \item Theoretical Dynamics
      \item Electricity \& Magnetism
      \item Computational
      \item Quantum
    \end{itemize}
  }
  \cvcolumn{Math}{
    \begin{itemize}
      \item Real Analysis
      \item Advanced Topology
      \item Complex Variables
      \item Differential Equations
    \end{itemize}
  }
  \cvcolumn{CS}{
    \begin{itemize}
      \item Artificial Intelligence
      \item Computer Graphics
      \item Theory of Computation
      \item Computer Architecture
    \end{itemize}
  }
\end{cvcolumns}


\section{Relevant Projects}
\cvitem{Lexici}{I constructed an algorithmic and historical comparison of ~20
different programming languages.}
\cvitem{Ray Tracer}{I programmed an implementation of a ray tracer,
incorporating retractions, reflections, and model loading.}
\cvitem{Chat Client}{I developed the frontend and backend of a chat server,
implementing rich text messages.}
\cvitem{Mathematical}{I created an interpreted language for evaluating
arbitrary mathematical expressions, with arbitrary precision.}
\cvitem{FEM}{Independent research into the mathematics and developing an
implementation of finite element method.}
% \cvitem{Lexici}{I constructed a comparison of 20 different programming
% languages. I did this by implementing the same algorithm in each of the
% languages, and thus I was able to compare run time with respect to the
% source size of the code in order to fairly compare different programming
% languages. In addition to the data comparison, I have researched the influence
% tree of the different languages, and written up a description of how each
% language differs from the rest.}
% \cvitem{Ray Tracer}{I programmed a ray tracer implementation in C++, using
% OpenCL and multi-threading to accelerate the rendering time. I was able to
% implement reflections, refractions, and the importing of models/materials from
% several different modeling programs.}
% \cvitem{Chat Client}{I programmed a back end and a client side interface for an
% online message board system. The back end incorporated user authentication, with
% passwords saved in a salted hash, and the front end implemented rich text
% rendering, and a system to asynchronously updated messages from the server.}
% \cvitem{Mathematical}{I created an interpreted language based on standard
% mathematical expressions. The expressions are passed through a lexer and parsed
% into an abstract-syntax-tree. Then the tree is lazy evaluated in order to
% resolve the final answer to the expression. The language implements arbitrary
% precision floating point numbers, and all of the standard mathematical
% functions. The implementation of symbolic differentiation and integration in
% still in progress.}

\section{References}
\begin{cvcolumns}
  \cvcolumn{Name}{
    \begin{itemize}
      \item Michael Broide
      \item Jeffrey Ely
      \item Paul Allen
    \end{itemize}
  }
  \cvcolumn{Email}{
    \begin{itemize}
      \item[] broide@lclark.edu
      \item[] jeff@lclark.edu
      \item[] ptallen@lclark.edu
    \end{itemize}
  }
\end{cvcolumns}
\end{document}
