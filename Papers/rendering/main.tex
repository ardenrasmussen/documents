\documentclass[10pt]{amsart}
\usepackage{amsfonts}
\usepackage{amsmath}
\usepackage{amsthm}
\usepackage{amscd}

\usepackage[margin=1in]{geometry}

\usepackage{multicol}
\usepackage{subfiles}
\usepackage{caption}
\usepackage{enumitem}
\usepackage{graphics}

\newenvironment{Figure}
{\par\medskip\noindent\minipage{\linewidth}}
{\endminipage\par\medskip}


\title{Modern Ray Tracing}
\author{Arden Rasmussen}
\date{\today}

\begin{document}
\maketitle

\begin{abstract}
  Ray Tracing is the basis of most modern rendering systems, as it attempts to
  simulate the physices of light as it interacts with the environment. This
  paper serves to comment on the major methods that are used in modern
  implementations of ray tracing based rendering. There is significant more
  information that cannot be compleatly explained here, but the primary aspects
  of the major methods are explained. The three primary sections of this paper
  are \textit{rendering}, \textit{shading}, \textit{material}, and
  \textit{lighting}. These are the four main components in rendering engines.
  Each portion will be explained in more detail in its respective sections.
\end{abstract}

\begin{multicols}{2}
  \subfile{sec/rendering.tex}
\end{multicols}

\end{document}
