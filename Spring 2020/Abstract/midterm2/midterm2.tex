\hypertarget{problem-1}{%
\section{Problem 1}\label{problem-1}}

\begin{quote}
Festival of finite abelian groups.\footnote{I used \(\mathbb{Z}_i\) and
  \(\mathbb{Z}/i\mathbb{Z}\) interchanably, for eash of typing, sorry.}
\end{quote}

\hypertarget{a}{%
\subsection{a}\label{a}}

\begin{quote}
Please prove that \(\mathbb{Z}/7\mathbb{Z}\times\mathbb{Z}/7\mathbb{Z}\)
is not isomorphic to \(\mathbb{Z}/49\mathbb{Z}\).
\end{quote}

By Proposition 5.2.6 we known
\(\mathbb{Z}/7\mathbb{Z}\times\mathbb{Z}/7\mathbb{Z}\cong\mathbb{Z}/49\mathbb{Z}\)
if and only if \(gcd(7,7)=1\), but clearly \(gcd(7,7)=7\), so we
conclude that
\(\mathbb{Z}/7\mathbb{Z}\times\mathbb{Z}/7\mathbb{Z}\not\cong\mathbb{Z}/49\mathbb{Z}\).

\hypertarget{b}{%
\subsection{b}\label{b}}

\begin{quote}
Let \(A\) be an abelian group of order \(392\). List all possible
isomorphism classes of \(A\).
\end{quote}

By the fundamental theorem of finite abelian groups, we will first
consider the factorization of \(|A|=392=2^3\cdot7^2\). Thus the list of
all isomorphism classes of \(A\) are given below

\begin{itemize}
\tightlist
\item
  \(\mathbb{Z}_2\times\mathbb{Z}_2\times\mathbb{Z}_2\times\mathbb{Z}_7\times\mathbb{Z}_7\)
\item
  \(\mathbb{Z}_4\times\mathbb{Z}_2\times\mathbb{Z}_7\times\mathbb{Z}_7\)
\item
  \(\mathbb{Z}_8\times\mathbb{Z}_7\times\mathbb{Z}_7\)
\item
  \(\mathbb{Z}_2\times\mathbb{Z}_2\times\mathbb{Z}_2\times\mathbb{Z}_{49}\)
\item
  \(\mathbb{Z}_4\times\mathbb{Z}_2\times\mathbb{Z}_{49}\)
\item
  \(\mathbb{Z}_8\times\mathbb{Z}_{49}\)
\end{itemize}

\hypertarget{c}{%
\subsection{c}\label{c}}

\begin{quote}
Assume further that \(A\) contains an element of order \(196\). List the
possible isomorphism classes of \(A\).
\end{quote}

To enforce the existance of an element of order \(196=2^2\cdot7^2\) so
thus we know that this element is the pair of an element of order \(4\),
and an element of order \(49\). So from the list constructed in (1.b),
we will select the options with groups contaiings elements of these
orders.

\begin{itemize}
\tightlist
\item
  \(\mathbb{Z}_{4}\times\mathbb{Z}_{2}\times\mathbb{Z}_{49}\).
\item
  \(\mathbb{Z}_{8}\times\mathbb{Z}_{49}\).
\end{itemize}

\hypertarget{d}{%
\subsection{d}\label{d}}

\begin{quote}
Let
\(G=\mathbb{Z}/49\mathbb{Z}\times\mathbb{Z}/4\mathbb{Z}\times\mathbb{Z}/2\mathbb{Z}\).
Find subgroup \(H,K\) of \(G\) both of order \(2\) so that \(G/H\) and
\(G/K\) are not isomorphic.
\end{quote}

Consider \(H=1\times\langle 2\rangle\times 1\), and
\(K=1\times 1\times\mathbb{Z}_2\). It is clear to show that both \(H\)
and \(K\) have order two by their construction. Now we consider
\(G/H\cong(\mathbb{Z}/49\mathbb{Z}\times\mathbb{Z}/4\mathbb{Z}\times\mathbb{Z}/2\mathbb{Z})/(1\times\langle 2\rangle\times 1)\),
which is inturn isomorphic to
\[G/H\cong((\mathbb{Z}/49\mathbb{Z})/1)\times((\mathbb{Z}/4\mathbb{Z})/\langle
2\rangle)\times((\mathbb{Z}/2\mathbb{Z})/1)\cong
\mathbb{Z}/49\mathbb{Z}\times\mathbb{Z}/2\mathbb{Z}\times\mathbb{Z}/2\mathbb{Z}.\]

Repeating this for \(G/K\) we find
\[G/K\cong\mathbb{Z}/49\mathbb{Z}\times\mathbb{Z}/4\mathbb{Z}\times 1.\]

Then by a simmilar argument to (1.a), we can conclude that
\(\mathbb{Z}/2\mathbb{Z}\times\mathbb{Z}/2\mathbb{Z}\not\cong\mathbb{Z}/4\mathbb{Z}\times1\),
and so we can thus conclude that there is indeed no isomorphism between
\(G/H\) and \(G/K\).

\hypertarget{problem-2}{%
\section{Problem 2}\label{problem-2}}

\begin{quote}
Suppose a group of prime order \(p\) acts on a finite set. What are the
possible sizes of the orbits of this action?
\end{quote}

Given some group \(G\) of prime order \(p\), acting on a finite set
\(S\), then By propositin 4.1.2, we known that
\(|\mathcal{O}(s)|=|G:G_s|\) for all \(s\in S\). and we know that
\(|G|=|G:G_s||G_s|\). Thus we are able to express \[
|\mathcal{O}(s)|=\frac{|G|}{|G_s|}
\] Since, \(G_s\leq G\), then \(|G_s|\big\vert |G|\), by lagranges
theorem, since \(|G|\) is prime, then we must have \(|G_s|=1\) or
\(|G_s|=p\). Thus we conclude that \[
|\mathcal{O}(s)|=\frac{|G|}{1}=|G|=p\\
|\mathcal{O}(s)|=\frac{|G|}{p}=1.
\] so the orbits of this action must be of order \(1\) or \(p\).

\hypertarget{problem-3}{%
\section{Problem 3}\label{problem-3}}

\begin{quote}
Here is a nice fact: ``If \(G\) is a finite group and \(p\) is a prime
dividing \(|G|\), then \(G\) has an element of order \(p\)''. There are
many proofs of this fact. For this problem, please follow the steps
below to prove this result.
\end{quote}

\hypertarget{a-1}{%
\subsection{a}\label{a-1}}

\begin{quote}
Let \(S\) denote the set of ordered \(p\)-tuples of element sof \(G\)
the product of whose coordinates is \(1\). So \[
S=\left\{(x_1,x_2,\ldots,x_p):x_i\in G \text{ and } x_1x_2\cdots x_p=1\right\}.
\] Show that \(S\) contains \(|G|^{p-1}\) elements.
\end{quote}

We can consider the produce \(x_1x_2\cdots x_p=1\), as
\((x_1x_2\cdots x_{p-1})\cdot x_p=1\). In this case we can see that
\(x_p\) must be the inverse of the product of all the elements that come
prior to it. This means that after picking any value for \(x_1\) to
\(x_{p-1}\), then \(x_p\) is forced to be a specific value. Thus we are
able to select \(p-1\) elements from \(G\). So we can see that
\(|S|=|G|^{p-1}\).

\hypertarget{b-1}{%
\subsection{b}\label{b-1}}

\begin{quote}
We would like to define an action of the cyclic group of order \(p\),
\(C_p\), on \(S\). Do this by letting a permutation act on the indices
of an element of \(S\). Please prove that this is a group action.
\end{quote}

To prove that this is a group action, we will first show that for some
element \(a_1,a_2\in C_p\), and \((x_1,\ldots,x_p)\in S\), then
\(a_1.(a_2.(x_1\ldots,x_p))=(a_1a_2).(x_1,\ldots,x_p)\). First we note,
that \(a_1=a^\alpha\), and \(a_2=a^\beta\). Now let us compute \[
a^\alpha.(a^\beta.(x_1,\ldots,x_p))=a^\alpha.(x_{1+\beta},\ldots,x_{p-\beta})\\
=(x_{1+\beta+\alpha},\ldots,x_{p-\beta-\alpha})\\
=(x_{1+(\beta+\alpha)},\ldots,x_{p-(\beta+\alpha)})\\
=(a^(\alpha+\beta))(x_1,\ldots,x_p)\\
=(a^\alpha a^\beta)(x_1,\ldots,x_p).
\]

Now it remains to show that
\(a^0.(x_1,\ldots,x_p)=(x_1,\ldots,x_p)\quad\forall (x_1,\ldots,x_p)\in S\).
So let us compute \[
a^0.(x_1,\ldots,x_p)=(x_1,\ldots,x_p).
\] Thus we can conclude that this is indeed a group action.

\hypertarget{c-1}{%
\subsection{c}\label{c-1}}

\begin{quote}
Using your work above, including Problem 2, prove the nice fact.
\end{quote}

From problem 2, we know that the order of orbits in \(S\) of this action
must be either \(1\) or \(p\). We can see that the size of an orbit is
\(1\) is true only for tuples of the form \((x,x,\ldots,x)\), with
\(x^p=1\). Thus if there are any orbits of size \(1\) other than the
orbit of \((1,1,\ldots,1)\), then that element must be of order \(p\).
Let us assume that there is no such other orbit, so all orbits are of
size \(p\). Let us say that there are \(k\) orbits of size \(p\), and
\(m\) orbits of size \(1\). Then we can see that \(|G|^{p-1}=kp+m\).
Then since \(p\big\vert|G|\), we can also conclude that \(p|m\). Since
\((1,1,\ldots,1)\) has an orbit of size \(1\), then we know that
\(m\neq 0\), thus we must conclude that there are at least \(p-1\) other
elements that correspond to an orbit of size \(1\), and that implies
that \(x^p=1\).

\hypertarget{problem-4}{%
\section{Problem 4}\label{problem-4}}

\begin{quote}
The Class Equation expresses the order of a finite group as the sum of a
list of natural numbers \(n_1+n_2+\cdots+n_k\). Consider the following
sums. Please rule out those that could not appear on the right hand side
of the Class Equation. Please explain your reasoning.
\end{quote}

\hypertarget{a-2}{%
\subsection{a}\label{a-2}}

\begin{quote}
\(3+2+5\)
\end{quote}

With the equation of the form \(|Z(G)|+|G:C_G(g_1)|+|G:C_G(G_2)|\), we
know that \(Z(G)\leq G\), and so thus \(|Z(G)|\big\vert|G|\), thus
\(|Z(G)|=2\) or \(|Z(G)|=5\), Then we know that
\(|G:C_G(g_i)|=\frac{|G|}{|C_G(g_i)}\), and thus we can conclude that
for some \(i\), \(3=\frac{10}{|C_G(g_i)}\) and thus
\(|C_G(g_i)|=\frac{10}{3}\), which is impossible. Thus this cannot
appear on the right hand side of the class equation.

\hypertarget{b-2}{%
\subsection{b}\label{b-2}}

\begin{quote}
\(1+2+2+5\)
\end{quote}

This is the class equation.

\hypertarget{c-2}{%
\subsection{c}\label{c-2}}

\begin{quote}
\(1+2+3+4\)
\end{quote}

By a similar argument to (4.a), we know that each \(n_i\) must divide
\(|G|=10\), but \(3\), and \(4\) do not divide \(10\), and so this
cannot appear in the class equation.

\hypertarget{d-1}{%
\subsection{d}\label{d-1}}

\begin{quote}
\(2+2+2+2+2\)
\end{quote}

This is not possible, as then \(|Z(G)|=2\), and this leads to
\(|G/Z|=5\), thus \(G/Z\) is cyclic, and so we would conclude that \(G\)
is abelian. If \(G\) were abelian, then the class equation would be
\(10\).

\hypertarget{problem-5}{%
\section{Problem 5}\label{problem-5}}

\begin{quote}
Let \(E/F\) be an extension of fields. Suppose \(f(x),g(x)\in F[x]\) are
not both zero. Let \(d_F(x)\) be the gcd of \(f(x)\) and \(g(x)\) in
\(F[x]\). Now view \(f(x),g(x)\) as elements of \(E[x]\), and let
\(d_E(x)\) be the gcd of \(f(x)\) and \(g(x)\) in \(E[x]\). Show
\(d_F(x)=d_E(x)\). (This is a bit surprising since various questions
involving divisibility such as irreducibility depend on the field be
used.)
\end{quote}

Consider some \(f(x),g(x)\in F[x]\), with \(d_F(x)\) be the gcd of the
two in \(F[x]\). Then by the GCD theorem, we know that we can express
this as \[
d_F(x)=d=a(x)f(x)+b(x)g(x).
\] since \(d_E(x)\) divides both \(f\) and \(g\), then we must conclude
that \(d_E(x)\) also divides \(d_F(x)\). A similar argument can be made
to show that \(d_F(x)\) divides \(d_E(x)\). Thus we can conclude that
\(d_F(x)=d_E(x)\).

\hypertarget{problem-6}{%
\section{Problem 6}\label{problem-6}}

\begin{quote}
The algebraic numbers \(\mathcal{A}\) are all numbers in \(\mathbb{C}\)
that are algebraic over \(\mathbb{Q}\). They are a subfield of
\(\mathbb{C}\); you can assume this without proof. (Its not a bad proof,
feel free to enjoy it in a non-test setting.) Please prove that
\(\mathcal{A}\) is not a finite extension of \(\mathbb{Q}\). Give lots
of details.
\end{quote}

Let us consider the polynomials given by \(f(x)=x^{2n}-2\). We can see
that by using Eisenstein irreducibility criterion, with the prime \(2\),
that \(x^{2n}-2\) is irreducible in \(\mathbb{Z}\), and irreducible in
\(\mathbb{Q}\). We can see that \(\sqrt[2n]{2}\) is a root of this
polynomial, and thus \(\sqrt[2n]{2}\) is algebraic and so
\(\sqrt[2n]{2}\in\mathcal{A}\).

We see that the degree of the field extension is thus
\([\mathbb{Q}(\sqrt{2},\ldots,\sqrt[2n]{2}):\mathbb{Q}]=2n\), as
\(\left\{1,\sqrt{2},\ldots,\sqrt[2n]{2},\sqrt{2}\sqrt[4]{2},\ldots\right\}\)
form a basis. Since
\(\mathbb{Q}(\sqrt{2},\ldots,\sqrt[2n]{2})\subseteq\mathcal{A}\). Then
we know that
\(2n=[\mathbb{Q}(\sqrt{2},\ldots,\sqrt[2n]{2}):\mathbb{Q}]\leq[\mathcal{A}:\mathbb{Q}]\),
for any \(n\in\mathbb{N}\). Thus we can see that
\([\mathcal{A}:\mathbb{Q}]\geq 2n\) for all \(n\in\mathbb{N}\), and
since \(\mathbb{N}\) is infinite, then \([\mathcal{A}:\mathbb{Q}]\) must
also be infinite.
