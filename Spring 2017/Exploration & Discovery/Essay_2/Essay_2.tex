\documentclass[12pt, twoside]{article}

\usepackage{hyperref}
\usepackage[backend=bibtex, style=mla]{biblatex}
\usepackage[margin=1in]{geometry}
\usepackage{fancyhdr}
\addbibresource{Essay_2.bib}
\title{The Discrimination of Immigrant Laborers}
\author{Arden Rasmussen}
\date{15 March 2017}
\pagestyle{fancy}
\fancyhf{}
\fancyhead[LE,RO]{Arden Rasmussen}
\fancyhead[RE,LO]{\Large The Discrimination of Immigrant Laborers}
\fancyfoot[C]{\thepage}
\setlength{\headheight}{22pt}
\linespread{1.6}
\begin{document}
\pagenumbering{gobble}
\maketitle
\newpage
\pagenumbering{arabic}
\par
Immigrants have played a vital role in the development of America and yet are often overlooked. However, in the past, there were wildly varying changes in the American population's attitudes towards the immigrant workers. There have been several times in America's history where immigrant laborers have been sought after, and then a few years layer are being deported. These sudden fluctuations between needed immigrants to wanting to be rid of them, was caused by a number of reasons. Americans production could not be paused in times when there were few American workers, and when those times ended, there was an overabundance of labor available in the U.S. Because of these sudden shifts in the perception of the immigrant labor force, they were often discriminated against and segregated from the rest of the American population. And to some extent, the humanity of the immigrants was forgotten.
\par
The immigrants in the Northwest had a difficult time dealing with the needs and demands of the American industries. The need for immigrant workers varied greatly based upon the American economy. When the economy was doing well, there were plenty of jobs, and even to some extents not enough people to fill the jobs. This was when the industries called for more immigrant workers who could fill the available positions. This also happened furring wars, when much of the American work force was at war, there was a large need for someone to be back in the states working to keep the country moving, and this is where the immigrants came into play. However, when the work force came back, there was an over abundance of people, and not nearly enough jobs for everyone. This caused must dislike of the immigrant workers because the American citizens couldn't get a job because they were all taken. This is a clear cause of discrimination against the immigrant workers. Because they were tanking the jobs of the other American workers. This is despite the fact that the immigrants are equally if not more qualified than their American counter parts. From the text of \emph{Oregon's Hispanic Heritage}, it is clear that the United States wanted Mexican immigrants to come work. ``Even as European and Asian immigration was restricted, land needed to be cleared, railroads constructed, irrigation projects developed, and far production expanded. Much of this growth relied extensively on a cheap and plentiful supply of Mexican labor''\cite{OHH}. Because of advancing industries in the United States. there was a great desire for the cheap labor the Mexicans provided. However, this influx of Mexican workers would soon end. ``Beginning in 1930 and lasting until World War II, employment opportunities evaporated and Mexican laborers were no longer welcomed in the U.S.''\cite[147]{OHH}. Because Americans also needed jobs, the Mexican laborers became unwanted as they did the same work for less money, and therefor they took many jobs the Americans could have occupied. This easy come and go of the need and hate of the foreign laborers is clearly evident. As it was a mater of a few years, that Mexicans were being brought into the States for work, to when they were being forceful deported. This is a clear example of one of the several attributes that caused the American opinion of immigrant workers to fluctuate so quickly.
\par
Another issue that the immigrant laborers had to deal with was the work opportunities that were provided to them was extremely limited. Most jobs that the laborers received was simple labor jobs or positions that American workers didn't want to have. It was very rare for a Chinese immigrant to have a fully professional job, what they could maintain for a long period of time. Most were just employed as common laborers or railroad construction workers. One point that stands out, is that it would be believed that as time went on, Americans would grow more adjusted to the Chinese laborers, and provide them with more professional jobs. However, reality proved to be the opposite of this. As time passed, the number of Chinese immigrants with a job that was considered a professional job actually dropped in the Northwest area. In contrast to the drop in professional jobs, the number of Chinese who were employed as railroad workers jumped by about 1200% from 17 to 207 workers\cite[13]{TCIO}. By comparing the percentage of Chinese who worked is different categories it is clear that there is a large shift occurring. Many jobs which were more stable such as domestic worker, and laundry worker fell, some to a greater extent than others. While the only categories that had a large increase were the number or railroad workers, and cannery workers, both of which are considered less desirable positions. This is a clear shift form the Chinese working in a more integrated environment with Americans, to begin only good for physical labor, and the lowest class of work. This shift in professions of the Chinese could be one of the other causes for the sudden shift in attitude. Because it was already thought of that they were only good for medial and simple jobs, they were not necessarily in society, and could easily be removed without much consequence.
\par
Another challenge that the Chinese labor force had to deal with, was the general attitude of the American people. Although the American people never fully liked that Chinese laborers, they were still needed as they filled the less desirable positions and did so extremely cheaply. However, the American citizen's attitude towards the Chinese was never the warmest. There was clear and unrelenting racism against the Chinese workers. People in Oregon never viewed the Chinese as part of their communities. ``They were not seen as permanent residents, much fewer citizens of the state or its communities''\cite[5]{RCSP}. This racism even extends into the law where the Chinese minors were required to bribe the sheriff in order to keep their possessions\cite[5]{RCSP}. Despite the clear importance of the labor the Chinese immigrants provided, they were treated with clear discrimination and placed in the lowest social classes. Even when the Chinese were segregated and lived in their own communities, this was again used as a tool for propaganda against them. ``[I]t did lend credence to the anti-Chinese propaganda that these `Celestials' were clannish and huddled together.''\cite[9]{TCIO}. This clear discrimination and racism, in both society and law, against the Chinese meant that when other factors came into effect, the American people would already be glad to be rid of the immigrants. Although this was not a direct cause for the greater attitude change, it was a factor that meant that there would be little to no effort to assist the Chinese, and when possible America would, in fact, remove the Chinese to provide more for other Americans. With the other factors coming into effect as attitudes towards the Chinese workers turned from bad to worse, this clear racism only grew more and more.
\par
The Chinese workers and other immigrant workers have come to the Northwest for many opportunities that are possible here, but they have not alway been welcome. Although there was a period where they were greatly needed for the cheap and efficient work that they could provide in the construction of the railroads, and other simple labor positions, this social relationship was not stable and would soon crumble. The collapse of the attitudes towards that immigrant workers came from an already founded deep rooted racism against them. The fact that most of the immigrants worked in simple labor jobs and very few were vitally important for anything, meant that they were extremely expendable. So when the time came there was no difficulty in removing them from their positions and finding replacements. A third   contributor is the fact that the demand for immigrants fluctuated greatly. Because Americans always wanted jobs before their immigrant counterparts, the number of available positions depended upon the number of American workers. Thus during a time of war, where few Americans were home for the jobs, many immigrants came to fill the positions, however, when the American workforce came back there were far too many people for far too few jobs. Throughout history, there has been discrimination and racism of immigrants, and Americans often ignore the difficulties that we have caused immigrants in this nation and focus on only how they harmed us. This is a mindset that needs to be broken so that we many consider everyone, and create a better society.
\newpage
\printbibliography
\end{document}
