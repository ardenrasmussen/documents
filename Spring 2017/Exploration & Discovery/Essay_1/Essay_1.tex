\documentclass{article}

\usepackage{hyperref}
\usepackage[backend=bibtex, style=mla]{biblatex}
\usepackage{geometry}
\addbibresource{Essay_1.bib}

\title{White Colonialism Lead to the Destruction of Native American Society}
\date{15 February 2017}
\author{Arden Rasmussen}

\geometry{
	a4paper,
	left = 1in,
	right = 1in,
	top =1in,
	bottom =1in,
	}
\linespread{1.6}

\begin{document}
\pagenumbering{gobble}
\maketitle
\newpage
\pagenumbering{arabic}

\par
In the earliest days of exploration in the Pacific Northwest, and specifically in the Columbia Basin, there was a great many clashing of ideas between the natives of the land and the new explorers. One of the main concepts that differed between these two groups of people was the idea of how man lived with the land. It appeared that the natives had worked their lives to live in harmony with the land, and only take what they needed. While at the same time it seemed that the explorers were instead looking for how they could remake the land so to better fit themselves, and their eastern lifestyle. This difference between these two ideas is extremely noticeable in texts describing the interactions between the natives and the new white explorers. While settler colonialism served the settlers very well, changing the landscape to suit whites' needs could only ever serve to undermine and destroy Native Americans' way of life, and therefore their literal lives. The white settlers though that in order for their own success, they needed to remove the Native Americans, because the societal differences would not be able to conform together. This left two options, convert the natives to the settlers way of life, or to remove the natives.
\par
In \textit{Northwest Passage; The Great Columbia River} by Dietrich, there are many accounts of when the natives and the explorers clashed over their disputes over how to treat the land. In all of these accounts, a common situation occurs. The natives don't want to change their land, and the white explorers are planning to change everything to suit them better. Throughout the early years of exploration in the northwest, explorers were keen on changing the environment. One of the motivators for their desire to change the land could be the fact that it was not their land. In a passage from Dietrich; "The European newcomers were from elsewhere, and if they trapped out or overgrazed or cut or mined or farmed an area to exhaustion, their experience told them they could move on"\cite[151]{NWP}. This is a viewpoint that the European settlers had when exploring this new land, that if they overused the resources that were available here, it would be easy to move on to a new place that they could start again. This is a viewpoint that is common in modern humanity, as major current issues are that we are depleting our resources, and killing the planet, however, this time there is nowhere to move on to. The Europeans viewed the land as something that they could exploit, and use up. This new territory was just another resource for them, and nothing more. The views of the Europeans was that land was not a home, but an opportunity to make money. They did not care for the land itself, the settlers simply saw it as a way to benefit themselves. As Dietrich says "Geography was fluid, a map of opportunity, something to be exploited"\cite[151]{NWP}. 
\par
Then natives voices were often posed against the European voices. The natives disagreed with the European viewpoint that the land could be destroyed and then they would move on, instead, the natives worked to preserve the land for as long as they could and make decisions to help their children and later generations. One clear example of this is in the section on the salmon in the Columbia river. "Explorer David Thompson was equally surprised by the willingness of the tribes to let large numbers of salmon go by to spawn, even when their members were hungry"\cite[155]{NWP}. This is in comparison to the fisheries which would catch as many salmon as was possible in a day. This description of the tribes makes it clear that the natives were conscious that their actions would affect the future. They were clearly making sure that some of the salmon were getting past their nets so that the following year there would be enough fish to feed everyone again. This shows that the natives didn't want to change their land because they would let fish though so that in the following years their ancestors could still remain fed off of the fish in the river. Some of the natives were very strong opinionated about farming when other tribes tried the practice. One of these was the prophet Smohalla. Smohalla said "But the work of the white man hardens soul and body. Nor is it right to tear up and mutilate the earth as white men do"\cite[152]{NWP}. He claims that the work of white men (farming) hardens the soul, and prevents men from dreaming, and becoming wise. He claims that it is unnatural work to be farming and transforming the land. Smohalla the creates a very descriptive metaphor, of the earth being the mother of humanity. "You ask me to dig for stone. Shall I dig under her skin for her bones?"\cite[152]{NWP}. He relates the mutilation of the earth, which can be considered the mother of all humanity to the mutilation of his own mother, which by anyone's opinions would be disgraceful, and yet it is not disgraceful for the white man to destroy the mother of humanity.
\par
The land of the natives in the northwest was solid and unchanging. That is to say that the different tribes had inhabited the same land for many generations, and had never moved out of their territory. However, when the white settlers arrived the tribes were suddenly forced to move off of their land, onto predetermined territories. These territories were often horrible conditions, and possess no useful resources. This is another way that the white settlers viewed the land as reformable and ever changing. Because the Europeans had changed and moved around with such regularity, the assumption was made that the natives could do so as well. But this is shown to be incorrect, as the native tribes in many cases refused to move off of their land, and would fight for it. "...the first Indian war in the Pacific Northwest"\cite{OHP}. The natives did not what their land to be changed, and this came in the form of the land itself begin changed, and the area that they could claim to be their land begin changed. Either form of this meant that the land was begin changed, altered, and commonly destroyed for farming. Even the land that was specifically set aside for the natives to inhabit was often reduced, for the white settlers to take advantage of the resources there. In the 1860's a Nez Perce territory was greatly reduced because of the white man's greed. "[Nez Perce] had reserved 7.5 million acres of land ... but this land base was reduced to less than a million acres in the early 1860s when gold was discovered in the area"\cite{OHP}. Although this land had been set aside for the Nez Perce tribes, the white settlers refused to respect this when it was realized that there were valuable resources that they could be exploiting from the land that the reservation inhabited. Another reservation is gradually reduced as more and more white farmers want to use the land for their farms. "The reservation quickly shrank as white farmers and townspeople demanded that it be 'opened up'"\cite{OHP}. Because of this, the government continued to reduce the land that was available for the tribes living on the reservation and allowed the white farmers to use more of the land destroying the earth and reshaping it because it would make them more money.
\par
When the white men were attempting to convince the natives to move off of their land, they would attempt to use a variety of persuasive arguments and bribes, as it is laid out in \textit{The Indians Hear a Treaty Speech}. Some of the items that the white men claim that they will do for the natives was; "The Americans will watch over you. They will make your fences. They will plow your land. They will fence your land. They will make you houses"\cite{VOH}. Although all of the objects that the speech claims will be given to the natives appear to be beneficial at first look, they are also all centered around changing the land and the natives to better suit the white man's wants. Although these propositions would not literally kill the natives, it was targeted to destroy their ways of life and convert them to become more like the white settlers and conform to the status that the settlers brought to the new land. The proposition would even give the natives clothes that would match that of the white settlers'. "You bill be given everything...breeches, hats, coats, overcoats"\cite{VOH}. The proposition even clearly states that "So then you will not be poor. All your tribespeople will be just like Americans"\cite{VOH}. This is phrased in a way that implies that to not conform to the white people's way of living would mean they are missing out on all the benefits of the white society. All of the actions of the settlers were to change the landscape to better suit themselves, and this is just an example of changing the social landscape to fit themselves, instead of the alternative of adapting to the natives' way of life.
\par
White settlers were relentless in their converse of the Pacific Northwest to better suit themselves. Settlers dug up land to create farms, fished the rivers into extinction, and converted the culture to become their own. Anything that didn't work out as they wanted, the settlers would either destroy it completely or would push it out of the way to a spot where it could go on ignored. The settlers destroyed land to the dismay of the Native Americans to better suit their farming. They fished the salmon almost to extinction, a population that the natives had been carefully maintaining for many years. The settler attempted to convert as many Native Americans to the "modern" way of life. However, the tribes that resisted the assimilation to the American culture where forced into the reservations, and away from their historical land. Because their land was valuable for the settlers, who would exploit the resources that lay in the soil. The white settlers came in and began changing the physical and social landscape to match what they had previously known and disregard the effect on the Native Americans and on the environment.
\newpage
\printbibliography
\end{document}
