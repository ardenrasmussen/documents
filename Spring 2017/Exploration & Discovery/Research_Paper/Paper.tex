\documentclass[12pt,twoside]{article}
\usepackage{hyperref}
\usepackage[backend=bibtex]{biblatex}
\addbibresource{Paper.bib}
\usepackage[margin=1in]{geometry}
\usepackage{fancyhdr}
\title{Dams on the Columbia River}
\author{Arden Rasmussen}
\date{01 May 2017}
\linespread{1.5}
\pagestyle{fancy}
\fancyhf{}
\fancyhead[LE,RO]{Arden Rasmussen}
\fancyhead[RE,LO]{Dams on the Columbia}
\fancyfoot[C]{\thepage}
\setlength{\headheight}{22pt}
\begin{document}
\pagenumbering{gobble}
\maketitle
\newpage
\pagenumbering{arabic}

Dams along the Columbia river have been an indisputably important factor in the
long history of the northwest. The dams that were constructed along the rivers
in the area, provided huge amounts of electricity and water to vast new areas
in the northwest. Then on top of that with the massive surplus of water that
was made available, the agriculture industry grew rapidly, and expanded to
cover more territory in of the inland areas of the state. The electricity that
was made extremely cheap allowed more intensive industries like manufacturing
to become extremely viable and profitable in the northwest. Theses industries
kick-started the economy of the northwest and drew thousands of employees to
the area. However, these dams were also very negative for many people. They
destroyed the natural flow of the river. They flooded many traditional lands
that were used by the native Americans, and the dams greatly disrupted the
migration of native fish populations. There are may voices that speak out in
favor of the dams such as large companies, or major businesses, nominally the
people who are supporting the dams have an economic advantage. Then there are
the many other groups who are opposed to the dams along the Columbia river,
such as the native Americans, the environmentalists, many fishermen. These
people have much less economic interest in the dams, and some actually have an
economic interest in the lack of the dams. Each side of the argument has their
own voices for and against the damming of the river. The damming along the
Columbia river has been a critical par of the history of the northwest, and
without it, the region would not be anything that it is today, however, as
newer technology has allowed the benefits of the dams without the great amounts
of environmental harm, there has been a more successful push for the removal of
some of the dams along the Columbia. The dams on the Columbia were more
beneficial when they were constructed, but as time has passed their cons have
begun to out weigh the pros of the dams.

The first explorers to come to the northwest were on the search for a major
transportation river, that could connect to the eastern half of the country.
This great river highway would allow for Americans to easily cross the country
by boat, and would open up a wealth of new opportunities. The Columbia River
was initially hoped to be a major transportation network, but as further
exploration revealed, the river was extremely difficult to navigate, and the
section of the river that were navigable was extremely one sided. A ship could
take weeks traveling a short distance up river, a distance which could be
covered in only a few days when traveling down river. This discrepancy between
the efficiency of the travel speeds meant that the Columbia river and its major
tributaries were not a viable network for transportation of goods. And in a
world where every major city is constructed on the ocean or a major river for
transportation, this meant that the Columbia river would not be well suited for
many large cities further up river than Portland which was considered the
furthest upriver that major ocean ships could travel. However, with the
construction of the dams, this dream of the Columbia being a major
transportation network has come closer to a reality than it ever has been
before. It is now possible to ship thousands of tons of goods up and down the
Columbia river, and it is all only because of the dams along the river to
create the reservoirs that river boats can successfully navigate. Although it
is not perfect, and has caused numerous other issues, the Columbia river has
been a major shipping route for the northwest, shipping thousands of tons of
goods a year~\cite{TRM}. With the vast amounts of goods that are exported by
means of the Columbia river, all of which would not be possible at the present
scale if not for the dams along the river. However, with newer technology such
as aircraft, or more efficient trains, these goods can be shipped faster and
for similar to lesser costs then they historically could by means of the river.

The dams along the Columbia and Snake river are vital to the agriculture
industry of the northwest. The dams provide huge amounts of water for the
inland farms that were constructed in the desert of the northwest. However,
this extremely fertile soil can be accessed, because of the dams. Dams such as
the Grand Coulee dam were constructed to easily provide irrigation water to
thousands of farmers in the area. The water that these dams can provide for
inland farmers has been a key factor in the agricultural growth of the
northwest. This water has in fact allowed the Columbia and Snake river area the
largest exporter of wheat in the United States, with an export amount of
currently 11 million tons of grain exported from the Columbia
river~\cite{USDA_GRAIN}. Thousands of farmers rely on the water provided by the
dams along the Columbia river, there are approximately 5,100,000 acres of land
that are irrigated by the Columbia river water~\cite{NWC_IRR}. And this water
is a vital part of the industry. Other than hydroelectric power, agriculture is
the next largest consumer of Columbia river water. The huge industry of
agriculture needs the dams to exist, as most of the extremely fertile land in
the northwest is located in deserts, and does not naturally receive enough
water for proper irrigation, and agriculture. One clear aspect of a voice that
clearly shows its support of the dams is from the Washington Grain Commission
(WGC). The WGC has a list of ``Facts'' on their website, which although all of
their sources are trustworthy, the facts that were selected all clearly
demonstrate their bias towards the dams, and their clear support of the
dams~\cite{WGC}. The reliance of the Columbia river dams for irrigation water,
and the fact that without the dams, millions of acres of farmland would not
exist, is a clear sign that the agricultural industries heavily relies on the
dams, and needs them in order to be profitable.

The large farmers were greatly benefited by the Columbia basin project, because
the dams provided plenty of water and electricity for any farm land. Even
though the initial plans were for much smaller farm lands, and so the units of
land were started in a much smaller range of 10 to 160 acres depending on the
quality of the land~\cite{PCP}. However, these small sized farm units grew
outdated, as new technology made it easy for a few farmers to maintain a large
amount of land.This growth of technology while the rules of the lands remained
stagnant was a major cause for the shift to larger farms. Because a few farmers
could easily manage land in access of the previous maximum 160 acres, congress
passes a new law that allowed farm operators to hold multiple units each up to
160 acres, and along with that, there was the free ability to lease farm units
to others~\cite{PCP}. So it was common for a few farmer to be managing much
more land that the plan initially intended. The average farm size grew from 84
acres to 107 acres.

The small farmers in the northwest were greatly affected by the construction of
the dams. As the dams allowed their water and electricity to be extremely
cheap. This meant that all the farms in the region could easily water all of
their land and keep their crops sustainable for many years. Thousands of acres
of land was opened up for the small farmers to use, and for thousands of
immigrants to begin a new life with. Regions around the rivers that had a dam
were suddenly a destination for travelers because of the vast availability of
the land. Many of the counties even created hundreds of blocks of land that
were allotted to farmers who wanted them. Most of these farm lots consisted of
140 acres, so every farm was relatively small, and designed to be managed by a
single family. All of these steps were taken in order to benefit the small
farmer as much as possible. However, their results were the opposite of what
the planners initially wanted. In the following years most of these small farms
were taken over by large super farms. These super farms control hundreds of
acres, much beyond the initially allotted 140 acres. These super farms were
able to thrive in the situations that dams had created. Because electricity and
water was mad extremely cheap, the super farms could easily afford to pay for
the irrigation of their enter lands. Then with the advancement in technology,
these super farms could maintain more land with fewer workers, making it much
more viable to control large tracts of land. Along with all of those factors,
the farming equipment that was necessary for irrigation of the land, was very
expensive, so the small farmers who needed it could not afford it, but the
super farms, could easily afford the equipment, and would earn enough money to
pay it off quickly. Although the dams were constructed with the intention of
supporting the small farmer, they incidentally created an environment that was
perfect for large scale industrialized farms. 

Small family run farms, is most peoples hope of the perfect farm lands.
However, this is rarely the reality, instead most farm land is large scale
corporate owned farm land, that is run by machines, and very few workers. With
the Columbia Basin Project many organizers were hopeful that they could change
this and achieve and area of family run farmlands. The Columbia Basin Project
was an initiative that aligned with the construction of the dams along the
Columbia river, to irrigate much of the extremely dry Columbia basin. The
project very ambitiously planed to irrigate 1,100,000 acres of land with the
water that would be backed up due to the construction of the Grand Coulee Dam.
The plan was to distribute the land into blocks of 140 acres, and divide this
land up among thousands of small farm families who could then live there, and
being a new farm~\cite{GCD_CBP}. The initial intentions were valid for the
planers, to try to draw small farmers to the area with cheap and subsidized
land, then provided government-subsidized irrigation for the land by means of
the Grand Coulee Dam. However, this did not proceed as it was initially
intended. The first draft of the plan was to only provide small farms for
family farmers, however as time passed, and the price for the dam grew, the
organizers came to realize that larger farms would be much more profitable and
pay for the cost of the dam much sooner, than small farms. Thus the size of the
farm plots grew, and grew until it was no longer for small family farmers, but
for large corporate organizations~\cite{CBP}. Another issue that occurred, was
that the federal government and the local one, both wanted to assist small
farmers, but they both wanted to have control of the project, as a result of
this neither organization succeeded, and the family farms began to fail in
favor to the larger corporate farms. One of the major causes for the failure of
these smaller farms, was technology. As technology rapidly developed, the large
farms could easily afford the new tools, that made them even more cost
efficient, while a small family farmer could not even dream of catching up with
the rapidly developing technology of agriculture tools~\cite{CBP}. The Columbia
basin project came nowhere close to what it was initially to be, and as more
time passes, antiquated decisions still hold, for example the government still
supplies many farmers in the area with water at significantly subsidized
prices.

Industry in the northwest was vital for the development of the population
centers in the region. Large industries create a high demand for workers, this
demand for workers makes the northwest a better destination for potential
migrants, who would have a job in the region. This means that the large number
of industries would have been a large draw for the thousands of Americans who
were moving to the west. One of the major industries that was created in the
northwest was the aluminum smelting. The aluminum mills required a huge amount
of electricity to run, and that was exactly what the dams on the Columbia river
would provide. Because the dams created such a surplus of electricity, and the
rivers could be used to export the aluminum, the locations around the dams
became the perfect location for these aluminum mills. The northwest became such
a large area for aluminum production that it was the largest aluminum exporter
in the country. These huge industries caused a huge draw for more people.
Because that was a great amount of possible profit in the aluminum industry,
but they needed many more employees, the number of people coming to the
northwest grew greatly. These major industries were a major kick start to the
economy of the northwest,as well as a jump start in the population of the
region. All of this resulted from the industries that sprouted up around the
Columbia river, and the industries were drawn to the Columbia because of the
cheap electricity that the river dams offered. Because of this the dams helped
kick start the economy and the population of the region. Along with the draw of
industries, the dams themselves provided thousands of jobs. During the peak of
construction of the Grand Coulee dam, there was approximately 8,800 workers,
only some of those were directly construction workers, the others were for
telephone, and water and other resources that were necessary for the small city
of people that came to work on the dam~\cite{MOTA}.

The largest of the dams along the Columbia river was the Grand Coulee dam,
which began construction in 1935 and was completed in 1941~\cite{GCD}. The
completion of the dam had a large effect on the industries in the region, in
particular the aluminum industry. When the dams along the river were completed,
the federal government encouraged aluminum companies to construct aluminum
smelters around the dam, in order to have a use for the vast amounts of energy
that was produced~\cite{OAI}. Many aluminum smelters were drawn to the
Northwest, by the invitation from the federal government and the whole sale
price of electricity. They were also drawn to the accessibility to aluminum
consumers, such as aircraft producers~\cite{OAH}. However, this draw of
industry was not a permanent one, as time has passed, the price of electricity
has become less important in the process of aluminum production, and has been
swapped for a favor of being located closer to the source of the aluminum ore,
and reduce transpiration costs. Because the northwest does not produce any of
its own aluminum ore, the aluminum industry has begun to move to location where
the ore is more readily available, such as china~\cite{OAH}. However, these
aluminum smelters were a critical part in drawing in industries to the
northwest, and support the economy of the developing region, and these
industries were only drawn because of the cheap electricity that was provided
by the dams.

When viewing the time of completion for the Grand Coulee Dam (1941), and the
census data for Oregon around that time, there is a clear jump in population
the few years after the dam is completed. The population jumped 114,000 people
in one year, which previous years was only showing a growth of about 20,000
people. This sudden growth is 5.7 times the average growth rate of the time,
and occurs the year following the construction of the dam~\cite{O_POP}.
Although the population of Oregon has grown much more than what it used to be,
and the jump present at that time appears less influential, the total
population at the time was only 1,107,000, so this jump represented about
$\%9.71$ of the population at the time. This is an undeniable growth of people,
which occurred directly after the completion of the largest dam along the
Columbia river. Some of these people came because of the offered land for small
farmers, as the dam provided water to more than 2,000 farms~\cite{GC}. And many
of the people form the population boom would be resultant of the entrance of
the aluminum industries, which employed more that 2,000 employees in Oregon
alone. It can be reasoned that the Grand Coulee dam was a major cause for the
influx of people.

Unfortunately these industries were placed around the river dams because they
required immense amounts of electrify, which then made is less available for
others. Another unfortunate side effect of these industries, in particular the
aluminum smelters, is that they produce an extreme amount of pollution.
Although these industries created many job positions, and greatly helped the
economy of the region, they were also extremely damaging to the natural
environment. These industries that were draw to the river dams, were damaging
an environment that was already hurt by the construction of the dams
themselves. All of these industries were very important in their own way for
the legacy of the northwest and the Columbia river, but they all were also very
damaging to an already damaged environment. And as there realizations have come
to be more prevalent in today's society, these industries are moving else
where, and leaving the northwest. This departure of the industries, means that
the power of the hydroelectric plants were not going to use, and so now these
dams cannot provide their maximum amount of energy, and are becoming much less
useful. A few of the dams along the rivers have been re-proposed to supply
electricity to data centers. Data centers are the new major industry, that
needs huge amounts of electricity to run, and the dams are one of the only
sources for them. The benefit that data centers have over other historically
conventional industries, is that location does not matter in any way. However,
data centers are not needed on every dam along the river, only one or two of
the dams along the rivers will be used for data centers. This means that there
are many more river dams that will lose most of their use, as public
electricity can be sourced from else where, and the major electricity consumers
leave.

The aluminum smelters were in themselves a large source of environmental
damage. Because of the process of extraction the aluminum metal creates a large
amount of green house gasses, which is then released into the atmosphere. When
people are already disapproving of the dams on the river for disrupting the
natural environment, the fact that more pollutant producing industries came to
the region, angered many of the environmentalists. The aluminum industries that
came to the region produce hundreds of tons of green house gasses a year. Some
of which are orders of magnitude more damaging the common carbon
dioxide~\cite{APA}. Just the fact that these dams enticed the aluminum smelters
to come to the northwest was a direct cause of the destruction of the
environment, more so that the dams were already causing to the environmental
damage. Although these industries that were drawn to the northwest because of
the dams had a significant footprint on the economy of the northwest, and were
able to bring thousands of workers to the region, the overall environmental
impact was disastrous~\cite{OAI}. The aluminum smelting plants produced immense
amounts of pollution, and because they were being provided electricity at a
whole sale rate, it was possible for them to produce aluminum at a very rapid
rate and for much less. And now as time has passed, and the aluminum smelters
are favoring more profitable locations for their plants, the plants in the
northwest are disappearing, and becoming obsolete, leaving their permanent mark
on the global environment, and moving to a new land where they can start anew.

When the planing begin for the dam construction along the Columbia river, the
native Americans were not taken into account in any way. Because of this
several native American villages were torn up in order for the dams to be
constructed, and with out much more than a though for the tribes that resided
in those villages. At most they were provided with some temporary shelter, and
at least they were provided with nothing~\cite{ST}. Many families were evicted
from their traditional homelands and forced to move to new lands, without being
compensated for anything. The most that the government has done is designate
several fishing sites along the river to make up for the traditional locations
that were flooded by the backwater of the dam reservoirs. The construction of
the dams displaced dozens of people, across several tribes, and the non-native
Americans who were displaced by the dam construction were treated in a much
different manner. The non-native Americans were provided permanent homes by the
government, along with assistance in relocation~\cite{ST}. This is a clear
distinction to the native tribes that had lived on these lands for generations,
who where forced to move without much consideration and with little to no
compensation for the loss of their historical tribal home. The view of the
planners tasked with the construction of the dams was all focused on the
economical gain that could be achieved, and they did not consider the natural
environments that they would be destroying, or the traditional lands of natives
that would be flooded. These dams were only created for the interest of the
investors, and with the price tag that they cam with it makes sense that the
primary view of the dam is how soon it can pay off the price. However, this
allowed the environment and the people how lived off of the river in its
current sate to become neglected by the project planers. 

The dams have had a very large and indisputable impact on the native fish
populations along the Columbia river, and its tributaries.The dams cause it to
be extremely difficult for many anadromous fish species to travel up and down
from their native spawning grounds and to the ocean. Despite the sheer
magnitude of money that has been poured into creating ways for the fish to
navigate around the dams, the damage has already been done, and cannot be
undone. Even as the powerful companies that are profiting form the dams attempt
to make amends by constructing fish hatcheries along the rivers, this is no
substitute for the original strand of fish that developed in the rivers and
tributaries. Although there have been continues development to find new ways to
help fish to swim past the dams. However, any method that is used for moving
fish around the dams can never be as good as the natural river was, as that is
what the fish evolved to do, and when this path was disrupted, there is almost
nothing that can substitute that. The two major fish species that live in the
Columbia river area are the salmon and steelhead. The first issue is that the
dams block passage of the fish between their spawning habitat and the ocean.
Many major dams have included fish ladders or other mechanisms to assist in
transporting fish around the dams, but there are many more dams that do not
include some form of fish passage, and because of this $\%55$ of the once
accessible spawning area has been blocked off permanently by dams~\cite{SS}.
The reservoirs that were created by the dams, can also be equally dangerous to
the salmon and steelhead. Because dams cause the flow of water of slow, the
temperature of the water can rise to dangerously high levels, and can even
become so hot that it is lethal to the fish~\cite{SS}. The variant flow of
water can be deadly for the fish as well. Because the dams further upriver can
cause the water level to rise and fall, the fish eggs can be shifted closer to
the surface. Then when the dam reduces the throughput, the water level falls,
and these fish eggs have been stranded in the open air, where they fish inside
will die~\cite{SS}. The natural environment that these fish evolved for has
been completely destroyed by the dams. Even though the dams have attempted to
make amends, by constructing fish hatcheries, however these have actually
resulted in the diluted genetic lines, which is also harming the fish
populations.

The dams along the Columbia River have been of great importance in the years
around their construction. The dams themselves were a huge provider of
employment during construction, then they drew a large number of industries
which also provided large numbers of jobs. Both of these were extremely
influential on the economy of the northwest at the time, and helped draw
thousands of people to move to the northwest. These dams also assisted in the
creation of much of the agricultural area in the northwest, through irrigation
and several programs to provide land to new farmers. However, the dams were
also a great cause of damage to the natural environment, and to some of the
people of the northwest. The native Americans were ignored, and displaced by
the construction of the dams, and a multitude of different habitats were
destroyed, and particularly the salmon and steelhead fish populations have lost
much of their traditional habitats, and routes. Although the dams were
extremely important when they were created, as they have provided vast amounts
of utility, they have begun to become obsolete with time. The major industries
that needed the electricity are moving away and farms have become extremely
efficient at utilizing very little electricity. And the region has grown enough
that the economy can be sustained through other means. Now the dams are mainly
damming the fish populations. Because of this gradual reduction in the utility
of the dams, some of them can be removed. However, some dams should always
remain standing, as they still have great utility in flood prevention, and
water management, but the number of dams that currently stands is no longer
necessary.

\newpage
\printbibliography{}
\end{document}
