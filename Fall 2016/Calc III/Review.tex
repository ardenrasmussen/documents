\documentclass{article}

\usepackage{multicol}
\usepackage{pgfplots}
\usepackage{graphicx}
\usepackage{amsmath}
\graphicspath{ {graphs/} }
\pgfplotsset{width=10cm,compat=1.9}

\title{CALCULUS III REVIEW}
\date{10-12-2016}
\author{Arden Rasmussen}

\begin{document}
\pagenumbering{gobble}
\maketitle
\newpage
\tableofcontents
\newpage
\pagenumbering{arabic}
\section{Geometry}
\subsection{Cartesian Coordinates}
Commonly sketching an equation and explaining it in words. Try to sketch out the graph, checking several point. Then attempt to classify the shape.
\subsection{Vector Algebra}
Things that appear commonly with vector algebra questions.
\begin{itemize}
\item Sketching

Simply drawing the vector from base point.
\item Transformation to vector from $u,v$

Given $\vec{a}$ and $\vec{b}$.
\begin{gather*}
\left(\begin{array}{ccc}
x \\ y \\ z
\end{array}\right) = 
\left(\begin{array}{ccc}
x_{0} \\ y_{0} \\ z_{0}
\end{array}\right) + u
\left<\begin{array}{ccc}
\vec{a_{x}} \\ \vec{a_{y}} \\ \vec{a_{z}}
\end{array}\right> + v
\left<\begin{array}{ccc}
\vec{b_{x}} \\ \vec{b_{y}} \\ \vec{b_{z}}
\end{array}\right> \\
T(u,v) = \left\{\begin{array}{ccc}
x = x_{0} + u \vec{a_{x}} + v \vec{b_{x}} \\
y = y_{0} + u \vec{a_{y}} + v \vec{b_{y}} \\
z = z_{0} + u \vec{a_{z}} + v \vec{b_{z}}
\end{array}
\right.
\end{gather*}
\item Dot Product ($\cdot$)
\begin{gather*}
a_x b_x + a_y b_y + a_z b_z = 
\Vert\vec{a}\Vert \cdot \Vert\vec{b}\Vert \cdot \cos (\gamma)
\end{gather*}
\item Cross product
\begin{gather*}
\vec{a}\times\vec{b} = \left|\begin{array}{ccc}
\vec{i} & \vec{j} & \vec{k} \\
a_x & a_y & a_z \\
b_x & b_y & b_z
\end{array}\right|
\end{gather*}
\item Length of a vector
\begin{gather*}
L = \sqrt{\vec{a} \cdot \vec{b}}
\end{gather*}
\item Area of parallelogram
\begin{gather*}
A^2 = \det \left|\begin{array}{cc}
\vec{a} \cdot \vec{a} & \vec{a} \cdot \vec{b} \\
\vec{a} \cdot \vec{b} & \vec{b} \cdot \vec{b} 
\end{array}\right|
\end{gather*}
\item Volume of parallelotope
\begin{gather*}
V^2 = \det \left|\begin{array}{ccc}
\vec{a} \cdot \vec{a} & \vec{a} \cdot \vec{b} & \vec{a} \cdot \vec{c} \\
\vec{a} \cdot \vec{b} & \vec{b} \cdot \vec{b} & \vec{b} \cdot \vec{c} \\ 
\vec{a} \cdot \vec{c} & \vec{b} \cdot \vec{c} & \vec{c} \cdot \vec{c}
\end{array}\right|
\end{gather*}
\end{itemize}
\subsection{Curvilinear Coordinates}
Determining transformations, and parameterization of equations, computing $\partial_u$ and $\partial_v$, computing area element($dA$), and normal vector field($\vec{N}$).
\begin{gather*}
Parameterization:T(u,v)
\end{gather*}
\begin{itemize}
\item $\partial_u$ / $\partial_v$
\begin{gather*}
\partial_u = \left<\begin{array}{cc}
\frac{\partial_x}{\partial_u} & \frac{\partial_y}{\partial_u}
\end{array}\right>\\
\partial_v = \left<\begin{array}{cc}
\frac{\partial_x}{\partial_v} & \frac{\partial_y}{\partial_v}
\end{array}\right>
\end{gather*}
\item Area element($dA$)
\begin{gather*}
dA = \sqrt{\det \left|\begin{array}{cc}
\partial_u \cdot \partial_u & \partial_u \cdot \partial_v \\
\partial_u \cdot \partial_v & \partial_v \cdot \partial_v 
\end{array}\right|}
\end{gather*}
\item Normal Vector Field($\vec{N}$)
\begin{gather*}
\vec{N} = \partial_u \times \partial_v
\end{gather*}
\end{itemize}
\subsubsection{Polar Coordinates}
\begin{gather*}
x=r\cos(\theta)\\
y=r\sin(\theta)\\
(r, \theta)\\
dA = r dr d\theta
\end{gather*}
\subsubsection{Cylindrical Coordinates}

\begin{gather*}
x=r\cos(\theta) \\
y=r\sin(\theta) \\
z=z \\
(r, \theta, z) \\
dA = r dr d\theta dz
\end{gather*}
\subsubsection{Spherical Coordinates}
\begin{gather*}
x = r\sin(\phi)\cos(\theta) \\
y = r\sin(\phi)\sin(\theta) \\
z = r\cos(\phi) \\
(r, \theta, \phi) \\
dA = r^2\sin(\phi) dr d\theta d\phi
\end{gather*}
\section{Multi-Variable Integration}
Setting up integrals and evaluating them.
\subsection{1 Dimensional}
\begin{gather*}
\int \rho\ dS \\
\int \rho(u)\ du
\end{gather*}
\subsection{2 Dimensional}
\begin{gather*}
\iint \rho\ dA \\
\iint \rho(u,v)\cdot Stretch\ dudv
\end{gather*}
\subsection{3 Dimensional}
\begin{gather*}
\iiint \rho\ dV \\
\iiint \rho(u,v,w)\cdot Stretch\ dudvdw
\end{gather*}
\section{Differential Calculus}
\subsection{Linearizion/Taylor Approximation}
\begin{gather*}
Jacobian: \\
DT = \left[\begin{matrix}
\partial u_1 & \partial v_1 \\
\partial u_2 & \partial v_2
\end{matrix}\right] \\
Linearization: \\
\left(\begin{matrix}
x \\ y
\end{matrix}\right) \approx f(x_0, y_0) + 
\left[\begin{matrix}
\frac{dx}{du}|_{(x_0,y_0)} & \frac{dx}{dv}|_{(x_0,y_0)} \\
\frac{dy}{du}|_{(x_0,y_0)} & \frac{dy}{dv}|_{(x_0,y_0)}
\end{matrix}\right] \cdot \left[\begin{matrix}
\Delta u \\ \Delta v
\end{matrix}\right]\\
x \approx x_0 + \left[\begin{matrix}
\frac{dx}{du}|_{(x_0,y_0)} & \frac{dx}{dv}|_{(x_0,y_0)}
\end{matrix}\right] \cdot \left[\begin{matrix}
\Delta u \\ \Delta v
\end{matrix}\right]\\
y \approx y_0 + \left[\begin{matrix}
\frac{dy}{du}|_{(x_0,y_0)} & \frac{dy}{dv}|_{(x_0,y_0)}
\end{matrix}\right] \cdot \left[\begin{matrix}
\Delta u \\ \Delta v
\end{matrix}\right]\\
\\
Quadratic Approximation:
\end{gather*}
\begin{multline*}
\left(\begin{matrix}
x \\ y
\end{matrix}\right) \approx f(x_0, y_0) + 
\left[\begin{matrix}
\frac{dx}{du}|_{(x_0,y_0)} & \frac{dx}{dv}|_{(x_0,y_0)} \\
\frac{dy}{du}|_{(x_0,y_0)} & \frac{dy}{dv}|_{(x_0,y_0)}
\end{matrix}\right] \cdot \left[\begin{matrix}
\Delta u \\ \Delta v
\end{matrix}\right]\\ + \frac{1}{2} \cdot
\left[\begin{matrix}
\Delta x & \Delta y
\end{matrix}\right] \cdot 
\left[\begin{matrix}
\frac{d^2x}{du^2}|_{(x_0,y_0)} & \frac{d^2x}{dudv}|_{(x_0,y_0)} \\
\frac{d^2y}{dudv}|_{(x_0,y_0)} & \frac{d^2y}{dv^2}|_{(x_0,y_0)}
\end{matrix}\right] \cdot
\left[\begin{matrix}
\Delta x \\ \Delta y
\end{matrix}\right]
\end{multline*}
\subsection{Optimization(With or Without Constraints}
Determine when the $Jacobian$ is $0$. That is a critical point, then using $Hessian$ and quadratic approximation to determine shape($Bowl$, $Saddle$, $etc.$).
HELP!
\subsection{Chain Rule}
Derivative of outside evaluated at $(x_0,y_0)$ of the inside times the derivative of the inside evaluated at the base point.
\begin{gather*}
(x,y)=T_1(u, v)\\
(z,w)=T_2(x, y)\\
DT_1 = Jacobian\ for\ T_1\\DT_2 = Jacobian\ for\ T_2\\
D(T_2 \circ T_1) = DT_2|_{(x_0,y_0)} \cdot DT_1|_{(u_0, v_0)}\\
DT_2(T_1(u,v))\cdot DT_1(u,v)\\
\left[\begin{matrix}
\frac{dz}{du} & \frac{dz}{dv} \\ \frac{dw}{du} & \frac{dw}{dv}
\end{matrix}\right] = \left[\begin{matrix}
\frac{dz}{dx} & \frac{dz}{dy} \\ \frac{dw}{dx} & \frac{dw}{dy}
\end{matrix}\right] \cdot \left[\begin{matrix}
\frac{dx}{du} & \frac{dx}{dv} \\ \frac{dy}{du} & \frac{dy}{dv}
\end{matrix}\right]
\end{gather*}
\section{Vector Calculus}
\subsection{What are (div/curl/grad/potential)}
\subsubsection{Divergence}
\begin{enumerate}
\item Extent that a vector field exits a bound area.
\item Extent that a vector field goes away from a point.
\end{enumerate}
\begin{gather*}
div(\vec{V}) = \frac{\partial P}{\partial x} + \frac{\partial Q}{\partial y} + \frac{\partial R}{\partial z}
\end{gather*}
\subsubsection{Circulation}
\begin{enumerate}
\item Extent that a vector field follows the boundary of a bound area.
\item Extent that a vector field goes around a point.
\end{enumerate}
\begin{gather*}
\overrightarrow{curl}(\vec{V}) = \left<\frac{\partial R}{\partial y} - \frac{\partial Q}{\partial z}, \frac{\partial P}{\partial z} - \frac{\partial R}{\partial x}, \frac{\partial Q}{\partial x} - \frac{\partial P}{\partial y}\right>
\end{gather*}
\subsubsection{Gradient}
$grad$ is the rate of change in $f(x,y)$ as $x$ and $y$ change.
\begin{gather*}
\overrightarrow{grad}(f) = \left<\frac{\partial f}{\partial x}, \frac{\partial f}{\partial y},\frac{\partial f}{\partial z}\right>
\end{gather*}
\subsubsection{Potential}
HELP!
\begin{enumerate}
\item Create a system of equations equaling $\overrightarrow{grad}(f)$ to $\vec{V}$.
\item Solve the first equation for $f$ through integration, resulting with $C$ in terms of the other variables.
\item Derive $f$ in terms of $y$ and $z$ with a $\frac{\partial C}{\partial y}$ and a $\frac{\partial C}{\partial z}$.
\item Equal the other equations to the just derived values for $\frac{\partial f}{\partial y}$ and $\frac{\partial f}{\partial z}$.
\item Solve the second equation for $C$ through integration, resulting with a $D$ in terms of the final variable.
\item Derive $C$ in terms of the final variable with a $\frac{\partial D}{\partial z}$.
\item Equal the final equation to the just derived value for $\frac{\partial C}{\partial z}$.
\item Solve the final equation for $D$ through integration, resulting with an $E$ that is just a number.
\item Combine all solved functions into one final $f$.
\end{enumerate}
\begin{gather*}
\left\{\begin{matrix}
\frac{\partial f}{\partial x} = \vec{V}_x \\
\frac{\partial f}{\partial y} = \vec{V}_y \\
\frac{\partial f}{\partial z} = \vec{V}_z
\end{matrix}\right. \\
\Downarrow \\
f = f(x,y,z) + C(y,z) + D(z) + E
\end{gather*}
\subsection{State the Fundamental Theorems}
HELP!
\begin{itemize}
\item Gradient Theorem:
\begin{gather*}
\int_{a}^{b} \overrightarrow{grad}(f) = f(b) - f(a)
\end{gather*}
\item Stokes' Theorem:
\begin{gather*}
\iint_S \overrightarrow{curl}(\vec{V})\cdot\vec{N}dA = \int_C \vec{V} \cdot \vec{T} ds
\end{gather*}
\item Divergence Theorem:
\begin{gather*}
\iiint_\Omega div(\vec{V})dA = \iint_S \vec{V} \cdot \vec{N} ds
\end{gather*}
\item Green's Theorem(Circulation):
\begin{gather*}
\iint_S curl(\vec{V})dA = \int_C \vec{V} \cdot \vec{T} ds
\end{gather*}
\item Green's Theorem(Divergence):
\begin{gather*}
\iint_S div(\vec{V})dA = \int_C \vec{V} \cdot \vec{N} ds
\end{gather*}
\end{itemize}
\subsection{Compute Using Fundamental Theorems}
Enter in equation, and use theorems when possible.
\end{document}