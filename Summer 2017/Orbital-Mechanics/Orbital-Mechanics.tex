\documentclass[12pt]{article}
\usepackage{amsmath}
\usepackage{mathtools}
\usepackage{graphicx}

\title{Orbital Mechanics\\\small{Wikipedia}}
\date{\today}
\author{Arden Rasmussen}

\begin{document}
\pagenumbering{gobble}
\maketitle
\newpage
\pagenumbering{roman}
\begin{abstract}
  Orbital mechanics or astrodynamics is the applicaiton of pallistics and celestial mechanics to the practicale problems concerning the motion of rockets and other spacecraft. The motion of these objects is usually calculated from Newton's laws of motion and Newton's law of universal gravitation. It is a core disciplin within space mission design and control. Celestial mechanics trats more broadly the orbital dynamics of systems under the influence of gravity, including both spacecraft and natural astronomical bodies such as star systems, planets, moons, and comets. Orbital mechanics focuses on spacecraft trajectories, including orbital maneuvers, orbit plane changes, and interplanetary transfers, and is used by mission planners to predict the result of propulsive maneuvers. General relativity is a more exact theory than Newton's laws for calculating orbits, and is sometimes necessary for greater accuracy or in high-gravity situations (such as orbits close to the Sun).
\end{abstract}
\newpage
\tableofcontents
\newpage
\pagenumbering{arabic}

\section{Laws of astrodynamics}

The fundamental laws of astrodynamics are Newton's law of universal gravitation and Newton's laws of motion, while the fundamental mathematical tool is his differential calculus.

Every orbit and trajectory outside atmospheres is in principle reversible, i.e., in the space-time function the time is reversed. The velocities are reversed and the accelerations are the same, including those due to rocket bursts. Thus if a rocket burst is in the direction of the velocity, in the reversed case it is opposite to the velocity. Of course in the case of rocket bursts there is no full reversal of events, both ways the same delta-v is used and the same mass ratio applies.

Standard assumptions in astrodynamics include non-interference form outside bodies, negligible mass for one of the bodies, and negligible other forces (such as from the solar wind, atmospheric drag, etc.). More accurate calculations can be made without these simplifing assumptions, but they are more complicated. The increased accuracy often does not make enough of a difference in the calculation to be worthwhile.

Kepler's laws of planetary motion may be derived from Newton's laws, when it is assumed that the orbiting body is subject only to the gravitational force of the central attractor. When an engine thrust or propulsive force is present, Newton's laws still apply, but Kepler's laws are invalidated. When the thrust stops, the resulting orbit will be different but will one again be described by Kepler's laws. The three laws are:

\begin{enumerate}
  \item The orbit of every planet is an ellipse with the sun at one of the foci.
  \item A line joining a planet and the sun sweeps out equal areas durring equal intervals of time.
  \item The squares of the orbital periods of planets are directly proportional to the cubes of the semi-major axis of the orbits.
\end{enumerate}
\subsection{Escape velocity}

The formula for an escape velocity is easily derived as follows. The specific energy (energy per unit mass) of any space vehicle is composed of two components, the specific potential energy and the specific kinetic energy. The specific potential energy associatied with a planet of mass \(M\) is given by

\[\epsilon_{p}=-\frac{GM}{r}\]

while the specific kinetic energy of an object is given by

\[\epsilon_{k}=\frac{v^{2}}{2}\]

Since energy is conserved,

\[\epsilon=\frac{v^{2}}{2}-\frac{GM}{r}\]

does not depend on the distance, \(r\), from the center of the central body to the space vehicle in question. Therefor, the object can reach infinite \(r\) only if this quantity is nonnegative, which implies

\[v\geq\sqrt{\frac{2GM}{r}}\]

The escape velocity from Earth's surface is about \(11\frac{km}{s}\), but that is insufficient to send the body an infinite distance because of the gravitational pull of the Sun. To escape the Solar System from a location at a distance from the Sun equal to the distance Sun-Earth, but not close to the Earth, requires around \(42\frac{km}{s}\) velocity, but there will be "part credit" for the Earth's orbital velocity for spacecraft launced from Earth, if their further acceleration (due to the propulsion system) carries them in the same direction as Earth travels in its orbit.

\subsection{Formulae for free orbits}

Orbits are conic sections, so the formula for the distance of a body for a given angle corresponds to the formula for that curve in polar corrdinates, which is:

\begin{align*}
  r &= \frac{p}{1+e\cos\theta} \\
  \mu &= G(m_1+m_2) \\
  p &=\frac{h^2}{\mu}
\end{align*}

\(\mu\) is called the gravitational parameter. \(m_1\) and \(m_2\) are the masses of objects 1 and 2, and \(h\) is the specific angular momentum of object 2 with respect to object 1. The parameter \(\theta\) is known as the true anomaly, \(p\) is the smi-latus rectum, while \(e\) is the orbital eccentricity, all obtainable form the various forms of the six independent orbital elements.

\subsection{Circular orbits}

All bound orbits where the gravity of a central body dominates are elliptical in nature. A special case of this is the circular orbit, which is an ellipse of zero eccentricity. The formula for the velocity of a body in a circular orbit at distance \(r\) from the center of gravity of mass \(M\) can be derived as follows -

Centribetal acceleration matches the acceleration due to gravity. So,

\[\frac{v^{2}}{r}=\frac{GM}{r^{2}}\]

Therefore,

\[v=\sqrt{\frac{GM}{r}}\]

where \(G\) is the gravitational constant, equal to

\[6.67384x10^{-11}\frac{m^{3}}{kg\cdot s^2}\]

To properly use this formula, the units must be consistent; for example \(M\) must be in kilograms, and \(r\) must be in meters. The answer will be in meters per second.

The quantity \(GM\) is often termed the standard gravitational parameter, which has a different value for every planet or moon in the solar system.

Once the circular orbital velocity is known, the escape velocity is easily found by multiplying by the square root of 2:

\[v=\sqrt{2}\sqrt{\frac{GM}{r}}=\sqrt{\frac{2GM}{r}}\]

To escape from gravity, the kinetic energy must at least match the negative potential energy. So,

\[\frac{1}{2}mv^{2}=\frac{GMm}{r}\]

and therefore,

\[v=\sqrt{\frac{2GM}{r}}\]

\subsection{Elliptical orbits}

If \(0<e<1\), then the denominator of the equation of free orbits varies with the true anomaly \(\theta\), but remains positive, never becoming zero. Therefore, the relative posiiton vector remains bound, having its smallest magnitude at periapsis \(r_{p}\), which is given by:

\[r_{p}=\frac{p}{1+e}\]

The maximum value \(r\) is reached when \(\theta=180^{\circ}\). This point is called the apoapsis, and its radial coordinate, denoted \(r_{a}\), is

\[r_{a}=\frac{p}{1-e}\]

Let \(2a\) be the distance measured along the apse line from periapsis P to apoapsis A, as illustrated in the equation below:

\[2a=r_p+r_a\]

Substituting the equations above, we get:

\[a=\frac{p}{1-e^2}\]

\(a\) is the semimajor axis of the sllipse. Solving for \(p\), and substituting the result in the conicc section curve formula above, we get:

\[r=\frac{a(1-e^2)}{1+e\cos\theta}\]

\subsubsection{Orbital period}

Under standard assumptions the orbital period (\(T\)) of a body traveling along an elliptic orbit can be computed as:

\[T=2\pi\sqrt{\frac{a^3}{\mu}}\]

where:

\begin{itemize}
  \item \(\mu\) is standard gravitational parameter,
  \item \(a\) is length of semi-major axis.
\end{itemize}

Conclusions:

\begin{itemize}
  \item The orbital period is equal to that for a circular orbit with the orbit radius equal to the semi-major axis (\(a\)),
  \item For a given semi-major axis the orbital period does not depend on the eccentricity.
\end{itemize}

\subsubsection{Velocity}

Under standard assumptions the orbital speed (\(v\)) of a body traveling along an elliptic orbit can be computed from the vis-viva equation as:

\[v=\sqrt{\mu\left({\frac{2}{r}-\frac{1}{a}}\right)}\]

where:

\begin{itemize}
  \item \(\mu\) is the standard gravitational parameter,
  \item \(r\) is the distance between the orbiting bodies.
  \item \(a\) is the length of the semi-major axis.
\end{itemize}

The velocity equaiton for a hyperbolic trajectory has either \(+\frac{1}{a}\), or it is the same with the convention that in that case \(a\) is negative.

\subsubsection{Energy}

Under standard assumptions, specific orbital energ (\(\epsilon\)) of elliptic orbit is negative and the orbital energy conservation equaiton (the vis-viva equaiton) for this orbit can take the form:

\[\frac{v^2}{2}-\frac{\mu}{r}=-\frac{\mu}{2a}=\epsilon<0\]

where:

\begin{itemize}
  \item \(v\), is the speed of the orbiting body,
  \item \(r\), is the distance of the orbiting body from the center of mass of the central body,
  \item \(a\), is the semi-major axis,
  \item \(\mu\), is the standard gravitational parameter.
\end{itemize}

Conclusions:

\begin{itemize}
  \item For a given semi-major axis the specific orbital energy is independent of the eccentricity.
\end{itemize}

Using the viral theorem we find:

\begin{itemize}
  \item the time-average of the specific potential energy is equal to \(2\epsilon\)
    \begin{itemize}
      \item the time-average of \(v^{-1}\) is \(a^{-1}\)
    \end{itemize}
  \item the time-average of the specific kinetic energy is equal to \(-\epsilon\)
\end{itemize}

\subsection{Parabolic orbits}

If the eccentricity equals 1, then the orbit equaiton becomes:

\[r=\frac{h^2}{\mu}\frac{1}{1+\cos\theta}\]

where:

\begin{itemize}
  \item \(r\), is the radial distance of the orbiting body from the mass center of the central body,
  \item \(h\), is specific angular momentum of the orbiting body,
  \item \(\theta\), is the true anomaly of the orbiting body,
  \item \(\mu\), is the standard ravitational parameter.
\end{itemize}

As the true anomaly \(\theta\) approaches \(180^{\circ}\), the denominator approaches zero, so that \(r\) tends towards infinity. Hence, the energy of the trajectory for which \(e=1\) is zero, and is given by:

\[\epsilon=\frac{v^2}{2}-\frac{\mu}{r}=0\]

where:

\begin{itemize}
  \item \(v\), is the speed of the orbiting body.
\end{itemize}

In other words, the speed anywhere on a parabolic path is:

\[v=\sqrt{\frac{2\mu}{r}}\]

\subsection{Hyperbolic orbits}

if \(e>1\), the orbit formula,

\[r=\frac{h^2}{\mu}\frac{1}{1+e\cos\theta}\]

describes the geometry of the hyperbolic orbit. The system consits of two symmetric curves. The orbiting body occupies one of them. The other one is its empty mathmatical image. Clearly, the denominator of the equation above goes to zero when \(\cos\theta=\frac{-1}{e}\). We denote this value of true anomaly

\[\theta_{\infty}=\cos^{-1}\left(-{\frac{1}{e}}\right)\]

since the radial distance approaches infinity as the true anomaly approaches \(\theta_{\infty}\), known as the true anomaly of the asymptote. Observe that \(\theta_{\infty}\) lies between \(90^{\circ}\) and \(180^{\circ}\). From the trigonometric identity \(\sin^{2}\theta+\cos^{2}\theta=1\) it follows that:

\[\sin\theta_{\infty}=\frac{1}{e}\sqrt{e^{2}-1}\]

\subsubsection{Energy}

Under standard assumptions, specific orbital energy \(\epsilon\) of a hyperbolic trajectory is greater than zero and the orbital energy conservation equaiton for this kind of trajectory takes form:

\[\epsilon=\frac{v^2}{2}-\frac{\mu}{r}=\frac{\mu}{-2a}\]

where:

\begin{itemize}
  \item \(v\), is the orbital velocity of orbiting body,
  \item \(r\), is the radial distance of orbiting body from central body,
  \item \(a\), is the negative semi-major axis,
  \item \(\mu\), is standard gravitational parameter.
\end{itemize}

\subsubsection{Hyperbolic excess velocity}

Under standard assumptions the body traveling along hyperbolic trajectory will attain in infinity an orbital velocity called hyperbolic excess velocity (\(v_\infty\)) that can be computed as:

\[v_\infty=\sqrt{\frac{\mu}{-a}}\]

where:

\begin{itemize}
  \item \(\mu\) is standard gravitational parameter,
  \item \(a\) is the negative semi-major axis of orbit's hyperbola.
\end{itemize}

The hyperbolic excess velocity is related to the specific orbital energy or characteristic energy by

\[2\epsilon=C_3=v_{\infty}^2\]

\section{Calculating trajectories}

\subsection{Kepler's equation}

One approach to calculaiting orbits (mainly used historically) is to use Kepler's equation:

\[M=E-\epsilon\cdot\sin E\]

where \(M\) is the mean anomaly, \(E\) is the eccentric anomaly, and \(\epsilon\) is the eccentricity.

With Kepler's formuala, finding the time-of-flight to reach an angle (true anomaly) of \(\theta\) from periapsis is broken into two steps:

\begin{enumerate}
  \item Compute the eccentric anomaly \(E\) from true anomaly \(\theta\)
  \item Compute the time-of-flight \(t\) from the eccentric anomaly \(E\)
\end{enumerate}

Finding the eccentric anomaly at a given tim e(the inverse problem) is more difficlut. Kepler's equation is transcendental in \(E\), meaning it cannot be solved for \(E\) algebraically. Kepler's equaiton can be solved for \(E\) analytically by inversion.

A solution of kepler's equation, valid for all real values of \(\epsilon\) is:

\begin{align*}
  E=\begin{dcases}
  \sum\limits_{n=1}^{\infty} {\frac{M^{\frac{n}{3}}}{n!}} \lim\limits_{\theta \to 0} {\left(\frac{\text{d}^{n-1}}{\text{d}\theta^{n-1}} {\left(\frac{\theta}{\sqrt[3]{\theta-\sin\theta}}\right)}^{n}\right)} , & \epsilon = 1 \\
  \sum\limits_{n=1}^{\infty} {\frac{M^{n}}{n!}} \lim\limits_{\theta \to 0} {\left(\frac{\text{d}^{n-1}}{\text{d}\theta^{n-1}} {\left(\frac{\theta}{\theta-\epsilon \cdot \sin \theta}\right)}^{n}\right)} , & \epsilon \ne 1
  \end{dcases}
\end{align*}

Evaluating this yields:

\begin{displaymath}
  \hspace*{-0.0 \textwidth} \resizebox{ 1.0 \textwidth}{!}{$E=\begin{dcases} x + \frac{1}{60} x^3 + \frac{1}{1400} x^5 + \frac{1}{25200} x^7 + \frac{43}{17248000}x^9 + \frac{1213}{7207200000} x^{11} + \frac{151439}{12713500800000} x^{13} \cdots \ | \ x=(6M)^\frac{1}{3} , & \epsilon = 1 \\ \frac{1}{1-\epsilon}M-\frac{\epsilon}{(1-\epsilon)^4}\frac{M^3}{3!} + \frac{(9\epsilon^2+\epsilon)}{(1-\epsilon)^7}\frac{M^5}{5!} - \frac{(225\epsilon^3+54\epsilon^2+\epsilon)}{(1-\epsilon)^{10}}\frac{M^7}{7!} = \frac{(11025\epsilon^4 + 4131\epsilon^3+243\epsilon^2+\epsilon)}{(1-\epsilon)^{13}}\frac{M^9}{9!} \cdots , & \epsilon \ne 1 \end{dcases} $
    }
\end{displaymath}

Alternatively, Kepler's Equation can be solved numerically. First one must guess a value of \(E\) and solve for time-of-flight; then adjust \(E\) as necessary to bring the computed time-of-flight closer to the desired value until the required precision is achieved. Usually, Newton's method is used to achieve relatively fast convergentce.

The main difficulty with this approach is that is=t can take prohibitively long to converge for the extream elliptical orbits. FOr near-parabolic orbits, eccentricity \(\epsilon\) is nearly \(1\), and plugging \(e=1\) into the formula mean anomaly, \(E - \sin E\), we find ourselves subtracting two nearly-equal values, and accuracy suffers. For near-circular orbits, it is hard to find the periapsis in the first place (and truly circular orbits have no periapsis at all). Furthermore, the equation was derived on the assumption of an elliptical orbit, and so does not hold for parabolic or hyperbolic orbits. These difficulties are what led to the development of the universal variable formulation, described below.

\subsection{Conic orbits}

For simple procedures, such as computing the delta-v for coplanar transfer ellipses, traditional approaches are fairl effective. Others, such as time-of-flight are far more complicated, especially for near-circular and hyperbolic orbits.

\subsection{The patched conic approximation}

The Hohmann transfer orbit alone is a poor approximation for interplanetary trajectories because it neglects the planets' own gravity. Planetary gravity dominates the behavior of the space craft in the vicinity of a planet and in most cases Hohmann severely overestimates delta-v, and produces highly inaccurate prescriptions for burn timings.

A relatively simple way to get a first-order approzimation of delta-v is based on the 'Patched Conic Approximation' technique. One must chose the one domainant gravitating body in each region of space through which the trajectory will pass, and to model only that body's effects in that region. For instance, on trajectory from Earth to Mars, one woul dbegin by considering only the Earth's gravity until the trajectory reaches a distance where the Earth's gravity no longer dominates that of the sun. The spacecraft would be given escape velocity to send it on its way to interplanetery space. Next, one would consider only the Sun's gravity until the trajectory reaches the neighbourhood of Mars. During this stage, the transfer orbit model is appropriate. Finally, only Mars's gravity is considered during the final portion of the trajectory where Mars's gravity dominates the spacecraft's behaviour. The spacecraft would approach Mars on a hyperbolic orbit, and a final retrograde burn would slow the spacecraft enought to be captured by Mars.

The size of the "neighborhoods" (or spheres of influence) vary with radius \(r_{SOI}\):

\[R_{SOI} = a_p{\left(\frac{m_p}{m_s}\right)}^{\frac{2}{5}}\]

where \(a_p\) is the semimajor axis of the planet's orbit relative to the Sun; \(m_p\) and \(m_s\) are the masses of the planet and Sun, respectively.

This simplification is sufficient to compute rough estimates of fuel requirements, and rough time-of-flight estimates, but it is not generally accurate enough to gude a spacecraft to its destination. For that, numerical methods are required.

\subsection{The universal variable formulation}

To address computational shortcomings of traditional approaches for solving the 2-body proble, the universal variable formulation was developed. It works equally well for circular, elliptical, parabolic, and hyperbolic cases, the differential equations converging well when integrated for any orbit. It also generalizes well to problems incoprorating perturbation theory.

\subsection{Perturbation}

The universal variable formulation works well with the variation of parameters technique, except now, insetead of the six Keplerian orbital elements, we use a different set of orbital elements: namely, the stallite's initial position and velocity vector's \(x_0\) and \(v_0\) at a given epoch \(t=0\). in a two-body simulation, these elements are sufficient to compute the satellite's position and velocity at any time in the future, using the universal variable formulation. Conversely, at any moment in the satellite's orbit, we can measure its positoin and velocity, and then use the universal variable approach to determin what is initial position and velocity would have been at the epoch. In perfectwo body motion, these orbital elements would be invariant (just like the Keplerian elements would be).

However, pertubations cause the orbital elements to change over time. Hence, we write the position element as \(x_0(t)\) and the velocity element as \(v_0(t)\), indicating that they vary with time. The technique to compute the effect of pertubations becomes one of finding expressions, either exact or approximate, for the functions \(x_0(t)\) and \(v_0(t)\).

The following are some effects which make real orbits differ from the simple models based on a spherical Earth. Most of them can be handled on short timescales (perhaps less than a few thousand orbits) by pertubation theory because they are small relative to the corresponding two-body effects.

\begin{itemize}
  \item Equatorial bulges cause precession of the node and the perigee
  \item Tesseral harmonics of the gravity field introduce additional pertubations
  \item Lunar and solar gravity pertubations alter the orbits
  \item Atmospheric drag reduces the semi-major axis unless make-up thrust is used
\end{itemize}

Over very long timescales (perhaps millions of orbits), even small pertubations can dominate, and the behaviour can become chaotic. On the other hand, the various pertubations can be orchestrated by clever astrodynamicists to assist with orbit maintenance tasks, such as station-keeping, ground track maintenance or adjustment, or phasing of perigee to cover selected targets at low altitude.

\section{Orbital maneuver}

In spaceflight, an orbital maneuver is the use of propulsion systems to change the orbit of a spacecraft. For spacecraft far from Earth--for example those in orbits around the Sun-- on orbital maneuver is called a deep-space maneuver (DSM).

\subsection{Orbital transfer}

Tansfer orbits are usually elliptical orbits that allow spacecraft ot move from one (usually substantially circular) orbit to another. Usually they require a burn at the start, a burn at the end, and sometimes one or more burns in the middle.

\begin{itemize}
  \item The Hohmann transfer orbit requires a minimal delta-v.
  \item A bi-elliptic transfer can require less energy than the Hohmann transfer, if the ratio of orbits is \(11.94\) or grater, but omes at the cost of increased trip time over the Hohmann transfer.
  \item Faster transfers may use any orbit that intersects both the original and destination orbits, at the cost of hight delta-v.
  \item Using low thrust engines (such as electrical propulsion), if the initial orbit is sypersynchronous to the final desired circular orbit then the optimal transfer orbit is achieved by thrusting continously in the direction of the velocity apogee. This method however takes much longer due to the low thrust.
\end{itemize}

For the case of orbital transfer between non-coplanar orbits, the change-of-plane thrust must be made at the point where the orbital planes intersect (the "node").

\subsection{Gravity assist and the Oberth effect}

In a gravity assist, a spacecraft swings by a planet and leaves in a different direction, at a different speed. This is useful to speed or slow a spacecraft instead of carrying more fuel.

This maneuver can be approximated by an elastic collision at large distances, though the flyby does not invlove any physical contact. Due to Newton's Third Law (equal and opposite reaction), any momentum gained by a spacecraft must be lost by the planet, or vice versa. However, because the planet is much, much more massive than the spacecraft, the effect on the planet's orbit is negligible.

The Oberth effect can be employed, particularly during a gravity assist operation. This effect is that use of a propulsion system works better at high speeds, and hence course changes are best done when close to a gravitation body; this can multiply the effective delta-v.

\subsection{Interplanetary Transport Network and fuzzy orbits}

It is now possible to use computers to search for routes using the nonlinearities in the gravity of the planets and moons of the Solar System. Collectively referred to as the Interplanetary Transport Network, these highly pertubative, even chaotic, orbital trajectories in principle need no fule beyond that needed to reach the Lagrange point (in practice keeping to the trajectory requires some course corrections). The biggest problem with them is they can be exceedingly slow, taking many years. In addition launch windows can be very far apart.

They have, however, been employed on projects such as Genesis.

\end{document}
