\documentclass{article}
\usepackage{multicol}
\title{Mathematic Formulation of the Double-Precision Orbit Determination
Program}
\author{Arden Rasmussen}

\begin{document}

\pagenumbering{gobble}

\maketitle

\newpage

\pagenumbering{roman}

\begin{abstract}
  This documents the complete mathematical model for the Double-Precision Orbit
  Determination Program (DPODP), a third-generation program which has recently
  been completed at the Jet Propulsion Laboratory. The DPODP processes
  earth-based doppler, range, and angular observable of the spacecraft to
  determine values of the parameters that specify the spacecraft trajectory for
  lunar and planetary missions.
\end{abstract}

\newpage

%\newpage

\tableofcontents

%\newpage
\newpage

\pagenumbering{arabic}

\section{Introduction}
\label{sec:introduction}

\begin{multicols}{2}
  This report documents the mathematical model for the Double-Precision Orbit
  DeteDetermination Program (DPODP). The DPODP will be used to determine values
  of the parameters that specify the spacecraft trajectory for lunar and
  planetaryplanetary missions; it will be used for both real-time and post-flight
  reduction of tracking data. The DPODP differentally corrects \textit{a priori}
  estimaes of injection parameters, physical constants, maneuver parameters, and
  station locations to minimize the sum of weighted squares of residual errors
  betbetween observed and computed quantities.
  
  The basic limitations on the \textit{accuracy of computed observables} are the
  inaccuracies in the troposphere and ionosphere corrections. Before these
  corrections are added, the computed values of the doppler and range observales
  have accuracies of \(10^{-5}m/s\) and \(0.1m\), respectivly. The parameters
  whose values may be estimated by the DPODP are:
  
  \begin{enumerate}
    \item Injection parameters. Rectangular components of the spacecraft position
      and velocity vectors at the injection epoch.
    \item Reference parameters. Parameters that affect the relative position and
      velocity of the sun, palents, and the moon:
      \begin{enumerate}
        \item[\(A_E\)] The number of kilometers per astronomical unit (AU). This
          parameter converts the precomputed eliocentric ephemrides of eight
          planeplanets and the earth-moon barycenter from astronomical units to
          kilometers.
        \item[\(R_E\)] Scaling factor for lunar ephemeris, km/fictitious earth
          radius. This factor converts the precomputed geopcentric lunar
          ephemeris from fictitious earth radii to kilometers.
        \item[\(E\)] Osculation orbital elements for the precomputed ephemeris of
          a planet, earth-moon barycenter, or the moon, The estimated correction
          \(\Delta E\) is used to differentially correct position and velocity
          obtobtained from the precomputed ephemeris.
        \item[\(\mu_E,\mu_M\)] Gravitational constants for the earth and moon,
          \(km^3/s^2\). These parameters affect the location of the earth-moon
          barycenter.
      \end{enumerate}
    \item Gravitational constats. The constant \(\mu_i\) is the gravitational
      constant for body \(i\), such as the sun, a planet, or the moon. (Note
      thathat \(\mu_E\) and \(\mu_M\) are also listed under reference
      parametersparameters.)
    \item Harmonic coefficient. The harmonic coefficients \(j_n\), \(C_{nm}\),
      \(S_{nm}\), along with the gravitational constant \(\mu\), describe the
      gravitational field of a planet or moon.
    \item Parameters affecting the acceleration of the spacecraft due to solar
      radiation pressure.
    \item Coefficient of quadratic for small acceleration acting along each
      spacecraft axis. These quadratics are used to represent gas leaks and
      small forces arising from operation of the attitude control system.
    \item Parameters affecting spacecraft motor burns.
    \item Parameters affecting the transformation from universal time to
      ephemeris time.
    \item Coefficient of quadratics which represent the departure of atomic
      time at each tracking station from broadcast UTC time.
    \item Station parameters.
      \begin{enumerate}
        \item Radius
        \item Latitude
        \item Longitude
      \end{enumerate}
      or
      \begin{enumerate}
        \item  Distance from spin axis
        \item Height above equator
        \item Longitude for each tracking station and a land spacecraft on a
          plaplanet or the moon.
      \end{enumerate}
      For a tracking ship:
      \begin{enumerate}
        \item Spherical coordinates at an epoch
        \item velocity
        \item azimuth
      \end{enumerate}
    \item Speed of light. And adopted constant which defines the light-second
      as the basic length unit; it is not normally included in the solution
      vector.
    \item Constant bias for range observables.
    \item Spacecraft transmitter frequency for one-way doppler.
    \item Biases affecting observed angles.
    \item Relativity parameter \(\gamma\). This parameter will be added to the
      program. It is equal to \((1+\omega)/(2+\omega)\) where \(\omega\) is the
      coupling constant of the scalar field, a free parameter of the Brans-Dike
      theory of gravitation.
  \end{enumerate}

  Given the \textit{a priori} estimate of the parameter vector \(q\), the program
  integrates the spacecraft acceleration using the second-sum numerical
  integration method to give position and velocity at any desired time. Using
  the spacecraft ephemeris along with the precomputed ephemerides for the other
  bodies within the solar system, and the parameter vector \(q\), the program
  computes values for each observed quantity (normally doppler, range, or
  angles) and forms the \textit{observed} minus \textit{computed} (O-C)
  residuals.

  In addition to integrating the acceleration of the spacecraft to obtaini the
  spacecraft ephemeris, the program integrates the partial derivative fo the
  spaspacecraft acceleration with respect to (wrt) the parameter vector \(q\)
  using the second-sum numerical integration procedure to give the partial
  derivderivative of the spacecraft state vector \(X\) (position and velocity
  components) wrt the parameter vector \(q\), \(\partial X/\partial q\). Using
  \(\partial X/\partial q\), thee program computes the partial derivative of
  each computed observalbe quantity \(z\) wrt \(q\), \(\partial z/\partial q\).
  Given the O -- C residuals, \(\partial z/\partial q\), and the weights
  applied to each residual along with the \textit{a priori} parameter vector
  andand its convariance matrix, the program computes the differential
  correction \(\Delta q\) to the parameter vector. Starting with \(q + \Delta
  q\), the program computess a new spacecraft ephemeris, residuals, and the
  partpartial derivatives and obtains a second differential correction \(\Delta
  q\). This process is repeated until convergence is obtained and the sum of
  weighted squares of residual errors between observed and computed quantities
  is minimized.

  The DPODP formulation was heavily influenced by the general theory of
  relativity. \ref{sec:relativistic_terms_of_dpodp_formulation} gives the equations from general relativity, which are
  the basis of the DPODP formulation, and also the principle relativistic
  equaitons containd in the formulation.

  The time transformations used throught the program and formulation for
  computing thre relative position, velocity, acceleration, and jerk for any
  two celestial bodies are described in \ref{sec}.
\end{multicols}

\section{Relativistic Terms of DPODP Formulation}
\label{sec:relativistic_terms_of_dpodp_formulation}

\begin{multicols}{2}
  In the general theory of relativity is basically a geometrical theory. The
  geometry is embodied in the components of the symmetrical metrict tensor
  \(g_{pq}\): 
  \begin{equation}
  g_{pq}=\left[\begin{array}{cccc}
    g_{11} & g_{12} & g_{13} & g_{14}\\
    g_{21} & g_{22} & g_{23} & g_{24}\\
    g_{31} & g_{32} & g_{33} & g_{34}\\
    g_{41} & g_{42} & g_{43} & g_{44}\\
  \end{array}\right]
  \end{equation}
  The subscripts \(1\), \(2\), \(3\), and \(4\) correspond to the space-time
  coordinates \(x^1\), \(x^2\), \(x^3\), and \(x^4\), which are associated with
  a particular space-time frame of reference.
\end{multicols}

\end{document}
